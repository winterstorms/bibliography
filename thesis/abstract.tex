Gyárfás and Sumner both conjectured that for each acyclic graph $F$, there exists a bound on the chromatic number depending only on a graphs' clique number such that any graph with larger chromatic number has an induced subgraph isomorphic to $F$. The conjecture remains open even if it was proven for certain families of trees.

This thesis summarizes the cases which are already known and presents a result by Kierstead and Rödl for graphs without a $K_{n,n}$. Furthermore, we prove a weaker result for graphs where any two vertices share a limited number of neighbors. Afterwards, we consider finding a counterexample for the Gyárfás-Sumner conjecture by constructing triangle-free graphs with arbitrarily large chromatic number. Therefore, we look at the Mycielskian construction and the Burling construction and show that for any tree $T$ both constructions yield graphs containing induced subgraphs isomorphic to $T$ and thus do not contradict the conjecture.