Making statements about induced subgraphs in a graph based on nothing but its chromatic number poses a great challenge. Erd\H{o}s and Hajnal proved that there for all graphs $H$ containing a cycle there exists graphs of arbitrarily large chromatic number not containing an induced subgraph isomorphic to $H$. Gyárfás and Sumner both conjectured that for each acyclic graph $F$, there exists a bound on the chromatic number depending only on a graphs clique number such that any graph with larger chromatic number has an induced subgraph isomorphic to $F$. This remains open even if the conjecture was proven for certain families of trees.

This thesis summarizes the cases which are already known and presents a result by Kierstead and Rödl for graphs without a $K_{n,n}$. Furthermore, we prove a weaker result for graphs with a certain girth and minimum degree. Afterwards, we consider finding a counterexample for the Gyárfás-Sumner conjecture by constructing triangle-free graphs with arbitrarily large chromatic number. Therefore, we look at the Mycielskian construction and the Burling construction and show that for any tree $T$ both constructions yield graphs containing induced subgraphs isomorphic to $T$ and thus do not contradict the conjecture.