% Vorlage für eine Bachelorarbeit
% Siehe auch LaTeX-Kurs von Mathematik-Online
% www.mathematik-online.org/kurse
% Anpassungen für die Fakultät für Mathematik
% am KIT durch Klaus Spitzmüller und Roland Schnaubelt Dezember 2011

\documentclass[12pt,a4paper]{scrartcl}
% scrartcl ist eine abgeleitete Artikel-Klasse im Koma-Skript
% zur Kontrolle des Umbruchs Klassenoption draft verwenden


% die folgenden Packete erlauben den Gebrauch von Umlauten und ß
% in der Latex Datei
\usepackage[utf8]{inputenc}
\usepackage[T1]{fontenc}
\usepackage[english]{babel}
\usepackage{verbatim}

\usepackage{graphicx}
\graphicspath{{images/}}
\usepackage{latexsym}
\usepackage{amsmath,amssymb,amsthm}
\usepackage{extarrows}

\usepackage[shortlabels]{enumitem}
\usepackage{cite}

% Abstand obere Blattkante zur Kopfzeile ist 2.54cm - 15mm
\setlength{\topmargin}{-15mm}

%nach Absaetzen Hoehe eines 'x' Abstand
%\setlength{\parskip}{0.5ex}


% Umgebungen für Definitionen, Sätze, usw.
% Es werden Sätze, Definitionen etc innerhalb einer Section mit
% 1.1, 1.2 etc durchnummeriert, ebenso die Gleichungen mit (1.1), (1.2) ..
\theoremstyle{plain}
\newtheorem{thm}{Theorem}[section]
\newtheorem{con}[thm]{Conjecture}
\newtheorem{lemma}[thm]{Lemma}
\newtheorem{cor}{Corollary}[thm]

\theoremstyle{definition}
\newtheorem{defn}[thm]{Definition} 
\newtheorem*{prf}{Proof}
\newtheorem{notn}[thm]{Notation}
\newtheorem{note}[thm]{Observation}

\renewenvironment{proof}{{\bfseries Proof}}{\qed}
\numberwithin{equation}{section} 

\newtheoremstyle{case}{}{}{}{}{}{:}{ }{}
\theoremstyle{case}
\newtheorem{case}{Case}


% einige Abkuerzungen
\newcommand{\C}{\mathbb{C}} % komplexe
\newcommand{\K}{\mathbb{K}} % komplexe
\newcommand{\R}{\mathbb{R}} % reelle
\newcommand{\Q}{\mathbb{Q}} % rationale
\newcommand{\Z}{\mathbb{Z}} % ganze
\newcommand{\N}{\mathbb{N}} % natuerliche

\usepackage{xpatch}
\makeatletter
\AtBeginDocument{\xpatchcmd{\@thm}{\thm@headpunct{.}}{\thm@headpunct{}}{}{}}
\makeatother


\begin{document}
  % Keine Seitenzahlen im Vorspann
  \pagestyle{empty}

  % Titelblatt der Arbeit
  \begin{titlepage}

\begin{figure}[h]
\includegraphics[scale=0.5]{KITLogo_RGB_eng} 
   %\includegraphics[scale=0.45]{kit-logo} 
\end{figure}
  
    \vspace*{2cm} 

 \begin{center} \large 
    
    Bachelor Thesis
    \vspace*{2cm}

    {\huge Bounding Chromatic Number of Graphs with Forbidden Induced Subgraphs}
    \vspace*{2.5cm}

    Frithjof Marquardt
    \vspace*{1.5cm}

    2nd April 2018
    \vspace*{3cm}
  
    \begin{tabbing}
      \hspace*{10em}\= \hspace*{6em} \= \kill % set the tabbings
      \> Reviewer:\> Prof. Maria Axenovich \\
      \> Advisor:  \> Yelena Yuditsky\\[0.5cm]
    \end{tabbing}
      
    Faculty of Mathematics \\[1cm]
  
		Karlsruhe Institute of Technology
  \end{center}
\end{titlepage}

 %\vspace*{5cm}
\begin{center}
Abstract
\end{center}
Making statements about induced subgraphs in a graph based on nothing but its chromatic number poses a great challenge. Erd\H{o}s and Hajnal proved that there for all graphs $H$ containing a cycle there exists graphs of arbitrarily large chromatic number not containing an induced subgraph isomorphic to $H$. Gyárfás and Sumner both conjectured that for each acyclic graph $F$, there exists a bound on the chromatic number depending only on a graphs clique number such that any graph with larger chromatic number has an induced subgraph isomorphic to $F$. This remains open even if the conjecture was proven for certain families of trees.

This thesis summarizes the cases which are already known and presents a result by Kierstead and Rödl for graphs without a $K_{n,n}$. Furthermore, we prove a weaker result for graphs with a certain girth and minimum degree. Afterwards, we consider finding a counterexample for the Gyárfás-Sumner conjecture by constructing triangle-free graphs with arbitrarily large chromatic number. Therefore, we look at the Mycielskian construction and the Burling construction and show that for any tree $T$ both constructions yield graphs containing induced subgraphs isomorphic to $T$ and thus do not contradict the conjecture.
  \newpage
  % Inhaltsverzeichnis
  \tableofcontents

  \newpage
 


  % Ab sofort Seitenzahlen in der Kopfzeile anzeigen
  \pagestyle{headings}

  \section{Introduction}\label{secIntro}
  Proving the conjecture as a whole is obviously a complex endeavor. That is why many proofs concentrate on certain graphs and prove it only partially. Therefore,


Finally, Chapter \ref{secConc} provides a summary of the results we proved throughout this thesis and gives an outlook which cases of the Gyárfás-Sumner conjecture till remain open.
  \newpage
  
  \section{Basic Concepts}\label{secBC}
  This chapter focuses on providing the basis for all following theorems and proofs by displaying most of the definitions regularly used in this thesis. They are collected and structured starting with the simple ones and advancing to more complex and specific concepts. This should give an overview and a better understanding of the methods we will use later on.\\

We will start with defining the concept of a graph. For a set $X$, let us denote the set of all $2$-element subsets of $X$ with ${X\choose 2} =\lbrace \lbrace x,y\rbrace : x,y\in V, x\neq y\rbrace$. A \textit{graph} $G$ is a tuple $G=(V,E)$, where $V=V(G)$ denotes the set of vertices and $E=E(G)\subseteq {V\choose 2}$ is the set of edges. Here, each edge is a set of two vertices from $V$.

By giving each edge a direction, i.e. saying the edge $(x,y)$ points from $x$ to $y$, we create a \textit{directed graph} $D=(V,E)$ where $V$ again denotes the set of vertices but $E\subseteq \lbrace (x,y) : x,y\in V, x\neq y\rbrace$ is now a set of tuples. In a directed graph, there can exist two edges between two vertices $x$ and $y$: one pointing to $x$ and one pointing to $y$. We call a directed graph $D$ also an \textit{oriented graph} if there exists at most one edge for any pair of vertices in $D$. See Figure \ref{f1ba} for an illustration of the differences between undirected, directed and oriented graphs. 

Note that any of the following concepts and definitions can be applied to directed graphs as well.

\begin{figure}[ht]
\begin{center}
\includegraphics[scale=1]{undir_dir_ori}
\end{center}
\caption{Undirected graph (left), directed graph (center) and oriented graph (right) on the same vertex set}
\label{f1ba}
\end{figure}

In a graph $G=(V,E)$ two vertices $u,v\in V$ are \textit{adjacent} if $\{u,v\}\in E$. We then also call them neighbors and say the edge is \textit{incident} to the vertices. We denote the neighbors of a vertex $v\in V$ in $G$ as $N_G(v)$ or just $N(v)$ if there is no confusion in regard of the underlying graph. Furthermore, the \textit{degree} $d(v)$ of $v$ is defined as $d(v) =\vert N(v)\vert$. The minimum degree $\delta (G)$ in a graph is defined as $\displaystyle\delta (G)=\min_{v\in V}d(v)$ and the maximum degree $\Delta (G)$ as $\displaystyle\Delta (G)=\max_{v\in V}d(v)$. Since edges in a directed graph do not necessarily all point in the same direction, there we distinguish between the \textit{in-degree} $d^{in}(v)$ of a vertex $v$ which counts the number of edges ending in $v$ and the \textit{out-degree} $d^{out}(v)$ which counts the number of edges starting at $v$. This concept also applies to the minimum and maximum degree of a directed graph.\\

After introducing the concept of a graph and the relations between edges and vertices we now start looking for structures of a greater scale in it either by considering just the vertices and edges or later by additionally defining colorings of the graph. 

Let $G=(V,E)$ be a graph. For a vertex set $A\subseteq V$ let us denote the set of edges in $B\subseteq E$ induced by $A$ as $B\vert_A=\lbrace \lbrace x,y\rbrace\in B : x,y\in A\rbrace$. Then, a graph $H=(V^\prime ,E^\prime )$ is a \textit{subgraph} of $G$, written $H\subseteq G$, if $V^\prime\subseteq V$ and $E^\prime\subseteq E\vert_{V^\prime}$. The graph $H$ is called an \textit{induced subgraph} of $G$ and we write $H\subseteq_I G$ if $E^\prime = E\vert_{V^\prime}$. The subgraph $(A,E\vert_A)$ induced in $G$ by a certain set of vertices $A\subseteq V$ is denoted as $G[A]$. If $A=V\setminus \{v\}$ for a single vertex $v\in V$ we often write just $G-\{v\}$ for the subgraph induced by $A$. Furthermore, the structures of two graphs can coincide completely: Two graphs $G$ and $H$ are called \textit{isomorphic} if there exists a bijection $\phi :V(G)\to V(H)$ such that $\lbrace\phi (u), \phi (v)\rbrace\in E(H)$ if and only if $\lbrace u, v\rbrace\in E(G)$. We denote such a relation as $G\approx H$.

A subset $S\subseteq V$ of pairwise non-adjacent vertices is called a \textit{stable} or \textit{independent} set. Opposite to that, a \textit{clique} is a set of vertices $Q\subseteq V$ such that all vertices in $Q$ are pairwise adjacent. Observe that for a stable set $S$ $E\vert_S =\emptyset$ holds while we have $E\vert_Q={Q\choose 2}$ for a clique $Q$. The size of the largest clique in $G$ is denoted by $\omega (G)$. A \textit{coloring} $c$ of $G$ with $r$ colors is a function $c:V\to [r]$ that maps a color to each vertex. It is a proper coloring if for all edges $e = \lbrace x,y\rbrace\in E$ $c(x)\neq c(y)$, i.e. no two adjacent vertices have the same color. Then, we can define the chromatic number $\chi (G)$ of $G$ as the minimum number of colors needed for a proper coloring of $G$. Observe that each color class of a proper coloring, i.e. all vertices of the same color, is a stable set. If $G$ is a directed graph, its chromatic number is defined as the chromatic number of the underlying undirected graph, i.e. the undirected graph on the same vertex set, where the edge set contains an edge $\{x,y\}$ if and only if $(x,y)$ or $(y,x)$ or both are in the edge set of $G$.

A way of obtaining a proper coloring for a graph $G$ that always works is the \textit{greedy coloring}: Order the vertices $v_1, v_2, \dots ,v_{\vert V(G) \vert }$ and define a coloring $c$ by assigning each vertex the first available color, i.e. the first color not used on a neighbor with smaller index. Note that it often is far from optimal regarding the number of used colors.\\

Since the crucial part of this thesis deals with the conjecture of Gyárfás and Sumner, we want to introduce the notion of $\chi$-bounded families of graphs. In addition, this part states the definitions of a hypergraph and the Ramsey number because they are used in many proofs in the following chapters.

\begin{defn}[Bounding function]\label{d1}
A $\chi$\textit{-bounding function} $f:\N\to\N$ for a family $\mathcal{F}$ of graphs is a function such that $\chi (G)\leq f\big(\omega (G)\big)$ holds for each $G\in\mathcal{F}$. Such a function does not necessarily exist for each family $\mathcal{F}$. If it exists though, $\mathcal{F}$ is called $\chi$-bounded. 
\end{defn}

A \textit{hypergraph} $H=(V,E)$ is a tuple where $V$ is a set of vertices and $E$ a set of non-empty subsets of $V$, called hyperedges. If each edge has the same size $s$, $H$ is an $s$-uniform hypergraph. Note that the concepts of colorings also apply for hypergraphs. We define a coloring $c$ of the hypergraph $H$ with $r$ colors as a function $c:V(H)\to [r]$ that assigns each vertex a color. A hypergraph coloring $c$ is proper if there exists no monochromatic edge, i.e. there is no edge $e\in E(H)$ with $c(v)=\alpha$ for all $v\in e$ and some color $\alpha\in [r]$.

The Ramsey number $R=R(r_1, r_2, \dots ,r_s)$ for integers $r_1, r_2, \dots r_s$ is the minimum number $R\in\N$, such that for every coloring $c:{[R]\choose 2}\to [s]$, there exist a color $\alpha\in [s]$ and a subset $X\subseteq [R]$, $\vert X\vert =r_{\alpha}$, with $c(\lbrace i,j\rbrace )=\alpha$ for all $\lbrace i,j\rbrace\in {X\choose 2}$. Note that the definition of the Ramsey number is based on edge colorings while the colorings of graphs we defined so far are vertex colorings.\\

The next definitions are a collection of the most used special graphs or characterizations of some graphs that appear throughout this thesis.

Consider a graph $G=(V,E)$ with $\vert V\vert = n$. If $E=\emptyset$, $V$ induces a stable set and we call $G$ the empty graph on $n$ vertices. If $E= {V\choose{2}}$, $V$ induces a clique of size $n$ and $G$ is also denoted as $K_n$, the complete graph on $n$ vertices. The graph $G$ is called bipartite if $V=A\dot{\cup} B$ for two disjoint stable sets $A$ and $B$. The complete bipartite graph $K_{m,n}$ is a bipartite graph with $\vert A\vert =m$, $\vert B\vert =n$ and $E=\lbrace \lbrace a,b\rbrace :a\in A,b\in B\rbrace$.

A \textit{path} of length $n\in\N$ is a graph $P=(V,E)$ on $n+1$ vertices, where we can order the vertices $v_0, v_1, \dots ,v_n$ such that $E=\lbrace\lbrace v_{i-1},v_i\rbrace : i\in [n]\rbrace$. We say it starts in $v_0$ and ends in $v_n$. We obtain a \textit{cycle} $C=(V,E^\prime)$ of length $n$ from a path $P=(V,E)$ of length $n$ by adding the edge $\lbrace v_0,v_n\rbrace$ to the edge set, i.e. $E^\prime = E\cup\lbrace v_0,v_n\rbrace$. The \textit{girth} of a graph $G$ is the length of the shortest cycle contained in $G$. A graph that does not contain a cycle as a subgraph is \textit{acyclic} and has infinite girth. We say a graph $G=(V,E)$ is \textit{connected}, if there exist a path starting at $u$ and ending at $v$ for each pair of vertices $u,v\in V$.

With the previous definition we define a \textit{tree} $T$ as a connected and acyclic graph. A union of disjoint trees is called a \textit{forest}. Note that in a tree vertices of degree one are called \textit{leaves}.\\

Additionally, we can group trees according to their form. We denote a \textit{path} on $n\in\N$ vertices as $P_n$. In the special case $n=1$, $P_1$ contains just a single vertex and no edge. A \textit{star} $K_{1,n}$ on $n+1$ vertices is a tree with one center vertex of degree $n$ and $n$ leaves. A \textit{caterpillar} is a tree $T$ such that if we remove all leaves from $T$ we obtain a subgraph $H$ isomorphic to a path. Observe that then $H$ contains exactly the vertices in $T$ of degree at least two. Figure \ref{f3ba} gives an example of all three types of trees.

\begin{figure}[ht]
\begin{center}
\includegraphics[scale=1]{trees}
\end{center}
\caption{A path $P$, a star $S$ and a caterpillar $T$}
\label{f3ba}
\end{figure}

A \textit{subdivision} of a graph $G$ is a graph $G_k$, $k\in\N$, such that there exists a sequence of graphs $G=G_0,G_1,\dots ,G_k$ where $G_i$ can be obtained from $G_{i-1}$ by an edge subdivision for all $i\in [k]$. Subdividing an edge $e=\lbrace u,v\rbrace\in E$ in a graph $G=(V,E)$ yields a graph $G^\prime =(V\dot{\cup} \lbrace w\rbrace, E^\prime )$ with $E^\prime =(E\setminus \lbrace e\rbrace ) \cup \lbrace \lbrace u,w\rbrace, \lbrace v,w \rbrace\rbrace$. Informally, we add a new vertex $w$ and replace $e$ with two new edges $\lbrace u,w\rbrace$ and $\lbrace w,v\rbrace$ as depicted in Figure \ref{f2ba}. Observe that a subdivision of a tree is still a tree.\\

\begin{figure}[ht]
\begin{center}
\includegraphics[scale=1]{subdivision}
\end{center}
\caption{Subdivision of an edge $e$ in a graph $G$}
\label{f2ba}
\end{figure}


For the sake of the completeness of this collection of definitions, we add a final one, since it is used to describe some trees already proven to be $\chi$-bounding. Again, let $G=(V,E)$ be a graph. Let the \textit{distance} $dist(u,v)$ between two vertices $u,v\in V$ denote the length of the shortest path in $G$ starting at $u$ and ending at $v$. The \textit{eccentricity} of a vertex $v\in V$ is defined as $\displaystyle e(v)=\max_{u \in V}\lbrace dist(v,u)\rbrace$. Then, the \textit{radius} $\displaystyle r= \min_{v\in V}\lbrace e(v)\rbrace$ of $G$ is the minimum number $r\in\N$, such that there exists a vertex $v\in V$ where the distance between $v$ and any other vertex $u\in V$ is at most $r$. We often name this vertex $v$ the root of the tree.
  \newpage 
  
  \section{$\chi$-Bound Families of Graphs}\label{secCR}
  For a graph $H$ define the family of graphs not containing an induced subgraph isomorphic to $H$ as $\textit{Forb}(H)$. For which $H$ can we find a function $f$ such that $\chi (G)\leq f(\omega (G))$ holds for every graph $G\in\textit{Forb}(H)$? I.e., by the notation in \ref{d1}, we want to find a bounding function for $\textit{Forb}(H)$. If $f$ exists, let us call $H$ $\chi $-bounding. Observe that if $H$ is $\chi $-bounding, $H$ must be acyclic. That is because of a result by Erd\H{o}s and Hajnal in \cite{EH66} showing the existence of graphs with arbitrarily large girth and chromatic number. Thus, if $H$ contains a cycle, we can always find a graph $G$ with girth greater than the longest cycle in $H$ and a large chromatic number. Therefore, it is impossible to find a bounding function for $\textit{Forb}(H)$. For every acyclic graph $H^\prime$ the family $\mathcal{F}(H^\prime )$ may be $\chi$-bounded. Indeed, Gyárfás \cite{Gy75} and Sumner \cite{Su81} independently conjectured that it is true for every forest.

\begin{con}[Gyárfás \cite{Gy75}, Sumner \cite{Su81}]\label{c1cr}
Every forest $F$ is $\chi$-bounding.
\end{con}

Actually, Gyárfás and Sumner conjectured that for every clique $K$ and tree $T$ the family of graphs inducing neither $K$ nor $T$ is $\chi$-bounded. Martin \cite{Ma16} mentions the general idea for proving that both statements are equivalent. He claims that the problem as formulated in the Conjecture \ref{c1cr} can be reduced to trees since a forest is $\chi$-bounding if and only if all its connected components are $\chi$-bounding. Then, \ref{c1cr} just states the Gyárfás-Sumner conjecture using Definition \ref{d1}. We give a proof of this claim.

\begin{thm}
A forest $F$ is $\chi$-bounding if and only if all its connected components are $\chi$-bounding.
\end{thm}

\begin{prf}
Let $F$ be a $\chi$-bounding forest with a bounding function $f$. Assume there exists a connected component $T$ of $F$ that is not $\chi$-bounding. Then, for all functions $g:\N\to\N$ there exist an $\omega\in\N$ and a graph $G\in \textit{Forb}(T)$ with $\omega (G)=\omega$ and $\chi (G)>g(\omega)$ by Definition \ref{d1}. Now let $g:=f$. Since any graph in $\textit{Forb}(T)\subseteq\textit{Forb}(F)$, there exist an $\omega\in\N$ and a graph $G\in\textit{Forb}(F)$ with $\omega (G) =\omega $ and $\chi (G)>f(\omega )$. This is a contradiction to $F$ being $\chi$-bounding.

To show the other direction, let $F$ be a forest with connected components $T_1, T_2,\dots ,T_k$ where $\textit{Forb}	(T_i)$ is $\chi$-bounded with a bounding function $f_i$, $i\in [k]$. Assume, $F$ is not $\chi$-bounding, i.e. for all functions $g:\N\to\N$ there exist an $\omega\in\N$ and a graph $G\in \textit{Forb}(F)$ with $\omega (G)=\omega$ and $\chi (G)>g(\omega)$. Let $g$ be a function with $g(\omega )\geq\max_{i\in [k]}f_i(\omega )$ for all $\omega\in\N$ and choose a graph $G\in\textit{Forb}(F)$ with $\chi (G)>g(\omega)$. Then, there exists an $i\in [k]$ such that $G\in\textit{Forb}(T_i)$ and $\chi (G)>g(\omega )\geq f_i(\omega )$. This contradicts the assumption that $T_i$ is $\chi$-bounding.\qed
\end{prf}

Let $\mathcal{H}$ be a set of graphs and let $\textit{Forb}(\mathcal{H})$ be the family of graphs forbidding any graph in $\mathcal{H}$ as an induced subgraph. The Conjecture \ref{c1cr} covers only cases where $\vert\mathcal{H}\vert =1$.
Now consider a family $\textit{Forb}(\mathcal{H})$ with $\vert\mathcal{H}\vert >1$. Note that every finite set containing a forest is $\chi$-bounding under the assumption that Conjecture \ref{c1cr} is true and if $\mathcal{H}$ is finite and $\chi$-bounding it must contain a forest because of the result by Erd\H{o}s and Hajnal \cite{EH66}. If $\mathcal{H}$ does not contain a forest, it has to be infinite. Thus, considering infinite sets of cycles is the next obvious step. In 1985, Gyárfás made three more conjectures (later published in \cite{Gy87}) concerning the relation between the chromatic number of a graph and the length of cycles it induces.
\begin{con}[Gyárfás \cite{Gy87}]
The following three statements hold.
\begin{enumerate}[(i)]
\item The family of graphs with no induced cycle of odd length is $\chi$-bounded.
\item Let $l\in\N$. The family of graphs with no induced cycle of length at least $l$ is $\chi$-bounded.
\item Let $l\in\N$. The family of graphs with no induced cycle of odd length at least $l$ is $\chi$-bounded.
\end{enumerate}
\end{con}

Observe that the third conjecture implies the first two but since they were stated and proven independently each seems to be worth a nomination on its own. Chudnovsky, Scott and Seymour dealt with these conjectures in a series of papers including three (\cite{SS14}, \cite{CSS15} and \cite{CSSS17}) in which they are proving them.
\\

Returning to the case were $\vert\mathcal{H}\vert =1$, it might be interesting to note that the conjecture becomes a rather simple problem in the case of general subgraphs instead of induced subgraphs. There exists a proof of an even stronger result by Gyárfás, Szemeredi and Tuza from 1980 \cite{GST80}. They show that a labeled tree $T$ on $k$ vertices is a subgraph of any labeled graph with chromatic number at least $k$. A \textit{labeled} graph is a graph with an assignment of labels, often represented by integers, to its vertices. When talking about a labeled graph $G$ without defining an explicit labeling, we refer to a graph labeled such that each vertex has a unique label. Usually each vertex has a label $i\in [\vert V(G)\vert ]$. A pair of labeled graphs is isomorphic if there exists a graph isomorphism that also heeds the assigned labels besides the structure of the graph.

An obvious consequence of this theorem is that the family of graphs that do not contain $T$ as a subgraph is $\chi$-bounded and for the bounding function $f$ holds $f(\omega )\equiv k$.

\begin{thm}[Gyárfás, Szemeredi, Tuza \cite{GST80}]\label{t4cr}
Let $k\in\N$ and $G$ be a labeled graph with $\chi (G) = k$. Then, $G$ contains every labeled tree $T$ on $k$ vertices as a subgraph. 
\end{thm}
\begin{prf}
We proof this by induction on $k$. For the base case let $k=1$. A graph $G$ with $\chi (G)=1$ trivially contains a labeled tree on a single vertex.

Now let $k>1$ and suppose the statement holds for all $1\leq k^\prime <k$. Consider a graph $G$ with $\chi (G) = k$ and a proper coloring $c$ in $k$ colors. Let its vertices be labeled with $1, 2, \dots , k$ according to their color under $c$. Let $T$ be a labeled tree on $k$ vertices. Choose a leaf $p$ of $T$ and call its neighbor $q$. Denote the label of $p$ as $l_p$ and the label of $q$ as $l_q$. Let $A$ be the set of vertices of $G$ with color $l_q$ such that each vertex in $A$ has a neighbor with color $l_p$. The set $A$ is not empty since $\chi (G)=k$. Consider the subgraph $G^\prime$ of $G$ where we remove the vertices of the color class $l_p$ and the vertices of color class $l_q$ that are not in $A$. It is a graph with chromatic number $k-1$ and by induction hypothesis it contains a labeled tree $T^\prime$ isomorphic to $T-\lbrace p\rbrace$. Since the vertex mapped to $q$ under the isomorphism is element of $A$, it has a neighbor $x\in V(G)$ with color $l_p$ and $T^\prime\cup\lbrace x\rbrace$ is isomorphic to $T$.\qed
\end{prf}
\begin{cor}\label{cor1cr}
A graph $G$ with $\chi (G) = k$ contains every tree $T$ on $k$ vertices as a subgraph. 
\end{cor}
\begin{prf}
A tree $T$ on $k$ vertices can be seen as a labeled tree on $k$ vertices. Thus, $T\subseteq G$ holds by Theorem \ref{t4cr}.\qed
\end{prf}

\begin{cor}
For a tree $T$, the family $\mathcal{F}$ of graphs not containing $T$ as a subgraph is $\chi$-bounded with a bounding function $f_T$ satisfying $f_T(\omega )\leq\vert T\vert$.
\end{cor}
\begin{prf}
The chromatic number of any graph $G\in\mathcal{F}$ is at most $\vert T\vert$ by Corollary \ref{cor1cr}.\qed
\end{prf}

The actual Gyárfas-Sumner conjecture, i.e. Conjecture \ref{c1cr}, for induced subgraphs remains open. Even though many tried to prove the conjecture for over forty years, there are only some trees found to be $\chi$-bounding so far. 

Note that finding a counterexample for Conjecture \ref{c1cr} is not easy either. As the following result shows, considering a random large graph and a tree $T$ does not work since the graph contains an induced subgraph isomorphic to $T$ with asymptotic probability $1$. For our purpose, a \textit{random} graph is a graph where each edge is independently determined to be present with probability $\frac{1}{2}$. For a graph property $P$ let $\mathcal{P}$ denote the family of graphs with this property. Let $G$ be a random graph on $n$ vertices. We say \textit{almost all graphs} have property $P$ if $Prob(G\in\mathcal{P})\xlongrightarrow{{n\to\infty}} 1$.
\begin{thm}
Let $T$ be a tree. Almost all graphs contain an induced subgraph isomorphic to $T$.
\end{thm}
\begin{prf}
Let $T$ be a tree on $k$ vertices and $G$ be a random graph on $n\geq k$ vertices. Observe that a vertex set $H\subseteq V(G)$ of size $k$ induces a graph isomorphic to $T$ with a strictly positive probability $p$ since all edges between vertices in $H$ are present with probabilitiy $\frac{1}{2}$. For the probability of $G$ containing an induced subgraph isomorphic to $T$ then holds:
\begin{equation*}
\begin{split}
\textit{Prob}(T\subseteq_I G) &= 1-\textit{Prob}(T\nsubseteq_I G)\\
&=1-\textit{Prob}(G[H]\not\approx T \hspace{0.4em}\forall H\subseteq V(G), \vert H\vert =k)\\
&=1-\prod_{H\in {V\choose k}}\textit{Prob}(G[H]\not\approx T)\\
&=1-\prod_{H\in {V\choose k}} (1-p)\\
&=1-(1-p)^{n\choose k}\xlongrightarrow{{n\to\infty}} 1
\end{split}
\end{equation*}
\qed
\end{prf}
Thus, we require a special construction to find a graph without an induced subgraph isomorphic to $T$ but with an arbitrarily large chromatic number and therefore many vertices. As mentioned in the introduction, we consider two constructions in Chapters \ref{secMy} and \ref{secLS}.

Continuing with the proofs that already exist, here is a list of them as stated by Chudnovsky, Scott and Seymour in \cite{CSS17} including the cases they proved in the same paper.

\begin{thm}\label{t3cr}
The following trees are $\chi$-bounding:
\begin{itemize}
\item trees of radius at most two (Kierstead and Penrice \cite{Ki94}),
\item trees obtained from a tree with radius at most two by subdividing every edge incident to the root exactly once (Kierstead and Zhu \cite{Ki04}),
\item subdivisions of stars (Scott \cite{Sc97}),
\item trees obtained by adding one vertex to a subdivided star (Chudnovsky, Scott and Seymour \cite{CSS17}),
\item trees obtained from a sudivided star and a star by adding a path joining their centers (Chudnovsky, Scott and Seymour \cite{CSS17}).
\end{itemize} 
\end{thm}

\begin{figure}[ht]
\begin{center}
\includegraphics[scale=1]{chi_bounding_trees}
\end{center}
\caption{Tree $T_1$ with root $r$ that is obtained from a radius-two tree by subdividing every edge incident to the root exactly once, a tree $T_2$ that is a subdivided star with one additional leaf $v$ and a tree $T_3$ obtained from joining the centers of a star and a subdivided star by a path}
\label{f2cr}
\end{figure}

See Figure \ref{f2cr} for trees serving as examples for the second, fourth and fifth result of Theorem \ref{t3cr}. Even if we do not provide the proofs for all these cases, know that the third result is an implication of the topological version of the Gyárfás-Sumner conjecture that has been proved by Scott in \cite{Sc97}. Obviously, the case of subdivided stars is included in the fourth result but since it was proven a long time before the fourth result it is worth a nomination on its own.

\begin{thm}[Scott \cite{Sc97}]
For each tree $T$, there exists a function $f:\N\to\N$ such that for all graphs $G$ not containing an induced subgraph that is a subdivision of $T$ $\chi (G)\leq f(\omega (G))$.
\end{thm}

If $T$ is a subdivision of a star, every graph that is a subdivision of $T$ contains an induced copy of $T$. Thus, no graph in $\textit{Forb}(T)$ contains an induced subgraph that is a subdivision of $T$. This proves the third result in Theorem \ref{t3cr}.\\

Instead of presenting the proofs for the other results, we now consider the cases of stars and paths. First, they provide an approach for proving some cases of the conjecture and second, they may serve as a reference for some results following in the other chapters of this thesis. The two following theorems proved by Gyárfás in \cite{Gy87} show that the families of graphs $\textit{Forb}(K_{1,n})$ and $\textit{Forb}(P_n)$ are $\chi$-bounded and even provide a lower bound on the bounding function in the first case.

\begin{thm}[Gyárfás \cite{Gy87}] The family of graphs $\textit{Forb}(K_{1,n})$ is $\chi$-bounded for all fixed $n \in\N$, $n\geq 2$, with a bounding function $f_n$ satisfying \[\dfrac{R(n,\omega + 1) - 1}{n-1}\leq f_n(\omega )\leq R(n,\omega ).\] 
\end{thm}
\begin{prf}
We prove the lower bound first. By definition of the Ramsey number, there is a graph on $R(n,\omega + 1) - 1$ vertices that contains no stable set of $n$ vertices and no clique of $\omega + 1$ vertices. Let $G$ be such a graph. Then, $G$ does not contain a $K_{1,n}$, since its leaves would form a stable set of $n$ vertices. Additionally, $\chi (G)\geq |V(G)| /(n-1)$ holds because every color class is a stable set. This gives the lower bound.

To prove the upper bound, let $G \in\textit{Forb} (K_{1,n})$ and $\omega (G) = \omega$. Assume, there exists a vertex $v\in V(G)$ with degree $R(n,\omega)$. Then, the neighborhood of $v$ either contains a stable set of $n$ vertices or a clique of $\omega$ vertices. In the first case, the stable set and $v$ would induce a $K_{1,n}$ in $G$. The second case contradicts $\omega (G) = \omega$. Thus, the maximum degree $\Delta (G)$ of $G$ is less than $R(n,\omega)$, i.e. a greedy coloring of $G$ yields $\chi (G)\leq\Delta (G) + 1\leq R(n,\omega)$.\qed
\end{prf}

\begin{thm}[Gyárfás \cite{Gy87}]
For all fixed $n\in\N$, $n\geq 2$, the family of graphs $\textit{Forb}(P_n)$ is $\chi$-bounded with a bounding function $f_n(\omega )\leq (n-1)^{\omega - 1}$.
\end{thm}

\begin{prf}
Let $n\in\N$ be fixed let $G$ be a graph with $\omega (G)=\omega$ such that $P_n\nsubseteq_I G$. We prove that $\chi (G)\leq (n-1)^{\omega -1}$ by induction on $\omega (G)$. Consider the base case for $\omega (G) =1$. Then, $\chi (G) = 1$ and the theorem trivially holds.

Suppose the theorem holds for all graphs $G^\prime\in\textit{Forb}(P_n)$ with $\omega (G^\prime ) \leq t$ for some $t\geq 1$, i.e. $(n-1)^{t-1}$ is a bounding function for them. Let $G\in\textit{Forb}(P_n)$ with $\omega (G) = t + 1$. Now, assume that $\chi (G) > (n-1)^{\omega - 1}$. We shall reach a contradiction by constructing an induced path $Q_n=(q_1,q_2,\dots ,q_n)$ in $G$. To do this, we will define subgraphs $G_i$ of $G$ with $q_i\in V(G_i)$ for all $i\in [n]$, such that $V(G_1)\supseteq V(G_2)\supseteq \dots\supseteq V(G_n)$ and each $G_i$ satisfies the following properties:
\begin{enumerate}[(i)]
\item $G_i$ is connected,
\item $\chi (G_i) > (n-i)(n-1)^{t-1}$,
\item for $1\leq j<i$ and $v\in V(G_i)$ $q_jv$ is an edge in $G$ if and only if $i=j+1$ and $v = q_i$.
\end{enumerate}
Let $G_1$ be a connected component of $G$ with $\chi (G_1) >(n-1)^t$ and let $q_1$ be any vertex of $G_1$. For $i>1$ assume that $G_1,G_2,\dots , G_{i-1}$ and $q_1,q_2,\dots , q_{i-1}$ are already defined and satisfy the wanted properties. Let $A$ denote the neighborhood of $q_{i-1}$ in $G_{i-1}$ and let $B=V(G_{i-1})\setminus (A\cup\{q_{i-1}\})$. Refer to Figure \ref{f1cr} for a visualization of the vertex subsets $A$ and $B$ in $G$. Note that $A$ does not necessarily have to be a stable set as depicted in the figure. The graph $G[A]$ induced by $A$ in $G$ satisfies $\omega (G[A])\leq t$ because a larger clique and $q_{i-1}$ would induce a clique larger than $t+1$ in $G$. Thus, we have $\chi (G[A])\leq (n-1)^{t-1}$ by induction hypothesis.

\begin{figure}[ht]
\begin{center}
\includegraphics[scale=1]{path_general}
\end{center}
\caption{Construction of an induced path $Q_i$ in $G$}
\label{f1cr}
\end{figure}


Assume, $B\neq\emptyset$. Observe that $\chi (G_{i-1})\leq \chi (G[A]) + \chi (G[B])$ since proper colorings of $G[A]$ and $G[B]$ in distinct colors and assigning $q_i$ any color used in $V(G[B])$ defines a proper coloring of $G_{i-1}$. Thus, \[\chi (G[B])\geq \chi (G_{i-1})-\chi (G[A])>(n-i+1)(n-1)^{t-1}-(n-1)^{t-1}=(n-i)(n-1)^{t-1}.\]
Therefore, there exists a connected component $H$ of $G[B]$ with $\chi (H)>(n-i)(n-1)^{t-1}$. Recall, that $G_{i-1}$ is connected, i.e. there also exists a vertex $q_i\in A$ such that $V(H)\cup \{q_i\}$ induces a connected subgraph $H^\prime$. We choose $q_i$ as the new vertex in the path $Q_i$ and $H^\prime$ as $G_i$. Then, $G_i$ obviously satisfies $(i)$ and $(ii)$. Since the only edge between $V(Q_{i-2})$ and $V(G_{i-1})$ is $q_{i-1}q_{i-2}$, there is no edge at all between $V(Q_{i-2})$ and $q_i$ and by definition the edge $q_{i-1}q_i$ exists. Thus, $(iii)$ is satisfied as well. 

If $B=\emptyset$, $\chi (G_{i-1})\leq \chi (G[A]) + 1$. By property $(ii)$ of $G_{i-1}$, $(n-i+1)(n-1)^{t-1} < (n-1)^{t-1} +1$ holds. Since $n\geq 2$ we have $i=n$ and we choose $q_n$ as any vertex of $A$. Note that $A\neq\emptyset$ by properties $(i)$ and $(ii)$ of $G_{i-1}$. With $G_n=\{q_n\}$ we are done.\qed
\end{prf}


Observe that if we weaken Conjecture \ref{c1cr} by making more requirements for the graphs that are supposed to contain the induced subgraphs that simplifies certain proofs greatly. Rather than presenting some other proofs concerning cases of the Gyárfás-Sumner conjecture, the next theorem shows that presuming a certain girth and minimum degree in a graph assures the existence of induced subgraphs isomorphic to some trees. 

\begin{thm}
Let $T$ be a tree on $k\in\N$ vertices with maximum degree $\Delta (T)=d$. A graph $G$ with girth $g$ such that $k+1\leq g<\infty$ and minimum degree $\delta (G)\geq d$ contains an induced subgraph isomorphic to $T$.
\end{thm}
\begin{prf}
We prove this by induction on $k$. If $k=1$, $T$ is a tree on a single vertex. A graph $G$ with girth $2\leq g<\infty$ has at least one vertex. Choose any vertex in $G$. The graph induced by this vertex is isomorphic to $T$.

For $k\geq 2$, $T$ is a tree on at least $2$ vertices. Thus, we choose a leaf $l$ of $T$ and call the vertex that $l$ is adjacent to $u$. Let $G$ be a graph with girth $k+1\leq g<\infty$ and minimum degree $\delta (G)\geq d$. Now consider $T^\prime = T-\{l\}$. The graph $T^\prime$ is a tree on $k-1$ vertices with maximum degree still at most $d$. By induction hypothesis holds $T^\prime\subseteq_I G$. Let $\varphi$ be the map, that embeds $T^\prime$ in $G$. Since $\delta (G)\geq d$ and $u$ has degree at most $d-1$ in $T^\prime$, we find a vertex $v\in V(G)$ adjacent to $\varphi (u)$ that is not part of the embedding of $T^\prime$ in $G$ with $\varphi$. Then,
\[\varphi^\prime :V(T)\to V(G), \varphi^\prime (w)=\begin{cases}v&\text{if }w=l,\\ \varphi (w) &\text{otherwise}\end{cases}\]
defines an embedding of $T$ in $G$. To verify that $\varphi^\prime (T)$ is indeed an induced subgraph we need to show that $v$ is not adjacent to any vertex from $\varphi (T^\prime )$ except $\varphi (u)$.

Assume it were and call the vertex $v$ is adjacent to $x$. Then, there exists a path in $G$ with endpoints $v$ and $x$ containing only edges from the embedding of $T^\prime$ because $T^\prime$ is connected. Together with the edge $\{v,x\}$ this path forms a cycle in $G$ on at most $k$ vertices since $T$ has $k$ vertices. This contradicts $g\geq k+1$. Thus, $\varphi^\prime (T)$ induces a subgraph isomorphic to $T$ in $G$.\qed
\end{prf}


The following results by Kierstead and Rödl \cite{Ki96} do not prove any cases of the Gyárfás-Sumner conjecture either. Nevertheless, they show the existence of a bounding function for the family of graphs $\textit{Forb}(T,K_{n,n})$ not containing an induced subgraph isomorphic to $T$ or $K_{n,n}$ for a tree $T$ and $n\in\N$. Despite being a weaker statement than the conjecture itself it proves quite useful in the following chapters: For proving or disproving the conjecture we only need to consider families of graphs $F$ such that for each $n\in\N$ there exists a $G\in\mathcal{F}$ inducing a $K_{n,n}$.

Before presenting their main theorem, we need to prove three preliminary lemmas. They use some hypergraph theory in their proofs, so we want to introduce it as well. Let $H=(V,E)$ be a hypergraph. A hypergraph $G=(V,E^\prime)$ is a \textit{covering hypergraph} by $H$ if for every edge $e\in E$ there exists an edge $e^\prime\in E^\prime$ such that $e^\prime\subseteq e$. 

They define a \textit{rooted hypergraph} as a triple $H=(V,E,r)$, where $(V,E)$ is a hypergraph and $r:E\to V$ a function, such that $r_e=r(e)\in e$ for all $e\in E$. We call $r_e$ the \textit{root} of $e$. The \textit{breadth} $b(H)$ of a rooted hypergraph $H$ is the smallest $b$ such that for every vertex $v\in V$ there exist at most $b$ edges $e_1,e_2,\dots ,e_b$ with $r(e_i)=v$, $i\in [b]$ and $e_i\cap e_j= \{v\}$ for $1\leq i<j\leq b$.

\begin{lemma}\label{l1cr}
If $G$ is a covering graph of a hypergraph $H$, then $\chi (H)\leq\chi (G)$.
\end{lemma}
\begin{prf}
Any proper coloring of $G$ is a proper coloring of $H$.\qed
\end{prf}

\begin{lemma}\label{l2cr}
If $G$ is an oriented graph, $\chi (G)\leq 2\Delta^{out}(G) +1$.
\end{lemma}
\begin{prf}
We prove this by induction on the number of vertices in $G$, say $n$. For $n=1$, $\chi (G) =1\leq 2\Delta^{out}(G) +1$ since $\Delta^{out}(G)=0$.
Let $n\geq 2$ and let $G$ be a graph on $n$ vertices with $\Delta^{out}(G)=k$. For the average out-degree $d_{avg}^{out}$ of $G$ holds \[d_{avg}^{out}=\dfrac{1}{n}\sum_{v\in V(G)}d^{out}(v)\leq\dfrac{1}{n} \cdot n\cdot k=k.\] Since each edge in $E(G)$ has exactly one starting vertex end one vertex it is pointing to, the average in- and out-degree of $G$ are equal. Thus, the overall degree of the vertices in $G$, $d^{all}=d^{out}+d^{in}$ averages at most $2k$. There exists a vertex $v\in V(G)$ with overall degree at most $2k$. Then, $G-v$ is a graph on $n-1$ vertices with maximum out-degree at most $k$ and therefore properly colorable with $2k+1$ colors by induction hypothesis. Let $c$ be such a proper coloring of $G-v$. Coloring every vertex of $G$ as in $c$ and $v$ with the color not used on its neighbors provides a proper coloring of $G$ with $2k+1$ colors.\qed
\end{prf}

Observe that for an oriented graph $G=(V,E)$, we can form a rooted graph $G^\prime =(V,E^\prime ,r)$ by setting $E^\prime =\{\{ x,y\} :(x,y)\in E \textit{ or } (y,x)\in E\}$ and for $\{ x,y\}\in E^\prime$, $r(\{ x,y\} )=x$ if and only if $(x,y)\in E$. Then, $\Delta^{out}(G)=b(G^\prime )$ and thus, $\chi (G)\leq 2b(G^\prime ) +1$.

\begin{lemma}\label{l3cr}
For a rooted s-uniform hypergraph $H$ $\chi (H)\leq 2(s-1)b(H) +1$.
\end{lemma}
\begin{prf}
Let $H=(V,E,r)$ be a rooted s-uniform hypergraph. For each vertex $v\in V$, let $C(v)=\{ e_1,e_2,\dots ,e_n\}$ be a maximum set of edges such that $r(e_i)=v$ for $i\in [n]$ and $e_i\cap e_j = \{ v\}$ for $1\leq i<j\leq n$. Then, $n\leq b(H)$. Define an oriented graph $G=(V,D)$ with $D=\{(v,w): v =r_e \textit{ for some } e\in E\textit{ and } w\in e_i -\{v\} \textit{ for some } e_i\in C(v)\}$. Now, consider any edge $e\in E$. If $e\in C(r_e)$, let $w$ be any vertex in $e-\{ r_e\}$; otherwise there exists an edge $f\in C(r_e)$ such that there exists a $w\in (e-\{ r_e\})\cap f$. Then, $(r_e,w)\in D$ and $\{r_e,w\}\subseteq e$. An illustration of this construction is shown in Figure \ref{f3cr}.

Hence, $G$ is a covering graph of $H$. Since $\Delta^{out}(G)\leq (s-1)b(H)$, for the chromatic number of $H$ holds $\chi (H)\leq\chi (G)\leq 2(s-1)b(H) +1$ by the previous Lemmas \ref{l1cr} and \ref{l2cr}.\qed
\end{prf}

\begin{figure}[ht]
\begin{center}
\includegraphics[scale=1]{hyper_ori_graph}
\end{center}
\caption{Oriented graph $D$ for a $3$-uniform rooted hypergraph $H$}
\label{f3cr}
\end{figure}

In the proof of the main theorem by Kierstead and Rödl they use an application of the Ramsey theorem not yet mentioned. Let $B=B(s,t,r)$ be the Ramsey function such that for any funtion $c:X\times Y\to [r]$, where $\vert X\vert =\vert Y\vert = B$, there exist subsets $X^\prime\subset X$, $Y^\prime\subset Y$ and a color $\alpha\in [r]$ satisfying $\vert X^\prime\vert = s$, $\vert Y^\prime\vert =t$ and $c(i,j) =\alpha$ for all $(i,j)\in X^\prime\times Y^\prime$. 

Additionally, we need to introduce the notion of oriented cliques and oriented $K_{n,n}$s since their proof works with directed graphs. Therefore, let $DK_{n,n}$ denote an oriented $K_{n,n}$ such that all arcs point from one part to the other part and $TK_m$ be the transitive tournament on $m$ vertices, i.e. an oriented clique $K_m$ such that we can order the vertices $v_1, v_2, \dots ,v_m$ in a way that every edge $(v_i,v_j)$, $i\neq j$, points to the vertex of higher index.

\begin{thm}[Kierstead, Rödl \cite{Ki96}]\label{t2cr}
For all $m,n\in\N$ and all oriented trees $T$, there exists a function $f=f(m,n,T)$ such that for every oriented graph $G$ the following statement holds. If $\chi (G)\geq f(m,n,T)$, then $G$ induces either $TK_m$, $DK_{n,n}$ or $T$.
\end{thm}

\begin{prf}
Let $G$ be an oriented graph. Observe that if $\omega (G)\geq m^\prime$, where $m^\prime = R(m,m)$, then $G$ contains $TK_m$. To see this, consider a clique of size $m^\prime$. Order the vertices of an oriented clique of size $m^\prime$ as $v_1,v_2,\dots v_{m^\prime}$ and choose a coloring $c$ with $c((v_i,v_j))=1$ if $i<j$ and $c((v_i, v_j))=2$ otherwise. By definition of the Ramsey number, we find a monochromatic clique of size $m$. The vertices in this clique induce a transitive tournament in $G$.

Second, note that if $G$ does not contain a clique of size $m^\prime$ but a $K_{n^\prime , n^\prime}$, where $n^\prime = R(m^\prime , B(n,n,2))$, $G$ induces a $DK_{n,n}$. Denote the parts of the $K_{n^\prime , n^\prime}$ as $A$ and $B$. We find stable sets of size $B(n,n,2)$ in both $A$ and $B$ again by the definition of the Ramsey number. After coloring all edges that point from $A$ to $B$ in the first color and all edges pointing from $B$ to $A$ in the second color, we obtain a monochromatic $K_{n,n}$ by definition of $B(n,n,2)$. Its vertices induce a $DK_{n,n}$ in $G$. 

Thus, it suffices to show, that $(\ast)$ there exists a function $g=g(m^\prime ,n^\prime ,T)$ such that every oriented graph $G$ with $\chi (G)\geq g(m^\prime ,n^\prime ,T)$ either has $\omega (G)\geq m^\prime$ or $G$ contains $K_{n^\prime ,n^\prime }$ or $G$ induces $T$. We prove this claim by induction on the number of vertices $v(T)$ of a tree $T$. 

The base case for $v(T)=1$ is trivial since a tree on one vertex is induced in every graph that has at least one vertex. 

Now suppose that $g(m^\prime ,n^\prime ,T^\prime )$ exists for all $m^\prime ,n^\prime \in\N$ and for all trees $T^\prime$ with $v(T^\prime )\leq s$. Let $T$ be a tree on $s+1$ vertices. We define a function $g(m^\prime ,n^\prime ,T)$ for all $m^\prime ,n^\prime\in\N$ so that $(\ast)$ holds as follows. Choose a leaf $l$ of $T$, which is adjacent to a vertex $u$ of $T$, and let $T^\prime = T- \{l\}$. Let $\sigma =$ \textit{in} if the edge between $l$ and $u$ points to $u$ and $\sigma = $ \textit{out} otherwise. Define $g(m^\prime ,n^\prime ,T) = 2(g(m^\prime ,n^\prime ,T^\prime )(s-1)+s)B$, where $B=B(n^\prime ,n^\prime ,s)$. We now show that $g(m^\prime ,n^\prime ,T)$ satisfies $(\ast)$. Let $G=(V,E)$ be a graph such that $\chi (G)\geq g(m^\prime ,n^\prime ,T)$, and suppose $G$ contains neither $K_{m^\prime}$ nor $K_{n^\prime ,n^\prime}$. We claim that $G$ induces $T$.

Let $W=\{v\in V:\delta^\sigma (v) <sB\}$. Then, by Lemma \ref{l2cr}, $\chi (G[W]) < 2sB$. Let $G^\prime = G[V^\prime ]$, where $V^\prime = V-W$. Since $\chi (G)\geq 2(g(m^\prime ,n^\prime ,T^\prime )(s-1)+s)B$ we need at least $2g(m^\prime ,n^\prime ,T^\prime )(s-1)B + 1$ colors for the vertices in $V^\prime$ in every proper coloring of $G^\prime$, i.e. $(\ast\ast)$ $\chi (G^\prime )\geq 2g(m^\prime ,n^\prime ,T^\prime )(s-1)B + 1$.

Next, construct a rooted $s$-uniform hypergraph $H=(V^\prime ,F,r)$ with the same set of vertices as $G^\prime$. An $s$-subset $S$ of $V^\prime$ is an edge of $H$ if and only if $T^\prime\approx G[S]$. Let $\phi$ be an isomorphism from $T^\prime$ to $G[S]$ and let $r_S$ be the image of $u$ under $\phi$. Note that in a proper coloring $c$ of $H$ no color class induces $T$. Thus, we can apply the induction hypothesis to each subgraph $H_C$ in $G$ that is induced by the vertices of a color class $C\subseteq V^\prime$ in $H$. Then, $\chi (H_C)\leq g(m^\prime , n^\prime , T^\prime )$ for each color class $C$. The union of all color classes contains every vertex in $V^\prime$. Thus, by coloring the vertices of each color class with its own set of colors, we get $\chi (G^\prime )\leq \chi (H)g(m^\prime ,n^\prime ,T^\prime )$. Together with $(\ast\ast)$, this yields $2(s-1)B +1\leq \chi (H)$.

By Lemma \ref{l3cr} $b(H)\geq B$ holds. Then, there exists a vertex $r\in V^\prime$ and a set of hyperedges of $H$, $Y = \{S_1,S_2,\dots ,S_B\}$, with $r(S_i)=r$ for all $i\in [B]$ and $S_i\cap S_j=\{r\}$ for all $1\leq i<j\leq B$. Let $S_i=\{r,y_i^1,\dots ,y_i^{s-1}\}$. Then, $\vert \bigcup_{i\in [B]} S_i\vert =1 +(s-1)B$. There are at most $(s-1)B$ vertices from $\bigcup_{i\in [B]} S_i$ in $N^\sigma (r)$ since $r\in \bigcup_{i\in [B]} S_i$ is not a neighbor of itself. Thus, we find a set $X\subseteq N^\sigma (r)$ with $\vert X\vert =B$ and $X\cap S_i =\emptyset$ for all $i\in [B]$ because $\vert N^\sigma (r)\vert\geq sB$. See Figure \ref{f4cr} for an illustration. 

Define the function $C:X\times Y\to [s]$ by setting $c(x,S_i)=s$ if $x$ is not adjacent to any vertex in $S_i-\{r\}$ and otherwise $c(x,S_i)=j$, $1\leq j\leq s-1$, where $j$ is the least index such that $x$ is adjacent to $y_i^j$. This is well-defined since $j$ can be at most $s-1$.

By the definition of $B$, we find sets $X^\prime\subseteq X$ and $Y^\prime\subseteq Y$ with $\vert X^\prime\vert = \vert Y^\prime\vert =n^\prime$ such that $c(x,y)=\alpha$ for all $(x,y)\in X^\prime\times Y^\prime$ and a color $\alpha\in [s]$. If $\alpha\neq s$, then $X^\prime\cup\{y_i^\alpha :S_i\in Y^\prime\}$ contains $K_{n^\prime ,n^\prime}$, which contradicts the choice of $G$. Thus, $\alpha = s$. Let $x\in X^\prime$. Then, $x$ is not adjacent to any vertex in $S_i \setminus \{r\}$ for any $S_i\in Y^\prime$ and $x$ is a $\sigma$-neighbor of $r$. Since $T^\prime\approx G[S_i]$ by an isomorphism mapping $u$ to $r$, $G[S_i\cup\{r\}]\approx T$.\qed
\end{prf}

\begin{figure}[ht]
\begin{center}
\includegraphics[scale=1]{kierstead_roedl_1}
\end{center}
\caption{Hyperedges from $Y$ and vertex set $X$}
\label{f4cr}
\end{figure}

\begin{note}
It is a direct implication of this theorem that the family $\textit{Forb}(T,K_{n,n})$ is $\chi$-bounded. This can be easily seen by replacing the directed graph $G$ by the underlying undirected graph $G^\prime$.
\end{note}

We finish this chapter with a motivation for considering an oriented version of the Gyárfás-Sumner conjecture even if that is not part of this thesis. An oriented graph $H$ is $\chi$-bounding if there exists a bounding function $f:\N\to\N$ such that any oriented graph $G$ that contains a subgraph isomorphic to $H$ has chromatic number $\chi (G)\leq f(\omega (G))$, where in this case $\omega (G)$ denotes the size of the largest oriented clique in $G$. Before looking at specific cases note that we are only considering trees for which we know Conjecture \ref{c1cr} to be true. There actually exist some results for oriented stars and oriented $P_4$. Aboulker et al. \cite{Ab16} conjectured that every oriented star is $\chi$-bounding and proved the cases for $S_{k,0}$, $S_{0,l}$ and $S_{1,1}$ where $S_{k,l}$ denotes an oriented star with $k$ outgoing edges and $l$ edges pointing to the center vertex. 

Next, observe that all paths on less than four vertices are stars and therefore included in the previous case. Let $or(P_4)$ denote the set of all pairwise non-isomorphic oriented $P_4$. Then, $or(P_4)$ contains only four elements, which we denote as \[\rightarrow\rightarrow\rightarrow , \rightarrow\leftarrow\rightarrow , \rightarrow\rightarrow\leftarrow , \leftarrow\leftarrow\rightarrow\] where the arrows depict the orientations of the edges. For oriented $P_4$ Aboulker et al. \cite{Ab16} conjectured that only $\rightarrow\rightarrow\leftarrow$ and $\leftarrow\leftarrow\rightarrow$ are $\chi$-bounding. Chudnosky, Scott and Seymour \cite{CSS17b} extended these results and showed that both conjectures are indeed true, so we can state:

\begin{thm}[Chudnovsky,Scott, Seymour \cite{CSS17b}]
The following oriented trees are $\chi$-bounding:
\begin{itemize}
\item oriented stars,
\item the oriented paths on four vertices $\rightarrow\rightarrow\leftarrow$ and $\leftarrow\leftarrow\rightarrow$.
\end{itemize}
\end{thm}

Note that most results concerning the oriented version of the Gyárfás-Sumner conjecture deal with the cases of oriented stars and oriented paths on four vertices. There is not much known beyond that.
  \newpage
  
  \section{Mycielskian Graphs}\label{secMy}
  Finding a counterexample for the Gyárfás-Sumner conjecture implies constructing a family $\mathcal{F}$ of graphs such that $\mathcal{F}\subseteq\textit{Forb}(T)$ for some tree $T$ but there is no bounding function for $\mathcal{F}$. Mycielski's construction of triangle-free graphs with high chromatic number \cite{My55} may provide such a family. The goal of this chapter is to eliminate the family of graphs obtained by the Mycielskian construction as candidate for disproving the Gyárfás-Sumner conjecture by showing that in each of these families we find infinitly many graphs containing an induced copy of $T$ for all trees $T$.

\begin{defn}
Let $G=(V,E)$ be a graph. Its \textit{Mycielskian graph} $\mu(G)$ is defined as the graph $G^\prime=(V\cup U\cup\{w\}, E^\prime)$, where $G^\prime [V]$ is isomorphic to $G$, the set $U$ contains a vertex $u_i$ for every vertex $v_i\in V$, $i\in [\vert V\vert]$, such that $V\cap U=\emptyset$, and $w\notin V\cup U$ is an additional vertex. In addition to the edges of the subgraph isomorphic to $G$, for each $u_i\in U$, $i\in [\vert V\vert]$, $E^\prime$ contains edges from $u_i$ to every vertex in $N_G(v_i)$ and an edge $\lbrace u_i ,w\rbrace$. 
\end{defn}
Mycielski's construction provides a sequence of graphs $G_0,G_1, G_2,\dots $ starting with $G_0=K_2$, where $G_k = \mu (G_{k-1})$, $k\in\N$, and let $G_k=(V_k,E_k)$. Let us denote the family of graphs in this sequence as $\mathcal{M}$. First, we state some properties of a graph $G_k$. Let us denote the set of copies of the vertices in $V_{k-1}$ as $U_k$ and the one additional vertex as $w_k$. For each vertex $v\in V_{k-1}$ we refer to the vertex $u\in U_k$ that is adjacent to exactly the vertices in $N_{G_{k-1}}(v)$ as the copy of $v$ in $G_k$. By the construction, the set $U_k$ is a stable set. Thus, together with the vertex $w_k$ the vertices in $U_k$ induce a star $K_{1, \vert U_k\vert}$ in $G_k$.
Second, observe that the neighborhood of each vertex we add, is a stable set.

With these facts in mind, we obtain the following two theorems which Myvielski proved as part of his construction.

\begin{thm}[Mycielski \cite{My55}]\label{t2my}
All graphs in $\mathcal{M}=\{G_k:k\in\Z, k\geq 0\}$ are triangle-free.
\end{thm}
\begin{prf}
We prove this by induction on $k$. The graph $G_0=K_2$ trivially does not contain any triangles. Now, let $k\geq 1$. Then $G_{k-1}$ is triangle-free. By the specifications of the construction the neighborhood of each vertex we add is a stable set: The vertex $w_k$ is only adjacent to vertices in $U_k$ which is a stable set. And the neighbors in $V_k$ of each vertex in $U_k$ must form a stable set as well since otherwise they would induce a triangle in $G_{k-1}$. Thus, no vertex we add creates a triangle in $G_k$.\qed 
\end{prf}

\begin{thm}[Mycielski \cite{My55}]\label{t3my}
For all $k\in\Z$, $k\geq 0$, the chromatic number of $G_k$ is at least $k$.
\end{thm}
\begin{prf}
We prove this also by induction on $k$. The graph $G_0=K_2$ has chromatic number $\chi (K_2)=2 \geq 0$. Let $k\geq 1$. We shall show that $G_k$ is not properly colorable with $k-1$ colors. By induction hypothesis, we know that there exists no proper coloring of $G_{k-1}$ that uses only $k-2$ colors. Thus, for every coloring $c$ of $G_k$ with $k-1$ colors there exists a vertex set $S\subseteq V_{k-1}$ of $k-1$ vertices such that the vertices have pairwise different colors and are adjacent to a vertex of each of the other colors. Then, for each vertex $v\in S$ the vertex in $U_k$ that is adjacent to the neighbors of $s$ must have the same color as $s$. Thus, for each of the $k-1$ colors, there exists a vertex in $U_k$ with this color. The vertex $w_k$ must be colored in a $k^{th}$ color which is a contradiction to the assumption that $c$ uses only $k-1$ colors.\qed
\end{prf}

Next, observe that if a graph $G_k$, $k\in\N$, contains an induced copy of a graph $H$, then all graphs of the sequence with a higher index contain an induced copy of $H$ as well.

Last, note that this construction indeed provides a family of graphs worthy to look at for disproving Conjecture \ref{c1cr}. As we have shown in Theorem \ref{t2my} and Theorem \ref{t3my}, $\mathcal{M}$ contains triangle-free graphs of arbitrarily high chromatic number. Thus, if it were possible to prove that $\mathcal{M}\subseteq\textit{Forb}(T)$, that would disprove Conjecture \ref{c1cr}. The first idea is to simply apply Theorem \ref{t2cr} to prove that $\mathcal{M}\subseteq\textit{Forb}(T)$. That is not possible as the following lemma shows since for all $n\in\N$ there exists a graph $G_k\in\mathcal{M}$ that contains an induced subgraph isomorphic to $K_{n,n}$. Thus, we need to have a closer look at the construction to decide whether $\mathcal{M}\subseteq\textit{Forb}(T)$. 

\begin{lemma}\label{l1my}
For all $n\in\N$ there exists a graph $G_k\in\mathcal{M}$ that contains an induced subgraph isomorphic to $K_{n,n}$.
\end{lemma}
\begin{prf}
Consider a graph $G_k\in\mathcal{M}$ such that $U_k$ has size at least $n$. Choose a subset $X\subseteq U_k$ of size $n$. Then, the vertex $w_k$ is adjacent to every vertex in $X$. Let $w_k^{i}$, $i\in [n-1]$, denote copy of $w_k$ in $U_{k+i}$. Then, each $w_k^{i}$, $i\in [n-1]$, is adjacent to the neighbors of $w_k$, including all verticces in $X$, and the vertices $w_k^{i}$, $i\in [n-1]$, form a stable set $S$ in $G_{k+n-1}$. Thus, the vertices in $S\cup X$ induce a $K_{n,n}$ in $G_{k+n-1}$.\qed
\end{prf}

As mentioned at the beginning of this chapter, our goal is to show that for each tree $T$, there exists a graph $G_k\in\mathcal{M}$ containing an induced subgraph isomorphic to $T$. Then, all graphs in $\mathcal{M}$ with a higher index contain an induced copy of $T$ as well and thus, it is not possible to use $\mathcal{M}$ aas counterexample for Conjecture \ref{c1cr}.
We start with two proofs by induction proving the existence of induced subgraphs isomorphic to paths and caterpillars in graphs of $\mathcal{M}$. For $k\in\N$ denote the copy of $w_k$ in $U_{k+1}$ as $w_k^1$ as in the proof of Lemma \ref{l1my} . Let $Q_n =(w_1^1 ,w_2^1 ,\dots ,w_n^1)$ be the path through the first $n$ such vertices. 
\begin{thm}\label{t1my}
For all $n\in\N$, $Q_n$ is an induced subgraph of $G_{n+1}$.
\end{thm}
\begin{prf}
We prove this theorem by induction on $n$. To launch the induction let $n=1$. The vertex $w_1^1$ trivially is a path on one vertex in the graph $G_2$.

Now let $n\geq 2$. By induction hypothesis, $Q_{n-1}$ is an induced subgraph of $G_n$. Then, the path $Q_{n-1}$ is also an induced subgraph of $G_{n+1}$. 
Now consider $w_n$. It is adjacent to any vertex in $U_n$ - including $w^1_{n-1}$ - but no other vertex, in particular no $w^1_i$, $i\in [n-2]$. Thus, as the copy of $w_n$ in $U_{n+1}$, $w^1_n$ is only adjacent to $w^1_{n-1}$ and no other vertex of $Q_{n-1}$. Thus, $Q_n$ is an induced subgraph of $G_{n+1}$.\qed
\end{prf}

\begin{cor}\label{c1my}
The graph $G_{n+1}$ contains an induced subgraph isomorphic to $P_n$ for all $n\in\N$.
\end{cor}
\begin{prf}
As a path on $n$ vertices $Q_n$ is isomorphic to $P_n$ and $Q_n\subseteq_I G_{n+1}$ by Theorem \ref{t1my}.\qed
\end{prf}

In the next proof we will use certain subgraphs of $Q_n$ to construct an induced copy of a caterpillar in some $G\in\mathcal{M}$. A \textit{caterpillar} is a tree $T$ with a subgraph $P$ isomorphic to a path that contains all vertices in $T$ with degree at least two. Then, $T$ also contains a subgraph isomorphic to a path that contains exactly the vertices in $T$ of degree at least $T$. That is since a vertex of degree one can only be an endpoint of $P$ and considering the subpath of $P$ without the endpoints if they are leaves yields a path as wanted.

Also, let $Q_{i,j}=(w_i^1 ,\dots , w_j^1 )$, $1\leq i\leq j\leq n$, denote the subpath of $Q_n$ with endpoints $w_i^\prime$ and $w_j^\prime$.

\begin{thm}
Let $T$ be a caterpillar. Then, there is a $G\in\mathcal{M}$ containing an induced subgraph isomorphic to $T$.
\end{thm}
\begin{prf}
Let $T$ be a caterpillar with maximum degree $d:=\Delta (T)$. Let $P=(p_1, p_2, \dots , p_n)$ denote the path in $T$ containing exactly the vertices in $T$ of degree at least two and let $n$ denote the number of vertices in $P$. Note that if such a $P$ does not exist, all vertices in $T$ have degree one. Then, $T$ is a path on one or two vertices and therefore $G_0=K_2$ trivially contains an induced subgraph isomorphic to $T$.

Suppose $P$ exists. All vertices with degree one have to be adjacent to a vertex in $P$ since $T$ is connected. If $d\leq 2$, $T$ is a path and Corollary \ref{c1my} then states, that it is an induced subgraph of $G_{n+1}$.

If not, we have $d>2$. Every vertex in $P$ is adjacent to no more than $d$ leaves of $T$. Now, let $G_k$ be a graph of size at least $d$ and consider $G_{k+n+1}$. By Theorem \ref{t1my}, the path $Q_{k+n}=(w^1_1,w^1_2,...,w^1_{k+n})\subseteq_I G_{k+n+1}$. Thus, $Q_{k+1,k+n}$ is an induced path on $n$ vertices in $G_{k+n+1}$. Let $N_1:=U_{k+1}$ and for $1<l\leq n$ let $N_l\subseteq U_{k+l}$ denote the copies in $U_{k+l}$ of the vertices in $N_{l-1}$. Then, no vertex from $\bigcup_{l\in [n]}N_l$ is also a vertex of $Q_{k+1,k+n}$. Observe, that $\vert N_l \vert \geq d$ for all $l\in [n]$. We then claim that the vertices in $S_i:=V(Q_{k+1,k+i})\cup\bigcup_{l=1}^i N_l$ induce a tree in $G_{k+n+1}$. We prove this by induction on $i$.

For the base case let $i=1$. The vertices in $N_1=U_{k+1}$ form a stable set  and are adjacent to $w_{k+1}$ by definition of $U_{k+1}$. Therefore, they are adjacent to $w^1_{k+1}$ as well. Thus, $S_1$ induces a subgraph isomorphic to a star $K_{1,\vert N_1\vert}$, i.e. $G_{k+n}[S_1]$ is an induced tree.

Now let $1<i\leq n$. By the inductive hypothesis, $S_{i-1}$ is an induced tree, i.e. from the vertices in $S_{i-1}$ the vertices in $N_{i-1}$ are only adjacent to $w_{k+i-1}^1$. Observe that $(S_{i-1}\setminus \lbrace w_{k+i-1}^1\rbrace )\subseteq V_{k+i-1}$. Consider $N_i\subseteq U_{k+i}$. By the definition of vertices in $U_{k+i}$, vertices in $N_i$ do not have an edge to any vertex from $S_{i-1}\setminus \lbrace w_{k+i-1}^1\rbrace$ since the vertices in $N_{i-1}$ do not have them. Furthermore, $N_i\cup\lbrace w_{k+i-1}^1\rbrace \subseteq U_{k+i}$ is a stable set by definition but each vertex in this set has an edge to $w_{k+i}$ and therefore $w_{k+i}^1$. What is left to show, is that $w_{k+i}^1$ is not adjacent to any vertex in $S_{i-1}\setminus \lbrace w_{k+i-1}^1\rbrace$. That is easy to see, since $(S_{i-1}\setminus \lbrace w_{k+i-1}^1\rbrace )\cap U_{k+i}=\emptyset$. Thus, the vertices in $S_i$ induce a tree in $G_{k+i+1}$.

By mapping $p_j$ to the vertices of $w^1_{k+j}$ and the leaves adjacent to the vertex $p_j$ to a vertex from $N_j$, $j\in [n]$, we obtain an induced subgraph of $G_{k+n+1}$ that is isomorphic to $T$.\qed
\end{prf}

After stating the proofs for some special cases of trees, there is actually a slightly different approach that proves that for all trees $T$ we can find a graph $G\in\mathcal{M}$ containing an induced subgraph isomorphic to $T$. Therefore, we need to introduce the concept of a rooted tree first. A \textit{rooted} tree $T$ is a tree in which one vertex $r\in T$ is set to be the \textit{root} of $T$. Then, the \textit{parent} of a vertex $v\in T$ is the vertex adjacent to it in the path from $v$ to the root. All vertices of $T$ except the root have a unique parent. We call a vertex $u\in T$ a \textit{child} of $v$ if $v$ is its parent.

Second, we want to order the vertices $v_1, v_2, \dots , v_n$ of a rooted tree on $n$ vertices in a certain way. An order of the vertices is called \textit{in-order} if it can be obtained by the following procedure. Let $v_1$ be the root of $T$ and for each $i\in [n-1]$ choose $v_{i+1}$ as a child of $v_i$ that has not been assigned yet. If there is no such child left, consider the previous vertices in decreasing order and choose a child of them if possible. Observe that each vertex except $v_1$ is a child of a vertex with a smaller index.

\begin{thm}
Let $T$ be tree. Then, there exists a $G\in\mathcal{M}$ containing an induced subgraph isomorphic to $T$.
\end{thm}
\begin{prf}
Let $n$ be the number of vertices in $T$ and choose a vertex $r$ as the root of $T$. Denote the vertices as $r=v_1, v_2, \dots , v_n$ such they are in-order.

We shall construct an induced subgraph $H$ isomorphic to $T$ with vertices $h_1, h_2, \dots , h_n$. In preparation of the construction, note that each vertex $h_i$ will be a copy in a set $U_{j_i}$, $j_i\in\N$, such that the sets $U_{j_i}$, $i\in [n]$, are pairwise disjoint.

We construct $H$ by choosing the root vertex $v_1$ as a copy from $U_1$ and inductively adding the other vertices in their order in the following way. Each vertex $v_i$, $i>1$, has a unique parent $v_k$ with $k<i$. The parent $v_k$ is element of the set $U_{j_k}$ and the vertex $v_{i-1}$ is element of the set $U_{j_{i-1}}$. Let vertex $h_i$ be the copy of $w_{j_k}$ in $U_m$ such that $m> j_{i-1}$. Since $w_{j_k}$ and $h_k$ are adjacent by definition, as copy of $w_{j_k}$ $h_i$ is adjacent to $c_k$ as well. Thus, for every edge in $T$, there exists an edge in $H$. Observe that all vertices $h_1, h_2, \dots , h_n$ are copies in different sets $U_i$, $i\in [n]$. 

We need to confirm that $H$ is indeed induced. We prove this by induction on $i$, claiming that the vertices in $H_i:=\lbrace h_1, h_2, \dots , h_i\rbrace$ induce a tree in $G_{j_i}$. To launch the induction, let $i=1$. One vertex by itself induces no edge. Thus, it trivially induces a tree in $G_m$.

Let $1<i\leq n$. By the induction hypothesis, the vertices in $H_{i-1}$ induce a tree in $G_{j_{i-1}}$ and therefore also in $G_{j_i}$ since $j_i >j_{i-1}$ and $G_{j_i}$ then contains an induced subgraph isomorphic to $G_{j_{i-1}}$. The vertex $h_i$ is a copy of some $w_{j_k}$, $k<i$. The vertex $w_{j_k}$ is only adjacent to vertices $U_{j_k}$, i.e. it is not adjacent to any vertex in $\lbrace h_1, h_2, \dots , h_{k-1}\rbrace$. Therefore, $h_i$ is also not adjacent to them. Since vertices $w_x$ and $w_y$, $x,y\in\N$, $x\neq y$, are never adjacent by definition, no vertex in $h_l\in\lbrace h_{k+1}, \dots , h_{i-1}\rbrace$ chosen after $h_k$ is, as a copy of some $w_x$, such that the parent of $h_l$ is element of $U_x$, adjacent to $w_{j_k}$. Therefore, $h_i$ is only adjacent to $h_k$ as wanted and the vertices $\lbrace h_1, h_2, \dots , h_i\rbrace$ induce a tree in $G_{j_i}$.

As we observed at the beginning $H=G_{j_i}[H_n]$ contains all edges that correspond to the edges present in $T$. It does not contain any other edges since then $H$ were not an induced tree anymore. Thus, $H\subseteq_I G_{j_i}$ is isomorphic to $T$.\qed
\end{prf}

  \newpage
  
  \section{Line Segment Intersection Graphs and Burling Graphs}\label{secLS}
  Besides the Mycielskian construction we also obtain triangle-free graphs of arbitrarily large chromatic number from the following construction by Pawlik et al. \cite{Paw14} using intersection graphs of line segments in an axis-aligned rectangle in the $\mathbb{R}^2$. As mentioned in a paper by Felsner et al. \cite{Fe18} the construction of Burling graphs results in exactly the same family of graphs. They state that the construction of these graphs actually traces back to the PhD Thesis by Burling himself \cite{Bu65}. Since the result is the same regardless of the construction, it is enough to consider one to exclude all constructions as candidates for disproving the Gyárfás-Sumner conjecture. We will look at both constructions and then prove that for all trees $T$ the family of graphs obtained by these constructions indeed contains infinitely many graphs that induce $T$.

Before proving 

\begin{defn}
Let $R$ be an axis-aligned rectangle in the $\mathbb{R}^2$ with boundaries $[a,b]\times [c,d]$ and $\mathcal{L}$ a family of line segments contained in the interior of $R$. A probe $P$ in $R$ is a rectangle $[a^\prime ,b]\times [c^\prime ,d^\prime]$ with the following properties:
\begin{enumerate}[(i)]
\item $a<a^\prime < b$ and $c<c^\prime <d^\prime < d$
\item no line segment in $\mathcal{L}$ intersects the left boundary of $P$
\item no line segment in $\mathcal{L}$ ends in $P$ or on the boundary on $P$
\item line segments in $\mathcal{L}$ intersecting $P$ are pairwise disjoint
\end{enumerate}
We call the rectangle $[a^\prime ,b^\prime ]\times [c^\prime ,d^\prime]$ with the maximal $b^\prime$, such that no line segment intersects it, the root of $P$.
\end{defn}

Let $R$ be an axis-aligned rectangle with a non-empty interior. Let $(s_i)_{i\in\N}$ and $(p_i)_{i\in\N}$ be sequences inductively defined by setting $s_1=p_1=1$, $s_{i+1}=(p_i+1)s_i+p_i^2$ and $p_{i+1}=2p_i^2$. For each $k\in\N$, we then define a family $\mathcal{L}_k$ of $s_k$ line segments and a set $\mathcal{P}_k$ of $p_k$ probes in $R$ as follows. Let $\mathcal{L}_1$ contain an arbitrary non-horizontal line segment in $R$ and choose any rectangle in $R$ that shares its right boundary with $R$ and intersects the line segment with its lower and upper bound as the probe in $\mathcal{P}_1$. Now suppose $k\geq 2$ for the inductive step. To construct $\mathcal{L}_k$ draw $\mathcal{L}_{k-1}$ in $R$ and for each probe $p\in\mathcal{P}_{k-1}$ place another copy $\mathcal{L}_p$ of $\mathcal{L}_{k-1}$ with probes $\mathcal{P}_{p}$ in the root of $p$. Note that the probes in $\mathcal{P}_{p}$ are not extending to the right boundary of $R$ but instead end at the right boundary of the root of $p$. Afterwards, for each $p\in\mathcal{P}_{k-1}$ and every $q\in\mathcal{P}_{p}$ draw the diagonal $d_q$ of $q$, i.e. a line segment from its bottom-left corner to its upper-right corner. The diagonal $d_q$ crosses all segments intersecting $q$ but no other. Then, $\mathcal{L}_k$ contains the line segments from the $p_{k-1} + 1$ copies of $\mathcal{L}_{k-1}$ and the $p_{k-1}^2$ diagonals, i.e. $\vert \mathcal{L}_k\vert =(p_{k-1} + 1)s_{k-1}+p_{k-1}^2=s_k$. 

Now we want to construct $\mathcal{P}_k$. For each $p\in\mathcal{P}_{k-1}$ and every $q\in\mathcal{P}_{p}$ let $\mathcal{S}(p)$ be the set of line segments in $\mathcal{L}_{k-1}$ intersecting $p$ and $\mathcal{S}_p(q)$ the segments in $\mathcal{L}_k$ intersecting $q$. For each such $q$ we add two probes to $\mathcal{P}_k$. We place the first one, the upper probe $u_q$, close to the top of $q$, such that the diagonal $d_q$ but no other segment in $\mathcal{S}_p(q)$ intersects it and choose the second probe, the lower probe $l_q$, close to the bottom of $q$ such that it contains all segments in $\mathcal{S}_p(q)$ but not $d_q$. Then, both probes end at the right boundary of $R$. Since line segments intersecting $p$ are pairwise disjoint by the induction hypothesis, and since we placed the line segments in $\mathcal{S}_p$ in a way that they do not cross any segment in $p$ as well, $\mathcal{S}(p)\cup \lbrace d_q\rbrace$ and $\mathcal{S}(p)\cup\mathcal{L}_p(q)$ are both independent sets, i.e. both $u_q$ and $l_q$ are proper probes. Finally, observe that $\vert \mathcal{P}_k\vert = 2p_{k-1}^2=p_k$.

Let $G_k$ denote the intersection graph of the line segments in $\mathcal{L}_k$, i.e. each vertex in $G_k$ represents a line segment of $\mathcal{L}_k$ and two vertices are adjacent if and only if the corresponding line segments intersect each other.\\

As mentioned before, the construction by Burling as presented in \cite{Fe18} results in the same family of graphs. For each $k\in\N$, construct the Burling graph $B_k$ and a corresponding set of stable sets $S(B_k)$ as follows. Let $B_1$ be the graph on a single vertex and let $S(B_1)$ contain the only stable set on one vertex in $B_1$. For $k\geq 2$, consider a copy $H$ of $B_{k-1}$ which we think of as the original copy and one further copy $H_S$ of $B_{k-1}$ for each stable set $S\in S(B_{k-1})$. Let $S(H_S)$ denote the copy of $S(H)$ in $H_S$. Furthermore, for each copy $H_S$ and stable set $X\in S(H_S)$ add another vertex $v_{S,X}$ adjacent to all vertices in $X$ but no other vertex. Denote the graph obtained from a copy $H_S$ and added new vertices for each stable set $X\in S(H_S)$ as $H_S^{\prime}$. Then, define the graph $B_k$ as $B_k=H\cup\bigcup_{S\in S(B_k)} H_S^{\prime}$. Its set of stable sets $S(B_k)$ consists of two sets for each $S\in S(H)$ and $X\in S(H_S)$, namely $S\cup \lbrace v_{S,X}\rbrace$ and $S\cup X$. Note that both sets are stable in $B_k$.

\begin{lemma}
The families of graphs obtained by the two constructions are the same, i.e. $\lbrace G_k :k\in\N\rbrace=\lbrace B_k :k\in\N\rbrace$. 
\end{lemma}
\begin{prf}
We shall show that $G_k$ and $B_k$ are isomorphic under an isomorphism $\varphi_k$ for all $k\in\N$ by induction on $k$. The idea is to define a bijection $f_k:\mathcal{P}_k\to S(B_k)$ between the probes in $\mathcal{P}_k$ and the stable sets in $S(B_k)$ in each step such that the images of the vertices corresponding to the line segments in a probe $p\in\mathcal{P}_k$ under $\varphi_k$ are exactly the vertices in $f_k(p)$. We then use these maps to define an isomorphism between $G_{k+1}$ and $B_{k+1}$.

For $k=1$, $G_1$ and $B_1$ are graphs on a single vertex and thus isomorphic with isomorphism $\varphi_1$. Define a bijection $f_1:\mathcal{P}_1\to S(B_1)$ that maps the one probe in $\mathcal{P}_k$ to the one stable set in $S(B_k)$.

Suppose $k\geq 2$, $G_{k-1}$ and $B_{k-1}$ are isomorphic, and the bijection $f_{k-1}$ between $P_{k-1}$ and $S(B_{k-1})$ exists as wanted. We extend the isomorphism to pairs of copies $(\mathcal{L}_p, H_{f(p)})$ of $G_{k-1}$ and $B_{k-1}$ respectively for all probes $p\in\P_{k-1}$ and thus covering all copies of $G_{k-1}$ in $G_k$ and $B_{k-1}$ in $B_{k}$. We do this by setting $\varphi_k (v^\prime)=u^\prime$ if and only if $v^\prime\in \mathcal{L}_p$ is the copy of a vertex $v\in G_{k-1}$ and $u^\prime$ is the copy of $\varphi_{k-1}(v)\in H_{f(p)}$. Then, we map the remaining vertices according to the bijection between $\mathcal{P}_k$ and $S(B_k)$: for each $p\in\mathcal{P}_{k-1}$ and $q\in\mathcal{P}_p$ ($q$ is the copy of some probe $q^\prime\in\mathcal{P}_{k-1}$) map the vertex corresponding to $d_q\in V(G_{k-1})$ to the vertex $v_{S,X}\in B_{k-1}$ where $S=f(p)$ and $X$ is the copy of $f(q^\prime)$ in $S(H_S)$. That yields an isomorphism $\varphi_k$ between $G_k$ and $B_k$ since the images of the neighbors of $d_q$ are the neighbors of $v_{S,X}$ in $B_k$. Finally, define $f_k:\mathcal{P}_k\to S(B_k)$ as $f(u_q)= S\cup \lbrace v_{S,X}\rbrace$ and $f(l_q)= S\cup X$.\qed
\end{prf}

Since both constructions yield the same family of graphs, let us call it $\mathcal{LG}$, it is enough to consider just the Burling graphs for the following proofs. The results hold for both constructions. First, we want to assert that the graphs are indeed triangle-free and that we can find graphs of arbitrarily large chromatic number in $\mathcal{LG}$.


\begin{thm}[Felsner et al. \cite{Fe18}]
All graphs in $\mathcal{LG}=\lbrace B_k :k\in\N\rbrace$ are triangle-free.
\end{thm}

\begin{prf}
We prove this by induction on $k$. The graph $B_1$ is trivially triangle-free. For each $k\geq 2$, the graph $B_k$ is triangle-free since $B_{k-1}$ is triangle-free by inductive hypothesis and the neighborhood of every vertex $v_{S,X}$ we introduce in $B_k$ is a stable set.\qed
\end{prf}

\begin{thm}[Felsner et al. \cite{Fe18}]\label{t2ls}
For all $k\in\N$, the graph $B_k$ has chromatic number $\chi (B_k)$ at least $k$.
\end{thm}

\begin{prf}
We show this by proving a stronger statement by induction on $k$. For any proper coloring $c$ of $B_k$, there exists a stable set $S\in S(B_k)$ such that $c$ uses at least $k$ colors on the vertices in $S$.

This is again trivial for $k=1$. Now, suppose $k\geq 2$ and $c$ is a proper coloring of $B_k$. By induction hypothesis, there exists a stable set $S\in S(B_{k-1})$ $c$ uses at least $k-1$ colors on. Also by inductive hypothesis, there exists a stable set $X\in S(H_S)$ $c$ uses at least $k-1$ colors on. If $c$ uses at least $k$ colors on $(S\cup X)\in S(B_k)$, we are done. Otherwise, $c$ uses the same $k-1$ colors on $S$ and $X$. Since $v_{S,X}$ is adjacent to each vertex in $X$, it is colored in a color not used on a vertex in $S$ and thus $c$ uses at least $k$ colors on $S\cup\lbrace v_{S,X}\rbrace$.\qed 
\end{prf}

\begin{comment}
\begin{thm}[Pawlik et al. \cite{Paw14}]
For every $k\in\N$ exists a family $\mathcal{L}$ of line segments in the plane with no three pairwise intersecting segments and $\chi (G)\geq k$, where $G$ is the intersection graph of $\mathcal{L}$.
\end{thm}

\begin{prf}
 For every $k\in\N$ we now construct a family $\mathcal{L}_k$ of $s_k$ line segments, such that no three are pairwise intersecting, and a family $\mathcal{P}_k$ of $p_k$ pairwise disjoint probes in $R$, such that for every proper coloring $c$ of $\mathcal{L}_k$ there is a probe $P\in\mathcal{P}_k$ for which $c$ uses at least $k$ colors on the line segments of $\mathcal{L}_k$ intersecting $P$, by induction on $k$. Then, the corresponding intersection graph $G_k$ for $\mathcal{L}_k$ is triangle-free and $\chi (G)\geq k$.

For the base case $k=1$ consider an arbitrary non-horizontal line segment in $R$. Let $\mathcal{L}_1$ just contain this segment and choose any rectangle in $R$ that shares its right boundary with $R$ and intersects the line segment with its lower and upper bound as the probe $p$ in $\mathcal{P}_1$. $G_1$ is obviously triangle-free and $p$ uses exactly one color on the segment it contains.

For the induction step, consider a given rectangle $R$ and families $\mathcal{L}_k$ and $\mathcal{P}_k$ for a fixed $k\in\N$. To construct $\mathcal{L}_{k+1}$ draw $\mathcal{L}_k$ in $R$ and for each probe $p\in\mathcal{P}_k$ place another copy $\mathcal{L}_p$ of $\mathcal{L}_k$ with probes $\mathcal{P}_{p}$ in the root of $p$. Afterwards, for each $p\in\mathcal{P}_k$ and every $q\in\mathcal{P}_{p}$ draw the diagonal $d_q$ of $q$, i.e. a line segment from its bottom-left corner to its upper-right corner. $d_q$ crosses all segments intersecting $q$ but no other. Then, $\mathcal{L}_k$ consists of the segments from the $p_k + 1$ copies of $\mathcal{L}_k$ and the $p_k^2$ diagonals, i.e. $\vert \mathcal{L}_{k+1}\vert =(p_k + 1)s_k+p_k^2=s_{k+1}$. Since the copies of $\mathcal{L}_k$ are triangle-free and disjoint and $d_q$ intersects only the segments in $q$, an independent set of segments, $\mathcal{L}_{k+1}$ is also triangle-free.

Now we want to construct $\mathcal{P}_{k+1}$. For each $p\in\mathcal{P}_k$ and every $q\in\mathcal{P}_{p}$ let $\mathcal{L}(p)$ be the set of segments in $\mathcal{P}_k$ intersecting $p$ and $\mathcal{L}_p(q)$ the segments in $\mathcal{L}(p)$ intersecting $q$. We add two probes to $\mathcal{P}_{k+1}$. We place the first one, the upper probe $u_q$, close to the top of $q$, such that the diagonal $d_q$ but no other segment in $\mathcal{L}_p(q)$ intersects it and choose the second probe, the lower probe $l_q$, close to the bottom of $q$ such that it contains all segments in $\mathcal{L}_p(q)$ but not $d_q$. Then, both probes end at the right boundary of $R$. By the induction hypothesis and the way we placed $\mathcal{L}_p$, $\mathcal{L}(p)\cup \lbrace d_q\rbrace$ and $\mathcal{L}(p)\cup\mathcal{L}_p(q)$ are both independent sets, i.e. both $u_q$ and $l_q$ are proper probes. Finally, observe that $\vert \mathcal{P}_{k+1}\vert = 2p_k^2=p_{k+1}$.

Let $c$ be a coloring of $\mathcal{L}_{k+1}$. Consider the restriction of $c$ to the original copy of $\mathcal{L}_k$. By the induction hypothesis, there exists a probe $p\in\mathcal{P}_k$ for which $c$ needs $k$ colors to paint the line segments in $p$. Now, consider the copy $\mathcal{L}_p$ of $\mathcal{L}_k$ in the root of $p$. Again, by induction hypothesis, there is a probe $q\in\mathcal{P}_p$ that uses $k$ colors on the segments in $\mathcal{L}_p$ intersecting $q$. If the colors used by $c$ in $p$ and $q$ are different, at least $k+1$ colors are used on the segments pierced by the lower probe $l_q$. Otherwise, $d_q$ has a different color than the colors used in $p$ and $q$ and thus, $c$ uses $k+1$ colors on the segments pierced by the upper probe $u_q$.\qed
\end{prf}
\end{comment}
Similar to the construction of triangle-free graphs of large chromatic number by Mycielski, there exist rather simple proofs showing that such graphs provided by the construction above contain induced copies of stars and paths as well.
First, note that $B_{k+1}$ contains an induced copy of $B_k$ for all $k\in\N$. This follows directly from the construction. With this knowledge it is enough to prove the existence of an induced copy of a tree $T$ in some $B_k$. Then $T$ is no suitable candidate to disprove the conjecture as it is contained in a graph of arbitrarily large chromatic number.

\begin{thm}
Let $n\in\N$. Then, $B_{n+1}$ contains an induced copy of $K_{1,n}$.
\end{thm}
\begin{prf}
Let $n\in\N$ be fixed and consider $B_n$. By Theorem \ref{t2ls}, there exists a stable set $S\in S(B_n)$ such that any proper coloring $c$ of $B_n$ uses at least $n$ colors on the vertices in $S$. Then $S$ has at least size $n$. Consider the copy $X\in S(H_S)$ of $S$ in $H_S$. By definition, $X$ is a stable set and contains at least $n$ vertices. Thus, the vertex set $X\cup \lbrace v_{S,X}\rbrace$ induces a copy of $K_{1,n}$ in $B_k$.\qed   
\end{prf}

\begin{thm}
Let $n\in\N$, $n\geq 2$. Then, $G_{n+1}$ contains an induced copy of $P_n$.
\end{thm}
\begin{prf}
We prove the existence of an induced copy of $P_n$ in $G_{n+1}$ by induction on $n$. Observe that in each step we choose a diagonal of some probe from $\mathcal{P}_{n-1}$ as the new line segment we add to our path in step $n$.

Launch the induction with $n=2$. Consider $G_2$. We have just one probe $p$ in $\mathcal{P}_1$, i.e. just one diagonal $d_p$ that intersects the copy $l^\prime$ of the line segment $l$ in $\mathcal{L}_1$ in $\mathcal{L}_p$. $Q_n = (l^\prime ,d_p)$ is an induced copy of $P_2$.

For the induction step, let $Q_n$ be an induced copy of $P_n$ in $G_n$ for some fixed $n\in\N$, $n\geq 2$, and construct an induced copy $Q_{n+1}$ of $P_{n+1}$ as follows. Let $q$ denote the probe in $\mathcal{P}_{n-1}$ we chose the diagonal $d_q$ that we use as the last vertex in our path $Q_n$ from. Then, the upper probe $u_q$ in $q$ is a probe in $\mathcal{P}_n$. Now choose an arbitrary probe $p\in\mathcal{P}_n$. Let $Q_n^\prime$ denote the copy of $Q_n$ in $\mathcal{L}_p$. Note that $Q_n^\prime$ is an induced copy of $P_n$ as well. Consider a copy $u_q^\prime$ in $\mathcal{P}_p$ of the probe $u_q$. Its diagonal $d$ intersects the copy of $d_q$ but no other vertex of $\mathcal{L}_p$. Thus, $Q_{n+1}=Q_{n}^\prime d$ is an induced copy of $P_{n+1}$ in $G_{n+1}$.\qed
\end{prf}

Finally, there exists an inductive proof showing that actually for every tree $T$ we can find an intersection graph $G_k$ for some $k\in\N$ that contains an induced copy of $T$. Therefore, let us define a sequence of trees $T_1, T_2, \dots$ where $T_1$ is the tree on one vertex and for all $i\in\N$ let $T_{i+1}$ be obtained from $T_i$ by adding a leaf to each vertex of $T_i$. Denote the vertices of $T_i$ by $t_1, t_2,\dots ,t_{2^i}$, where $t_{2^{i-1}+j}$ is the leaf we added to $t_j$ when constructing $T_i$.

\begin{thm}\label{t1ls}
Let $k\in\N$. The graph $G_k$ contains an induced copy of $T_k$.
\end{thm}

\begin{prf}
We shall prove a stronger statement. We claim that $T_k\subseteq_I G_k$ for some $k\in\N$ and that for each vertex $t\in V(T)$ there exists a probe $p\in\mathcal{P}_k$ containing exactly $t$ but no other vertex from $V(T)$. The proof goes by induction on $k\in\N$. Before the induction starts, let us introduce a notation. Denote the copy of $T_k$ with $C_k$ and its vertices with $c_1^k, c_2^k, \dots , c_{2^k}^k$.

To launch the induction let $k=1$. The tree $T_1$ consists of one single vertex $t_1$. Thus, by defining $c_1^1$ to be the single vertex in $G_1$ we found an induced copy of $T_1$ in $G_1$. Observe that the only probe in $\mathcal{P}_1$ pierces $c_1^1$.

Now let $k>1$ and suppose the statement holds for all $1\leq k^\prime <k$, $k^\prime\in\N$. By the induction hypothesis there exists an induced copy $C_{k-1}$ of $T_{k-1}$ in $G_{k-1}$. Consider $G_k$ and choose one of the $p_{k-1}$ induced copies of $G_{k-1}$. Call it $G_{k-1}^\prime$. Let $c_1^k, c_2^k,\dots , c_{2^{k-1}}^k$ be the copies of the vertices of $C_{k-1}$ in $G_{k-1}^\prime$. Also by induction hypothesis, for each vertex $c_i^{k-1}$, $i\in [2^{k-1}]$, there exists a probe $p_{c_i^{k-1}}\in\mathcal{P_{k-1}}$ containing $c_i^{k-1}$ but no other vertex of $C_{k-1}$. Let $q_i:=q_{c_i^{k-1}}$ denote the copy of this probe in $G_{k-1}^\prime$. Observe that $q_i$ contains $c_i^k$ but no other vertex $c_j^k$, $j\in [2^{k-1}]$, $i\neq j$. Then, we can choose the diagonal of $q_i$ as the leaf $c_{2^{k-1}+i}^k$ because it crosses $c_i^k$ but none of the other vertices, i.e. in the intersection graph $G_k$, $c_{2^{k-1}+i}^k$ is only adjacent to $c_i^k$. Since the probes $q_i$, $i\in [2^{k-1}]$, are pairwise disjoint by the definition of probes, the vertices $c_{2^{k-1}+i}^k$, $i\in [2^{k-1}]$, are pairwise non-adjacent as well. Thus, $C_k$ is an induced tree in $G_k$. Furthermore, with the new probes $u_{q_i}$ and $l_{q_i}$ in $\mathcal{P}_k$ we have one, $u_{q_i}$, that contains only $c_i^k$ and one, $l_{q_i}$, that only contains $c_{2^{k-1}+i}^k$ of the vertices in $C_k$. \qed
\end{prf}

\begin{cor}
For every tree $T$, there is a $k\in\N$ such that $T\subseteq_I G_k$.
\end{cor}
\begin{prf}
Let $T$ be a tree. Denote its radius by $r$ and let $v\in V(T)$ be a vertex such that the distance between $v$ and a vertex $u\in V(T)\setminus \lbrace v\rbrace$ is at most $r$. Define $k=\max_{u\in V(T)\setminus \lbrace v\rbrace} (dist(v,u)+\delta (u))$. Then $T$ is an induced subgraph of $T_k$ and therefore $T\subseteq_I G_k$ holds by theorem \ref{t1ls}.\qed
\end{prf}
  \newpage
  
  \section{Conclusion}\label{secConc}
  In this thesis we examined the Gyárfás-Sumner conjecture and considered some cases for simple graphs like stars and paths. We went on with a result by Kierstead and Rödl showing that $\textit{Forb}(K_{n,n},T)$ is $\chi$-bounded for each tree $T$ and $n\in\N$. Then, we proved some weaker statements. For example, a graph with girth at least $k+1$, $k\in\N$, and minumum degree at least $d\in\N$ contains an induced subgraph isomorphic to $T$ for each tree $T$ on at most $k$ vertices and maximum degree at most $d$.

Afterwards we focused on finding a counterexample for the Gyárfás-Sumner conjecture by checking the Mycielski construction and the Burling construction if one of the families of graphs provided by these constructions is a subset of $\textit{Forb}(T)$ for a tree $T$. Since the result was negative, we are now able to eliminate both constructions as candidates for disproving the Gyárfás-Sumner conjecture. This also applies to the modified construction we derived from Burlings construction.% Another approach to find a counterexample could be to combine these results and try to come up with a construction of graphs with arbitrarily large chromatic number where vertices of a high degree have a certain distance such that these graphs would not contain a subgraph isomorphic to 

In contrast, there are many cases of the Gyárfás-Sumner conjecture that are still open. It may be worth a try to prove the conjecture for trees obtained from three stars joined by a star through their center vertices, an arbitrary caterpillar tree or, although that seems hard, trees of a certain radius of at least three.


  \newpage


  \listoffigures
  \newpage
  % Literaturverzeichnis (beginnt auf einer ungeraden Seite)
  %\begin{thebibliography}{Lam00}
  %\end{thebibliography}
 
  \bibliography{/Users/Frithjof/Documents/studium/ba/arbeit/bib/refs.bib}
  \bibliographystyle{ieeetr}
      
  % ggf. hier Tabelle mit Symbolen 
  % (kann auch auf das Inhaltsverzeichnis folgen)

  \newpage
  
  \thispagestyle{empty}


  \vspace*{8cm}


  \section*{Declaration}

  %Hiermit versichere ich, dass ich diese Arbeit selbständig verfasst und keine anderen,  %als die angegebenen Quellen und Hilfsmittel benutzt, die wörtlich oder inhaltlich %übernommenen Stellen als solche kenntlich gemacht und die Satzung des Karlsruher %Instituts für Technologie zur Sicherung guter wissenschaftlicher Praxis in der jeweils %gültigen Fassung beachtet habe. 
I hereby declare that I have written this thesis independently and did not use sources or means other than those specified in the text
%, identified literally or substantively adopted passages,
 and respected the statutes of the Karlruhe Institute of Technology for protection of good scientific practice in its valid version.
\\[2ex] 





  \noindent
Karlsruhe, the Date \\[5ex]

% Unterschrift (handgeschrieben)



\end{document}

