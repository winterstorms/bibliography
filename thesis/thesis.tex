% Vorlage für eine Bachelorarbeit
% Siehe auch LaTeX-Kurs von Mathematik-Online
% www.mathematik-online.org/kurse
% Anpassungen für die Fakultät für Mathematik
% am KIT durch Klaus Spitzmüller und Roland Schnaubelt Dezember 2011

\documentclass[12pt,a4paper]{scrartcl}
% scrartcl ist eine abgeleitete Artikel-Klasse im Koma-Skript
% zur Kontrolle des Umbruchs Klassenoption draft verwenden


% die folgenden Packete erlauben den Gebrauch von Umlauten und ß
% in der Latex Datei
\usepackage[utf8]{inputenc}
\usepackage[T1]{fontenc}
\usepackage[english]{babel}
\usepackage{verbatim}

\usepackage{graphicx}
\graphicspath{{images/}}
\usepackage{latexsym}
\usepackage{amsmath,amssymb,amsthm}

\usepackage[shortlabels]{enumitem}
\usepackage{cite}

% Abstand obere Blattkante zur Kopfzeile ist 2.54cm - 15mm
\setlength{\topmargin}{-15mm}

%nach Absaetzen Hoehe eines 'x' Abstand
%\setlength{\parskip}{0.5ex}


% Umgebungen für Definitionen, Sätze, usw.
% Es werden Sätze, Definitionen etc innerhalb einer Section mit
% 1.1, 1.2 etc durchnummeriert, ebenso die Gleichungen mit (1.1), (1.2) ..
\theoremstyle{plain}
\newtheorem{thm}{Theorem}[section]
\newtheorem{con}[thm]{Conjecture}
\newtheorem{lemma}[thm]{Lemma}
\newtheorem{cor}{Corollary}[thm]

\theoremstyle{definition}
\newtheorem{defn}[thm]{Definition} 
\newtheorem*{prf}{Proof}
\newtheorem{notn}[thm]{Notation}
\newtheorem{note}[thm]{Observation}

\renewenvironment{proof}{{\bfseries Proof}}{\qed}
\numberwithin{equation}{section} 

% einige Abkuerzungen
\newcommand{\C}{\mathbb{C}} % komplexe
\newcommand{\K}{\mathbb{K}} % komplexe
\newcommand{\R}{\mathbb{R}} % reelle
\newcommand{\Q}{\mathbb{Q}} % rationale
\newcommand{\Z}{\mathbb{Z}} % ganze
\newcommand{\N}{\mathbb{N}} % natuerliche

\usepackage{xpatch}
\makeatletter
\AtBeginDocument{\xpatchcmd{\@thm}{\thm@headpunct{.}}{\thm@headpunct{}}{}{}}
\makeatother


\begin{document}
  % Keine Seitenzahlen im Vorspann
  \pagestyle{empty}

  % Titelblatt der Arbeit
  \begin{titlepage}

\begin{figure}[h]
\includegraphics[scale=0.5]{KITLogo_RGB_eng} 
   %\includegraphics[scale=0.45]{kit-logo} 
\end{figure}
  
    \vspace*{2cm} 

 \begin{center} \large 
    
    Bachelor Thesis
    \vspace*{2cm}

    {\huge Bounding Chromatic Number of Graphs with Forbidden Induced Subgraphs}
    \vspace*{2.5cm}

    Frithjof Marquardt
    \vspace*{1.5cm}

    Due Date
    \vspace*{4.5cm}


    Advisor: Name of Advisor \\[1cm]
    Faculty of Mathematics \\[1cm]
		Karlsruhe Institute of Technology
  \end{center}
\end{titlepage}

Abstract
\newpage
  % Inhaltsverzeichnis
  \tableofcontents

\newpage
 


  % Ab sofort Seitenzahlen in der Kopfzeile anzeigen
  \pagestyle{headings}

\section{Introduction}
Proving the conjecture as a whole is obviously a complex endeavor. That is why many proofs concentrate on certain graphs and prove it only partially. Therefore,


Finally, Chapter \ref{secConc} provides a summary of the results we proved throughout this thesis and gives an outlook which cases of the Gyárfás-Sumner conjecture till remain open.
\newpage
\section{Basic Concepts}
This chapter focuses on providing the basis for all following theorems and proofs by displaying most of the definitions regularly used in this thesis. They are collected and structured starting with the simple ones and advancing to more complex and specific concepts. This should give an overview and a better understanding of the methods we will use later on.\\

We will start with defining the concept of a graph. For a set $X$, let us denote the set of all $2$-element subsets of $X$ with ${X\choose 2} =\lbrace \lbrace x,y\rbrace : x,y\in V, x\neq y\rbrace$. A \textit{graph} $G$ is a tuple $G=(V,E)$, where $V=V(G)$ denotes the set of vertices and $E=E(G)\subseteq {V\choose 2}$ is the set of edges. Here, each edge is a set of two vertices from $V$.

By giving each edge a direction, i.e. saying the edge $(x,y)$ points from $x$ to $y$, we create a \textit{directed graph} $D=(V,E)$ where $V$ again denotes the set of vertices but $E\subseteq \lbrace (x,y) : x,y\in V, x\neq y\rbrace$ is now a set of tuples. In a directed graph, there can exist two edges between two vertices $x$ and $y$: one pointing to $x$ and one pointing to $y$. We call a directed graph $D$ also an \textit{oriented graph} if there exists at most one edge for any pair of vertices in $D$. See Figure \ref{f1ba} for an illustration of the differences between undirected, directed and oriented graphs. 

Note that any of the following concepts and definitions can be applied to directed graphs as well.

\begin{figure}[ht]
\begin{center}
\includegraphics[scale=1]{undir_dir_ori}
\end{center}
\caption{Undirected graph (left), directed graph (center) and oriented graph (right) on the same vertex set}
\label{f1ba}
\end{figure}

In a graph $G=(V,E)$ two vertices are \textit{adjacent} if they are neighbors, i.e. there exists an edge between them. Then, the edge and the vertices are called \textit{incident}. We denote the neighbors of a vertex $v\in V$ in $G$ as $N_G(v)$ or just $N(v)$ if there is no confusion in regard of the underlying graph. Furthermore, the \textit{degree} $d(v)$ of $v$ is defined as $d(v) =\vert N(v)\vert$. The minimum degree in a graph is denoted by $\delta (G)$, the maximum degree by $\Delta (G)$. Since edges in a directed graph do not necessarily all point in the same direction, there we distinguish between the \textit{in-degree} $d^{in}(v)$ of a vertex $v$ which counts the number of edges ending in $v$ and the \textit{out-degree} $d^{out}(v)$ which counts the number of edges starting at $v$. This concept also applies to the minimum and maximum degree of a directed graph.\\

After introducing the concept of a graph and the relations between edges and vertices we now start looking for structures of a greater scale in it either by considering just the vertices and edges or later by additionally defining colorings of the graph. 

Let $G=(V,E)$ be a graph. For a vertex set $A\subseteq V$ let us denote the set of edges in $B\subseteq E$ induced by $A$ as $B\vert_A=\lbrace \lbrace x,y\rbrace\in B : x,y\in A\rbrace$. Then, a graph $H=(V^\prime ,E^\prime )$ is a \textit{subgraph} of $G$, written $H\subseteq G$, if $V^\prime\subseteq V$ and $E^\prime\subseteq E\vert_{V^\prime}$. The graph $H$ is called an \textit{induced subgraph} of $G$ and we write $H\subseteq_I G$ if $E^\prime = E\vert_{V^\prime}$. The subgraph $(A,E\vert_A)$ induced in $G$ by a certain set of vertices $A\subseteq V$ is denoted as $G[A]$. Furthermore, the structures of two graphs can coincide completely: Two graphs $G$ and $H$ are called \textit{isomorphic} if there exists a bijection $\phi :V(G)\to V(H)$ such that $\lbrace\phi (u), \phi (v)\rbrace\in E(H)$ if and only if $\lbrace u, v\rbrace\in E(G)$. We denote such a relation as $G\approx H$.

A subset $S\subseteq V$ of pairwise non-adjacent vertices is called a \textit{stable} or \textit{independent} set. Opposite to that, a \textit{clique} is a set of vertices $Q\subseteq V$ such that all vertices in $Q$ are pairwise adjacent. Observe that for a stable set $S$ $E\vert_S =\emptyset$ holds while we have $E\vert_Q={Q\choose 2}$ for a clique $Q$. The size of the largest clique in $G$ is denoted by $\omega (G)$. A \textit{coloring} $c$ of $G$ with $r$ colors is a function $c:V\to [r]$ that maps a color to each vertex. It is a proper coloring if for all edges $e = \lbrace x,y\rbrace\in E: c(x)\neq c(y)$, i.e. no two adjacent vertices have the same color. Then, we can define the chromatic number $\chi (G)$ of $G$ as the minimum number of colors needed for a proper coloring of $G$. Observe that each color class of a proper coloring, i.e. all vertices of the same color, is a stable set. If $G$ is a directed graph, its chromatic number is defined as the chromatic number of the underlying undirected graph, i.e. the undirected graph on the same vertex set, where the edge set contains an edge $\{x,y\}$ if and only if $(x,y)$ or $(y,x)$ or both are in the edge set of $G$.

A way of obtaining a proper coloring for a graph $G$ that always works is the \textit{greedy coloring}: Order the vertices $v_1, v_2, \dots ,v_{\vert V(G) \vert }$ and define a coloring $c$ by assigning each vertex the first available color, i.e. the first color not used on a neighbor with smaller index. Note that it often is far from optimal regarding the number of used colors.\\

Since the crucial part of this thesis deals with the conjecture of Gyárfás and Sumner, we want to introduce the notion of $\chi$-bounded families of graphs. In addition, this part states the definitions of a hypergraph and the Ramsey number because they are used in many proofs in the following chapters.

\begin{defn}[Bounding function]\label{d1}
A $\chi$\textit{-bounding function} $f:\N\to\N$ for a family $\mathcal{F}$ of graphs is a function such that $\chi (G)\leq f(\omega (G))$ holds for each $G\in\mathcal{F}$. Such a function does not necessarily exist for each family $\mathcal{F}$. If it exists though, $\mathcal{F}$ is called $\chi$-bounded. 
\end{defn}

A \textit{hypergraph} $H=(V,E)$ is a tuple where $V$ is a set of vertices and $E$ a set of non-empty subsets of $V$, called hyperedges. If each edge has the same size $s$, $H$ is an $s$-uniform hypergraph. Note that the concepts of colorings also apply for hypergraphs. We define a coloring $c$ of the hypergraph $H$ with $r$ colors as a function $c:V(H)\to [r]$ that assigns each vertex a color. A hypergraph coloring $c$ is proper if there exists no monochromatic edge, i.e. there is no edge $e\in E(H)$ with $c(v)=\alpha$ for all $v\in e$ and some color $\alpha\in [r]$.

The Ramsey number $R=R(r_1, r_2, \dots ,r_s)$ for integers $r_1, r_2, \dots r_s$ is the minimum number $R\in\N$, such that for every coloring $c:{[R]\choose 2}\to [s]$, there exist a color $\alpha\in [s]$ and a subset $X\subseteq [R]$, $\vert X\vert =r_{\alpha}$, with $c(\lbrace i,j\rbrace )=\alpha$ for all $\lbrace i,j\rbrace\in {X\choose 2}$. Note that the definition of the Ramsey number is based on edge colorings while the colorings of graphs we defined so far are vertex colorings.\\

The next definitions are a collection of the most used special graphs or characterizations of some graphs that appear throughout this thesis.

Consider a graph $G=(V,E)$ with $\vert V\vert = n$. If $E=\emptyset$, we call $G$ the empty graph on $n$ vertices. If $E= {V\choose{2}}$, $G$ is also denoted as $K_n$, the complete graph on $n$ vertices. Note that then a clique of size $k$ is an induced copy of a $K_k$. The graph $G$ is called bipartite if $V=A\dot{\cup} B$ for two disjoint stable sets $A$ and $B$. The complete bipartite graph $K_{m,n}$ is a bipartite graph with $\vert A\vert =m$, $\vert B\vert =n$ and $E=\lbrace \lbrace a,b\rbrace :a\in A,b\in B\rbrace$.

A path of length $n\in\N$ is a graph $P=(V,E)$ on $n+1$ vertices, where we can order the vertices $v_0, v_1, \dots ,v_n$ such that $E=\lbrace\lbrace v_{i-1},v_i\rbrace : i\in [n]\rbrace$. We say it starts in $v_0$ and ends in $v_n$. We obtain a \textit{cycle} $C=(V,E^\prime)$ of length $n$ from a path $P=(V,E)$ of length $n$ by adding the edge $\lbrace v_0,v_n\rbrace$ to the edge set, i.e. $E^\prime = E\cup\lbrace v_0,v_n\rbrace$. A graph that does not contain a cycle as a subgraph is \textit{acyclic}. We say a graph $G=(V,E)$ is \textit{connected}, if there exist a path starting at $u$ and ending at $v$ for each pair of vertices $u,v\in V$.

With the previous definition we define a \textit{tree} $T$ as a connected and acyclic graph. A union of disjoint trees is called a \textit{forest}.

Additionally, we can group trees according to their form. We denote a \textit{path} on $n\in\N$ vertices as $P_n$. In the special case $n=1$, $P_1$ contains just a single vertex and no edge. A \textit{star} $K_{1,n}$ on $n+1$ vertices is a tree with one center vertex of degree $n$ and $n$ leaves, i.e. vertices of degree $1$. A \textit{caterpillar} is a tree $T$ that has a subgraph $H\subseteq T$ such that $H$ is isomorphic to a path and $H$ contains all vertices in $V(T)$ of degree at least two.  %We refer to the leaves of $T$ that are not contained in $H$ as the legs of $T$. 
Figure \ref{f3ba} gives an example of all three types of trees.

\begin{figure}[ht]
\begin{center}
\includegraphics[scale=1]{trees}
\end{center}
\caption{A path $P$, a star $S$ and a caterpillar $T$}
\label{f3ba}
\end{figure}

A \textit{subdivision} of a graph $G=(V,E)$ is a graph resulting from subdivisions of edges in $G$. Subdividing an edge $e=\lbrace u,v\rbrace\in E$ yields a graph $G^\prime =(V\cup \lbrace w\rbrace, E^\prime )$ with $E^\prime =(E\setminus \lbrace e\rbrace ) \cup \lbrace \lbrace u,w\rbrace, \lbrace v,w \rbrace\rbrace$. Informally, we add a new vertex $w$ and replace $e$ with two new edges $\lbrace u,w\rbrace$ and $\lbrace w,v\rbrace$ as depicted in Figure \ref{f2ba}. Observe that a subdivision of a tree is still a tree.\\

\begin{figure}[ht]
\begin{center}
\includegraphics[scale=1]{subdivision}
\end{center}
\caption{Subdivision of an edge $e$ in a graph $G$}
\label{f2ba}
\end{figure}


For the sake of the completeness of this collection of definitions, we add a final one, since it is used to describe some trees already proven to be $\chi$-bounding. Again, let $G=(V,E)$ be a graph. Let the distance $dist(u,v)$ between two vertices $u,v\in V$ denote the length of the shortest path in $G$ starting at $u$ and ending at $v$. The eccentricity of a vertex $v\in V$ is defined as $\displaystyle e(v)=\max_{u \in V}\lbrace dist(v,u)\rbrace$. Then, the radius $\displaystyle r= \min_{v\in V}\lbrace e(v)\rbrace$ of $G$ is the minimum number $r\in\N$, such that there exists a vertex $v\in V$ where the distance between $v$ and any other vertex $u\in V$ is at most $r$. We often name this vertex $v$ the root of the tree.
\newpage 
\section{Current State of Research}

Let $G$ be a graph and denote its chromatic number by $\chi (G)$ and its clique number by $\omega (G)$. For a graph $H$ define the family of graphs not inducing $H$ as a subgraph as $\textit{Forb}(H)$. For which $H$ can we find a function $f$ such that $\chi (G)\leq f(\omega (G))$ holds for every graph $G\in\textit{Forb}(H)$? I.e., by the notation in \ref{d1}, we want to find a bounding function for $\textit{Forb}(H)$. If $f$ exists, let us call $H$ $\chi $-bounding. Observe that if $H$ is $\chi $-bounding, $H$ must be acyclic. That is because of a result by Erd\H{o}s and Hajnal in \cite{EH66} showing the existence of graphs with arbitrarily high girth and chromatic number. Thus, if $H$ contains a cycle, we can always find a graph $G$ with girth greater than the longest cycle in $H$ and a large chromatic number. Therefore, it is impossible to find a bounding function for $\textit{Forb}(H)$. For every other graph the statement may hold. Indeed, Gyárfás \cite{Gy75} and Sumner \cite{Su81} independently conjectured that it is true for every forest.

\begin{con}[Gyárfás \cite{Gy75}, Sumner \cite{Su81}]\label{c1cr}
Every forest $F$ is $\chi$-bounding.
\end{con}

Actually, Gyárfás and Sumner conjectured that for every clique $K$ and tree $T$ the family of graphs neither inducing $K$ nor $T$ is $\chi$-bounded. It is not hard to see that both statements are equivalent. We claim that the problem as formulated in the conjecture \ref{c1cr} can be reduced to trees since a forest is $\chi$-bounding if and only if all its components are $\chi$-bounding. Then, \ref{c1cr} just states the Gyárfás-Sumner conjecture using the definitions of the previous chapter. 

A proof of this claim using induction on the number of connected components is mentioned in a paper by Martin \cite{Ma16}. The following proof is slightly modified to match the version of the Gyárfás-Sumner conjecture stated above.

\begin{thm}
A forest $F$ is $\chi$-bounding if and only if all its connected components are $\chi$-bounding.
\end{thm}

\begin{prf}
Let $F$ be a $\chi$-bounding forest with bounding function $f$. Assume, there exists a connected component $T$ of $F$ that is not $\chi$-bounding. Then any graph in $\textit{Forb}(T)$ is also in $\textit{Forb}(F)$ and there exist an $\omega \in\N$ and a graph $G\in\textit{Forb}(T)\subseteq\textit{Forb}(F)$ with $\omega (G) =\omega $ and $\chi (G)>f(\omega )$. This is a contradiction to $F$ being $\chi$-bounding.

To show the other direction, let $F$ be a forest with connected components $T_1, T_2,\dots ,T_k$ where $\textit{Forb}	(T_i)$ is $\chi$-boundedwith bounding function $f_{T_i}$. Assume, $F$ is not $\chi$-bounding. Then, there exists an $\omega\in\N$ such that for all $m\in\N$ there exists a $G\in\textit{Forb}(F)$ with $\omega (G)=\omega$ and $\chi (G)> m$. Choose a graph $G$ with $\chi (G)>\max_{i\in [k]}f_{T_i}(\omega)$. Since $G\in\textit{Forb}(F)$ there exists an $i\in [k]$ such that $G\in\textit{Forb}(T_i)$. This contradicts the assumption that $T_i$ is $\chi$-bounding.\qed
\end{prf}

Let $\mathcal{H}$ be a set of graphs and let $\textit{Forb}(\mathcal{H})$ be the family of graphs forbidding any graph in $\mathcal{H}$ as induced subgraph. The conjecture \ref{c1cr} covers only cases where $\vert\mathcal{H}\vert =1$.
Now consider a family $\textit{Forb}(\mathcal{H})$ with $\vert\mathcal{H}\vert >1$. Note that every finite set containing a forest is $\chi$-bounding under the assumption that conjecture \ref{c1cr} is true and if $\mathcal{H}$ is finite and $\chi$-bounding it must contain a forest because of the result by Erd\"os and Hajnal \cite{EH66}. If $\mathcal{H}$ does not contain a forest, it has to be infinite. Thus, considering infinite sets of cycles is the next obvious step. In 1985, Gyárfás made three more conjectures (later published in \cite{Gy87}) concerning the relation between the chromatic number of a graph and the length of cycles it induces:

\begin{enumerate}[(i)]
\item The family of graphs with no induced cycle of odd length is $\chi$-bounded.
\item Let $l\in\N$. The family of graphs with no induced cycle of length at least $l$ is $\chi$-bounded.
\item Let $l\in\N$. The family of graphs with no induced cycle of odd length at least $l$ is $\chi$-bounded.
\end{enumerate}
Observe that the third conjecture implies the first two but since they were stated and proven independently each seems to be worth a nomination on its own. Chudnovsky, Scott and Seymour dealt with these conjectures in a series of papers including three (\cite{SS14}, \cite{CSS15} and \cite{CSSS17}) in which they are proving them.
\\

Returning to the case were $\vert\mathcal{H}\vert =1$, it might be interesting to note that the conjecture becomes a rather simple problem in the case of general subgraphs instead of induced subgraphs. There exists a proof by Gyárfás, Szemeredi and Tuza from 1980 \cite{GST80} showing that a labeled tree $T$ on $k$ vertices is a subgraph of any graph with chromatic number at least $k$. A \textit{labeled} graph is a graph with an assignment of labels, often represented by integers, to its vertices. When talking about a labeled graph $G$ without defining an explicit labeling, we refer to a graph labeled such that each vertex has a unique label. Usually each vertex has a label $i\in [\vert V(G)\vert ]$. A pair of labeled graphs is isomorphic if there exists a graph isomorphism that also heeds the assigned labels besides the structure of the graph.

An obvious consequence of this theorem is that the family of graphs that do not contain $T$ as a subgraph is $\chi$-bounded and for the bounding function $f$ holds $f(\omega )\equiv k$.

\begin{thm}[Gyárfás, Szemeredi, Tuza \cite{GST80}]\label{t3cr}
Let $k\in\N$ and $G$ be a graph with $\chi (G) = k$. Then, $G$ contains every labeled tree $T$ on $k$ vertices as a subgraph. 
\end{thm}
\begin{prf}
We proof this by induction on $k$. For the base case let $k=1$. A graph $G$ with $\chi (G)=1$ trivially contains a labeled tree on a single vertex.

Now let $k>1$ and suppose the statement holds for all $1\leq k^\prime <k$. Consider a graph $G$ with $\chi (G) = k$ and a proper coloring $c$ in $k$ colors. Let its vertices be labeled with $1, 2, \dots , k$ according to their color under $c$. Let $T$ be a labeled tree on $k$ vertices. Choose a leaf $p$ of $T$ and call its neighbor $q$. Denote the label of $p$ as $l_p$ and the label of $q$ as $l_q$. Let $A$ be the set of vertices of $G$ with color $l_q$ such that each vertex in $A$ has a neighbor with color $l_p$. The set $A$ is not empty since $\chi (G)=k$. Consider the subgraph $G^\prime$ of $G$ where we remove the vertices of the color class $l_p$ and the vertices of color class $l_q$ that are not in $A$. It is a graph with chromatic number $k-1$ and by induction hypothesis it contains a labeled tree $T^\prime$ isomorphic to $T-\lbrace p\rbrace$. Since the vertex mapped to $q$ under the isomorphism is element of $A$, it has a neighbor $x\in V(G)$ with color $l_p$ and $T^\prime\cup\lbrace x\rbrace$ is isomorphic to $T$.\qed
\end{prf}
\begin{cor}
A graph $G$ with $\chi (G) = k$ contains every tree $T$ on $k$ vertices as a subgraph. 
\end{cor}
\begin{prf}
A tree $T$ on $k$ vertices can be seen as a labeled tree on $k$ vertices. Thus, $T\subseteq G$ holds by theorem \ref{t3cr}.\qed
\end{prf}

The actual Gyárfas-Sumner conjecture remains open. Even though many tried to prove the conjecture in over forty years there are only some trees found to be $\chi$-bounding so far. Here is a list of them as stated by Chudnovsky, Scott and Seymour in \cite{CSS17} including the cases they proved in the same paper:

\begin{itemize}
\item trees of radius at most two (Kierstead and Penrice \cite{Ki94}),
\item trees obtained from a tree with radius at most two by subdividing every edge incident to the root exactly once (Kierstead and Zhu \cite{Ki04}),
\item subdivisions of stars (direct result of the topological version of the Gyárfás-Sumner conjecture proved in \cite{Sc97}: for every tree $T$ there is a function $f$ such that $\chi (G)\leq f(\omega (G))$ for every graph $G$ not inducing any subdivision of $T$.),
\item trees obtained by adding one vertex to a subdivided star including e.g. two-legged caterpillars (Chudnovsky, Scott and Seymour \cite{CSS17}),
\item trees obtained from a sudivided star and a star by adding a path joining their centers (Chudnovsky, Scott and Seymour \cite{CSS17}).
\end{itemize} 

To gain an idea of how these proofs work I want to present some of them since they also serve as basis for the results of this thesis. The two following theorems proved by Gyárfás in \cite{Gy87} show that the families of graphs $\textit{Forb}(K_{1,n})$ and $\textit{Forb}(P_n)$ are $\chi$-bounded and even provide a lower bound to the bounding function in the first case.

\begin{thm}[Gyárfás \cite{Gy87}] The family of graphs $\textit{Forb}(K_{1,n})$ is $\chi$-bounded for all fixed $n \in\N$, $n\geq 2$ with bounding function $f_n$ satisfying \[\dfrac{R(n,\omega + 1) - 1}{n-1}\leq f_n(\omega )\leq R(n,\omega ).\] 
\end{thm}
\begin{prf}
We prove the lower bound first. By definition of the Ramsey number, there is a graph on $R(n,\omega + 1) - 1$ vertices that contains no stable set of $n$ vertices and no clique of $\omega + 1$ vertices. Let $G$ be such a graph. Then, $G$ does not contain a $K_{1,n}$, since its leafs would form a stable set of $n$ vertices and $\chi (G)\geq |V(G)| /(n-1)$ holds because every color class is a stable set. This gives the lower bound.

To prove the upper bound, let $G \in\textit{Forb} (K_{1,n})$ and $\omega (G) = \omega$. Assume, there exists a vertex $v\in V(G)$ with degree $R(n,\omega)$. Then, the neighborhood of $v$ either contains a stable set of $n$ vertices or a clique of $\omega$ vertices. In the first case, the stable set and $v$ would induce a $K_{1,n}$ in $G$. The second case contradicts $\omega (G) = \omega$. Thus, the maximum degree $\Delta (G)$ of $G$ is less than $R(n,\omega)$, i.e. a greedy coloring of $G$ yields $\chi (G)\leq\Delta (G) + 1\leq R(n,\omega)$.\qed
\end{prf}

\begin{thm}[Gyárfás \cite{Gy87}]
The family of graphs $\textit{Forb}(P_n)$ is $\chi$-bounded with bounding function $f_n(\omega )\leq (n-1)^{\omega - 1}$.
\end{thm}

\begin{prf}
Let $n\in\N$ be fixed and prove the claim by induction on $\omega (G)$. Consider the base case for $\omega (G) =1$. Then, $\chi (G) = 1$ and the theorem trivially holds.

Suppose the theorem holds for all graphs $G^\prime\in\textit{Forb}(P_n)$ with $\omega (G^\prime ) \leq t$ for some $t\geq 1$, i.e. $(n-1)^{t-1}$ is a bounding function for them. Let $G\in\textit{Forb}(P_n)$ with $\omega (G) = t + 1$. Now, assume that $\chi (G) > (n-1)^{\omega - 1}$. We shall reach a contradiction by constructing an induced path $Q_n=(q_1,q_2,\dots ,q_n)$ in $G$. To do this, we will define subgraphs $G_i$ of $G$ with $q_i\in V(G_i)$ for all $i\in [n]$, such that $V(G_1)\supseteq V(G_2)\supseteq \dots\supseteq V(G_n)$ and each $G_i$ satisfies the following properties:
\begin{enumerate}[(i)]
\item $G_i$ is connected;
\item $\chi (G_i) > (n-i)(n-1)^{t-1}$;
\item for $1\leq j<i$ and $v\in V(G_i)$ $q_jv$ is an edge in $G$ if and only if $i=j+1$ and $v = q_i$.
\end{enumerate}
Let $G_1$ be a connected component of $G$ with $\chi (G_1) >(n-1)^t$ and let $q_1$ be any vertex of $G_1$. For $i>1$ assume that $G_1,G_2,\dots , G_{i-1}$ and $q_1,q_2,\dots , q_{i-1}$ are already defined and satisfy the wanted properties. Let $A$ denote the neighborhood of $q_{i-1}$ in $G_{i-1}$ and let $B=V(G_{i-1})\setminus (A\cup\{q_{i-1}\})$. The graph $G[A]$ induced by $A$ in $G$ satisfies $\omega (G[A])\leq t$ because a greater clique and $q_{i-1}$ would induce a clique greater than $t+1$ in $G$. Thus, we have $\chi (G[A])\leq (n-1)^{t-1}$ by induction hypothesis.

Assume, $B\neq\emptyset$. Observe that $\chi (G_{i-1})\leq \chi (G[A]) + \chi (G[B])$ since proper colorings of $G[A]$ and $G[B]$ in distinct colors and assigning $q_i$ any color used in $V(G[B])$ define a proper coloring of $G_{i-1}$. Thus, \[\chi (G[B])\geq \chi (G_{i-1})-\chi (G[A])>(n-i+1)(n-1)^{t-1}-(n-1)^{t-1}=(n-i)(n-1)^{t-1}.\]
Therefore, there exists a connected component $H$ of $G[B]$ with $\chi (H)>(n-i)(n-1)^{t-1}$. Recall, that $G_{i-1}$ is connected, i.e. there also exists a $q_i\in A$ such that $V(H)\cup \{q_i\}$ induces a connected subgraph, which we choose as $G_i$. Then,  $G_i$ obviously satisfies $(i)$ and $(ii)$. Since the only edge between $V(Q_{i-2})$ and $V(G_{i-1})$ is $q_{i-1}q_{i-2}$, there is no edge at all between $V(Q_{i-2})$ and $q_i$ and by definition the edge $q_{i-1}q_i$ exists. Thus, $(iii)$ is satisfied as well. 

If $B=\emptyset$, $\chi (G_{i-1})\geq \chi (G[A]) + 1$. Therefore, $(n-i+1)(n-1)^{t-1}<(n-1)^{t-1} +1$ holds. It follows that $i=n$ and we can choose $q_n$ as any vertex of $A$ because $A\neq\emptyset$ by properties $(i)$ and $(ii)$. With $G_n=\{q_n\}$ we are done.\qed
\end{prf}

The following results by Kierstead and Rödl \cite{Ki96} do not prove any cases of the Gyárfás-Sumner conjecture. Nevertheless, they show the existence of a bounding function for the family of graphs $\textit{Forb}(T,K_{n,n})$ not inducing $T$ and $K_{n,n}$ for a tree $T$ and $n\in\N$. Despite being a weaker statement than the conjecture itself it proves quite useful in the following chapters: For proving or disproving the conjecture we only need to consider families of graphs $F$ such that for each $n\in\N$ there exist many $G\in\mathcal{F}$ inducing a $K_{n,n}$.


Before presenting their main theorem, we need to prove three preliminary lemmas. They use some hypergraph theory in their proofs, so we want to introduce it as well. Let $H=(V,E)$ be a hypergraph. A hypergraph $G=(V,E^\prime)$ is a \textit{covering hypergraph} of $H$ if for every edge $e\in E$ there exists an edge $e^\prime\in E^\prime$ such that $e^\prime\subseteq e$. 

They define a \textit{rooted hypergraph} as a triple $H=(V,E,r)$, where $(V,E)$ is a hypergraph and $r:E\to V$ a function, such that $r_e=r(e)\in e$ for all $e\in E$. We call $r_e$ the \textit{root} of $e$. The \textit{breadth} $b(H)$ of a rooted hypergraph $H$ is the smallest $b$ such that for every vertex $v\in V$ there exist at most $b$ edges $e_1,e_2,\dots ,e_b$ with $r(e_i)=v$, $i\in [b]$ and $e_i\cap e_j= \{v\}$ for $1\leq i<j\leq b$.

\begin{lemma}\label{l1cr}
If $G$ is a covering graph of a hypergraph $H$, then $\chi (H)\leq\chi (G)$.
\end{lemma}
\begin{prf}
Any proper coloring of $G$ is a proper coloring of $H$.\qed
\end{prf}

\begin{lemma}\label{l2cr}
If $G$ is a directed graph, $\chi (G)\leq 2\Delta^{out}(G) +1$.
\end{lemma}
\begin{prf}
The average in- and out-degree of $G$ are both at most $\Delta^{out}(G)$. Thus, the vertices of $G$ can be ordered as $v_1,v_2,\dots ,v_n$  such that $v_i$ has at most $2\Delta^{out}(G)$ neighbors $v_j$ with $j <i$ for each $i\in [n]$. A greedy coloring then needs at most $2\Delta^{out}(G) +1$ colors.\qed
\end{prf}

Observe that for a directed graph $G=(V,E)$, we can form a rooted graph $G^\prime =(V,E^\prime ,r)$ by setting $E^\prime =\{\{ x,y\} :(x,y)\in E \textit{ or } (y,x)\in E\}$ and for $\{ x,y\}\in E^\prime$, $r(\{ x,y\} )=x$ if and only if $(x,y)\in E$. Then, $\Delta^{out}(G)=b(G^\prime )$ and thus, $\chi (G)\leq 2b(G^\prime ) +1$.

\begin{lemma}\label{l3cr}
For a rooted s-uniform hypergraph $H$ holds $\chi (H)\leq 2(s-1)b(H) +1$.
\end{lemma}
\begin{prf}
Let $H=(V,E,r)$ be a rooted s-uniform hypergraph. For each vertex $v\in V$, let $C(v)=\{ e_1,e_2,\dots ,e_n\}$ be a maximum set of edges such that $r(e_i)=v$ for $i\in [n]$ and $e_i\cap e_j = \{ v\}$ for $1\leq i<j\leq n$. Then, $n\leq b(H)$. Define a directed graph $G=(V,D)$ with $D=\{(v,w): v =r_e \textit{ for some } e\in E\textit{ and } w\in (e_i -\{v\}) \textit{ for some } e_i\in C(v)\}$. Now, consider any edge $e\in E$. If $e\in C(r_e)$, let $w$ be any vertex in $(e-\{ r_e\})$; otherwise there exists an edge $f\in C(r_e)$ such that there exists a $w\in (e-\{ r_e\})\cap f$. Then, $(r_e,w)\in D$ and $\{r_e,w\}\subseteq e$. Hence, $G$ is a covering graph of $H$. Since $\Delta^{out}(G)\leq (s-1)b(H)$, $\chi (H)\leq\chi (G)\leq 2(s-1)b(H) +1$ by the previous lemmas \ref{l1cr} and \ref{l2cr}.\qed
\end{prf}

In the proof of the main theorem by Kierstead and Rödl they use an application of the Ramsey theorem not yet mentioned. Let $B=B(s,t,r)$ be the Ramsey function such that for any funtion $c:X\times Y\to [r]$, where $\vert X\vert =\vert Y\vert = B$, there exist subsets $X^\prime\subset X$, $Y^\prime\subset Y$ and a color $\alpha\in [r]$ satisfying $\vert X^\prime\vert = s$, $\vert Y^\prime\vert =t$ and $c(i,j) =\alpha$ for all $(i,j)\in X^\prime\times Y^\prime$. 

Additionally, we need to introduce the notion of oriented cliques and oriented $K_{n,n}$s since their proof works with directed graphs. Therefore, let $DK_{n,n}$ denote an oriented $K_{n,n}$ such that all arcs point from one part to the other part and $TK_m$ be the transitive tournament on $m$ vertices, i.e. an oriented clique $K_m$ such that we can order the vertices $v_1, v_2, \dots ,v_m$ in a way that every edge $(v_i,v_j)$, $i\neq j$, points to the vertex of higher index.

\begin{thm}[Kierstead, Rödl \cite{Ki96}]\label{t2cr}
For all $m,n\in\N$ and oriented trees $T$, there exists $f=f(m,n,T)$ such that for every oriented graph $G$ the following statement holds. If $\chi (G)\geq f(m,n,T)$, then $G$ induces either $TK_m$, $DK_{n,n}$ or $T$.
\end{thm}

\begin{prf}
Let $G$ be an oriented graph. If $\omega (G)\geq m^\prime$, where $m^\prime = R(m,m)$, then $G$ contains $TK_m$. To see this, consider a clique of size $m^\prime$. Order the vertices $v_1,v_2,\dots v_{m^\prime}$ and choose a coloring $c$ with $c((v_i,v_j))=1$ if $i<j$ and $c(v_iv_j)=2$ otherwise. By definition of the Ramsey number, we find a monochromatic clique of size $m$, i.e. a transitive tournament.

Second, note that if $G$ does not contain a clique of size $m^\prime$ but a $K_{n^\prime , n^\prime}$, where $n^\prime = R(m^\prime , B(n,n,2))$, $G$ induces a $DK_{n,n}$. That is because we find stable sets of size $B(n,n,2)$ in both parts of $K_{n^\prime , n^\prime}$ again by th definition of the Ramsey number. After coloring all edges that point to one part in one color, we obtain an induced $DK_{n,n}$ by definition of $B$. 

Thus, it suffices to show, that $(\ast)$ there exists a function $g=g(m^\prime ,n^\prime ,T)$ such that every oriented graph $G$ with $\chi (G)\geq g(m^\prime ,n^\prime ,T)$ either has $\omega (G)\geq m^\prime$ or $G$ contains $K_{n^\prime ,n^\prime }$ or $G$ induces $T$. We prove this claim by induction on the number of vertices $v(T)$ of a tree $T$. 

The base case for $v(T)=1$ is trivial since a tree on one vertex is induced in every graph that has at least one vertex. 

Now suppose that $g(m^\prime ,n^\prime ,T^\prime )$ exists for all $m^\prime ,n^\prime \in\N$ and for all trees $T^\prime$ with $v(T^\prime )\leq s$. Let $T$ be a tree on $s+1$ vertices. We define a function $g(m^\prime ,n^\prime ,T)$ for all $m^\prime ,n^\prime\in\N$ so that $(\ast)$ holds as follows. Choose a leaf $l$ of $T$, which is adjacent to a vertex $u$ of $T$, and let $T^\prime = T- \{l\}$. Let $\sigma =$ \textit{in} if the edge between $l$ and $u$ points to $u$ and $\sigma = $ \textit{out} otherwise. Define $g(m^\prime ,n^\prime ,T) = 2(g(m^\prime ,n^\prime ,T^\prime )(s-1)+s)B$, where $B=B(n^\prime ,n^\prime ,s)$. We now show that $g(m^\prime ,n^\prime ,T)$ satisfies $(\ast)$. Let $G=(V,E)$ be a graph such that $\chi (G)\geq g(m^\prime ,n^\prime ,T)$, and suppose $G$ neither contains $K_{m^\prime}$ nor $K_{n^\prime ,n^\prime}$. We claim that $G$ induces $T$.

Let $W=\{v\in V:\delta^\sigma <sB\}$. Then, by lemma \ref{l2cr}, $\chi (G[W]) < 2sB$. Let $G^\prime = G[V^\prime ]$, where $V^\prime = V-W$. Then $\chi (G^\prime )\geq 2g(m^\prime ,n^\prime ,T^\prime )(s-1)B + 1$. Next, construct a rooted $s$-uniform hypergraph $H=(V^\prime ,F,r)$. An $s$-subset $S$ of $V^\prime$ is an edge of $H$ if and only if $T^\prime\approx G[S]$. Let $\phi$ be an isomorphism from $T^\prime$ to $G[S]$ and let $r_S$ be the image of $u$ under $\phi$. Note that in a proper coloring $c$ of $H$ no color class induces $T$. Thus, we can apply the induction hypothesis on every color class and get $\chi (G^\prime )\leq \chi (H)g(m^\prime ,n^\prime ,T^\prime )$. Therefore, $2(s-1)B +1\leq \chi (H)$.

By lemma \ref{l3cr} $b(H)\geq B$ holds. Then, there exists a vertex $r\in V^\prime$ and a set of edges $Y = \{S_1,S_2,\dots ,S_B\}$ with $r(S_i)=r$ for all $i\in [B]$ and $S_i\cap S_j=\{r\}$ for all $1\leq i<j\leq B$. Let $S_i=\{r,y_i^1,\dots ,y_i^{s-1}\}$. We can find a set $X\subseteq N^\sigma (r)$ with $\vert X\vert =B$ and $X\cap S_i =\emptyset$ for all $i\in [B]$ because $\vert N^\sigma (r)\vert\geq sB$. Define the function $C:X\times Y\to [s]$ by setting $c(x,S_i)=s$ if $x$ is not adjacent to any vertex in $S_i-\{r\}$ and $c(x,S_i)=j$ otherwise, where $j$ is the least index such that $x$ is adjacent to $y_i^j$.

By the definition of $B$, we find sets $X^\prime\subseteq X$ and $Y^\prime\subseteq Y$ with $\vert X^\prime\vert = \vert Y^\prime\vert =n^\prime$ such that $c(x,y)=\alpha$ for all $(x,y)\in X^\prime\times Y^\prime$ and a color $\alpha\in [s]$. If $\alpha\neq s$, then $X^\prime\cup\{y_i^\alpha :S_i\in Y^\prime\}$ contains $K_{n^\prime ,n^\prime}$, which contradicts the choice of $G$. Thus, $\alpha = s$. Let $x\in X^\prime$. Then, $x$ is not adjacent to any vertex in $S_i - \{r\}$ for any $S_i\in Y^\prime$ and $x$ is a $\sigma$-neighbor of $r$. Since $T^\prime\approx G[S_i]$ by an isomorphism mapping $u$ to $r$, $G[S_i\cup\{r\}]\approx T$.\qed
\end{prf}

\begin{note}
It is a direct implication of this theorem that the family $\textit{Forb}(T,K_{n,n})$ is $\chi$-bounded. This can be easily seen by replacing the directed graph $G$ by the corresponding undirected graph $G^\prime$.
\end{note}

I want to finish this chapter with a motivation for considering an oriented version of the Gyárfás-Sumner conjecture even if that is not part of this thesis. Before looking at specific cases note that we are only considering trees for which conjecture \ref{c1cr} is true. There actually exist some results for oriented stars and oriented $P_4$. Aboulker et al. \cite{Ab16} conjectured that every oriented star is $\chi$-bounding and proved the cases for $S_{k,0}$, $S_{0,l}$ and $S_{1,1}$ where $S_{k,l}$ denotes an oriented star with $k$ outgoing edges and $l$ edges pointing to the center vertex. 

Next, observe that all paths on less vertices are stars and therefore included in the first case. Let $or(P_4)$ denote the set of all pairwise non-isomorphic $P_4s$. Then, $or(P_4)$ contains only four elements:

For oriented $P_4$ Aboulker et al. \cite{Ab16} conjectured that only $P^+(2,1)$ and $P^-(2,1)$ are $\chi$-bounding. Chudnosky, Scott and Seymour \cite{CSS17b} extended these results and showed that both conjectures are indeed true but there is not much more known. 
\newpage
\section{Induced trees in Mycielskian graphs}
Finding a counterexample for the Gyárfás-Sumner conjecture implies constructing a family $\mathcal{F}$ of graphs such that $\mathcal{F}\subseteq\textit{Forb}(T)$ for some tree $T$ but we can not find a bounding function for $\mathcal{F}$. Mycielski's construction of triangle-free graphs with high chromatic number \cite{My55} may provide such a family. The goal of this chapter is to eliminate families of graphs obtained by the Mycielskian construction as candidates for disproving the Gyárfás-Sumner conjecture by showing that in each of these families we find infinitly many graphs containing an induced copy of $T$ for all trees $T$.

\begin{defn}
Let $G=(V,E)$ be a graph. Its \textit{Mycielskian graph} $\mu(G)$ consists of an induced copy $G$, a set $U$ of copies $u_i$ for every vertex $v_i\in V$ and one further vertex $w$. We call $w$ the center vertex of $\mu (G)$. In addition to the edges of the induced copy of $G$, the edge set of $\mu(G)$ includes two edges $\lbrace v_i,u_j\rbrace$ and $\lbrace u_i,v_j\rbrace$ for every edge $\lbrace v_i,v_j\rbrace$ in $E$ and edges $\lbrace u,w\rbrace$ for every vertex $u\in U$. 
\end{defn}
Mycielski's construction provides a sequence of graphs $G_0,G_1, G_2,\dots $ starting with $G_0=K_2$, where $G_i = \mu (G_{i-1})$. Let us denote the family of graphs in this sequence as $\mathcal{M}(K_2)$. Then, $\omega (G_i) = 2$ for all $i>0$, and for the chromatic number holds $\chi (G_i)=\chi (G_{i-1}) +1$, $i\in\N$. Also for $i\in\N$, let $G_i=(V_i,E_i)$. Denote the set of copies in this graph as $U_i$ and the one additional vertex as $w_i$. First, observe the following fact.
\begin{note}\label{o1my}
If a graph $G_i$ contains an induced copy of a graph $H$, then all graphs of the sequence with a higher index contain an induced copy of $H$ as well.
\end{note}

Second, observe that it does not matter which graph $G$ we choose as the first in the sequence of the construction. The proofs depend not on the structure of $G$ but rather on the vertices we add in each step of the construction and the graphs in $\mathcal{M}(G)$ still have a bounded clique number as the construction does not even create triangles. For our purpose, consider the sequence of graphs $\mathcal{M}(K_2)$ obtained by the mycielskian construction that starts with $G = K_2$.

Third, note that this construction indeed provides families of graphs worthy to look at for disproving the conjecture. Obviously, every family $\mathcal{M}(G)$ contains graphs of arbitrarily large chromatic number and they all have the same clique number. Furthermore, it is easy to see that $\mathcal{M}(G)$ is not trivially $\chi$-bounded by theorem \ref{t2cr} since for each $n\in\N$ there is a graph $G_i$ in $\mathcal{M}(G)$ (and therefore all graphs of higher index in the sequence as well) with $K_{n,n}\subseteq_I G_i$. Consider the graph $G_i$ where the number of copies $u\in U$ is at least $n$. Together with the center vertex $w$ in $G_i$ and the copies of $w$ in the next $n-1$ graphs of the sequence, they contain an induced $K_{n,n}$, i.e. $\mathcal{M}(G)\nsubseteq\textit{Forb}(T,K_{n,n})$ for all trees $T$. Thus, if we would find a tree $T$ with $\mathcal{M}(G)\subseteq \textit{Forb}(T)$, we could not find a bound for the chromatic number of graphs in $\textit{Forb}(T)$.

We start with two proofs by induction proving the existence of induced copies of paths and caterpillars in graphs of $\mathcal{M}(K_2)$. For $i\in\N$ denote the copy of the center vertex $w_i$ in $G_{i+1}$ as $w_i^\prime$. Let $Q_n =(w_1^\prime ,w_2^\prime ,\dots ,w_n^\prime)$ be the path through the copies of the first $n$ center vertices. 
\begin{thm}\label{t1my}
The path $Q_n$ is an induced subgraph of $G_{n+1}$ for all $n\in\N$, $n>1$.
\end{thm}
\begin{prf}
We prove this theorem by induction on $n$. To launch the induction let $n=2$. In the graph $G_2$, $w_2$ and $w^\prime_1$ are adjacent because $w^\prime_1\in U_2$. Thus, $G_3$ includes an edge between $w^\prime_1$ and $w^\prime_2$. Hence, as a path with just two vertices, $Q_2$ is an induced subgraph of $G_3$.

Now let $n>2$ and assume $Q_{n^\prime}\subseteq_I G_{n^\prime +1}$ for all $2\leq n^\prime <n$. By induction hypothesis, $Q_{n-1}$ is an induced subgraph of $G_n$. Recall the observation \ref{o1my}. The path $Q_{n-1}$ is also an induced subgraph of $G_{n+1}$. 
Now consider $w_n$. It is adjacent to any vertex in $U_n$ - including $w^\prime_{n-1}$ - but no other vertex, in particular no $w^\prime_i$, $i\in [n-2]$. Thus, as the copy of $w_n$ in $G_{n+1}$, $w^\prime_n$ is only adjacent to $w^\prime_{n-1}$ and no other vertex of $Q_{n-1}$. Thus, $Q_n$ is an induced subgraph of $G_{n+1}$.\qed
\end{prf}

\begin{cor}\label{c1my}
The graph $G_{n+1}$ contains an induced copy of $P_n$ for all $n\in\N$, $n>1$.
\end{cor}
\begin{prf}
The path $Q_n$ is a copy of $P_n$ and $Q_n\subseteq_I G_{n+1}$ by theorem \ref{t1my}.\qed
\end{prf}

In the next proof we will use certain subgraphs of $Q_n$ to construct an induced copy of a caterpillar in some $G\in\mathcal{M}(K_2)$. Therefore, let $Q_{i,j}=(w_i^\prime ,\dots , w_j^\prime )$, $1\leq i < j\leq n$, denote the path between $w_i^\prime$ and $w_j^\prime$ coinciding with the vertices of $Q_n$. In the special case where $i=j$, let $V(Q_{i,i})$ just denote the set of one vertex $\lbrace w_i^\prime \rbrace$.

\begin{thm}
Let $T$ be a caterpillar tree. There is an $G\in\mathcal{M}(K_2)$ containing a copy of $T$ as an induced subgraph.
\end{thm}
\begin{prf}
Let $T$ be a caterpillar tree with maximum degree $d = \Delta (T)$. Let $P=(p_1,p_2,...,p_n)$ be a longest path in $T$. Then, $P$ is a path on $n$ vertices. If $d\leq 2$, $T$ is a path and therefore $T=P$. Corollary \ref{c1my} then states, that it is an induced subgraph of $G_{n+1}$. 
If not, we have $d>2$. Every vertex in $P$ has not more than $d$ leaves. Now, let $G_i$ be a graph with $|V(G_i)| \geq d$ and consider $G_{i+n+1}$. By theorem \ref{t1my}, the path $Q_{i+n}=(w^\prime_1,w^\prime_2,...,w^\prime_{i+n})\subseteq_I G_{i+n+1}$. Thus, $Q_{i+1,i+n}$ is still an induced path and a path on $n$ vertices. Let $N_1:=U_{i+1}$ and for $1<l\leq n$ let $N_l\subseteq U_{i+l}$ denote the copies of the vertices in $N_{l-1}$. Observe, that $\vert N_l \vert \geq d$ for all $l\in [n]$. We then claim that the vertices in $S_k:=V(Q_{i+1,i+k})\cup\bigcup_{l=1}^k N_l$ are an induced tree in $G_{i+n+1}$. We prove this by induction on $k$.

For the base case let $k=1$. The vertices in $N_1=U_{i+1}$ form a stable set  and are adjacent to $w_{i+1}$ by definition of $U_{i+1}$. Therefore they are adjacent to $w^\prime_{i+1}$ as well. Thus, $S_1$ forms an induced copy of a star $K_{1,\vert N_1\vert}$, i.e. $G_{i+n}[S_1]$ is an induced tree.

Now let $1<k\leq n$ and suppose $S_{k^\prime}$ is an induced tree for any $k^\prime\in\N$, $k^\prime <k$. By the inductive hypothesis, $S_{k-1}$ is an induced tree, i.e. from the vertices in $S_{k-1}$ the vertices in $N_{k-1}$ are only adjacent to $w_{i+k-1}^\prime$. Observe that $(S_{k-1}\setminus \lbrace w_{i+k-1}^\prime\rbrace )\subseteq V_{i+k-1}$. Consider $N_k\subseteq U_{i+k}$. By the definition of vertices in $U_{i+k}$, vertices in $N_k$ do not have an edge to any vertex from $S_{k-1}\setminus \lbrace w_{i+k-1}^\prime\rbrace$ since the vertices in $N_{k-1}$ do not have them. Furthermore, $N_k\cup\lbrace w_{i+k-1}^\prime\rbrace \subseteq U_{i+k}$ is a stable set by definition but each vertex in this set has an edge to $w_{i+k}$ and therefore $w_{i+k}^\prime$. What is left to show, is that $w_{i+k}^\prime$ is not adjacent to any vertex in $S_{k-1}\setminus \lbrace w_{i+k-1}^\prime\rbrace$. That is easy to see, since $(S_{k-1}\setminus \lbrace w_{i+k-1}^\prime\rbrace )\cap U_{i+k}=\emptyset$. Thus, $S_k$ is an induced tree in $G_{i+k+1}$.

By mapping $p_j$ to the vertices of $w^\prime_i+j$ and the leaves of every vertex $p_j$ to a vertex from $N_j$, $j\in [n]$, we obtain an induced copy of $T$ in $G_{i+n+1}$.\qed
\end{prf}

After stating the proofs for some special cases of trees, there is actually a slightly different approach that proves that for all trees $T$ we can find a graph $G\in\mathcal{M}(K_2)$ containing an induced copy of $T$. Therefore, we need to introduce the concept of a rooted tree first. We can \textit{root} a tree $T$ by choosing a vertex $r\in T$ that we call the \textit{root} of $T$ and inductively setting the \textit{parent} of all neighbors of a vertex $v$ except its parent to be $v$ and let $C(v) =N(v)\setminus \lbrace p(v)\rbrace$ denote the set of \textit{children} of $v$.
Second, consider a graph $G_i\in\mathcal{M}(K_2)$. If $u\in U_i$ holds for a vertex $u$, call $u$ a copy in level $i$.
\begin{thm}
Let $T$ be tree. Then, there exists a $G\in\mathcal{M}(K_2)$ containing an induced copy of $T$.
\end{thm}
\begin{prf}
Let $n$ be the number of vertices in $T$ and choose a vertex $r$ as the root of $T$. Denote the vertices as $r=v_1$, $v_2$, ..., $v_n$ in an order we would obtain by performing an in-order tree traversal. I.e. starting with the root vertex $r$ for $i\in [n-1]$ we choose $v_{i+1}$ as any vertex in $C(v_i)$. If there is no child left that is not assigned yet, consider the previous vertices in decreasing order and choose a child of them if possible. Observe that each vertex except $v_1$ is a child of a vertex with a smaller index.

We shall construct an induced copy of $T$ with vertices $c_1,\dots ,c_n$. 
In preparation of the construction, note that each vertex $c_i$ will be a copy in some level $j\in\N$, i.e. $c_i\in U_j$. We refer to this level as $l_i$. Recall that then we denote the center vertex of this level by $w_{l_i}$.

We construct the induced copy of $T$ by choosing the root vertex $v_1$ as a copy in some level $m\in\N$ and adding the other vertices in their order in the following way. Each vertex $v_i$, $i>1$, has a parent $v_k$ with $k<i$. Let vertex $c_i$ be the copy of $w_{l_k}$ in a level strictly higher than the level  $l_{i-1}$. Since $w_{l_k}$ and $c_k$ are adjacent, $c_i$ and $c_k$ are as well. Thus, for every edge in $T$, there is an edge in the copy of $T$. Observe that all vertices $c_1,\dots , c_n$ are copies in different levels. 

We need to confirm that this copy is indeed induced. We prove this by induction on $i$, claiming that the vertices in $\lbrace c_1, c_2, \dots , v_i\rbrace$ form an induced tree in $G_{l_i}$. To launch the induction, let $i=1$. One vertex by itself induces no edge. Thus, it trivially induces a tree in $G_m$.

Let $i\in [n]$, $i>1$, and suppose the statement holds for any $1\leq i^\prime < i$. Then, by the inductive hypothesis, the vertices in $\lbrace c_1, c_2, \dots , c_{i-1}\rbrace$ induce a tree in the level $l_{i-1}$ and therefore also in the level $l_i$ since $l_i >l_{i-1}$ and $G_{l_i}$ then contains an induced copy of $G_{l_{i-1}}$ by observation \ref{o1my}. The vertex $c_i$ is a copy of some $w_{l_k}$, $k<i$. The vertex $w_{l_k}$ is only adjacent to vertices $U_{l_k}$, i.e. it is not adjacent to any vertex in $\lbrace c_1, \dots , c_{k-1}$. Therefore, $c_i$ is also not adjacent to them. Since center vertices of different levels are never adjacent by definition, no vertex in $c_j\in\lbrace c_{k+1}, \dots ,c_{i^\prime}\rbrace$ chosen after $c_k$ can, as a copy of $w_{l_j}$, be adjacent to $w_{l_k}$. Therefore, $c_i$ is only adjacent to $c_k$ as wanted and the vertices $\lbrace c_1, c_2, \dots , v_i\rbrace$ induce a tree in $G_{l_k}$.

Thus, we constructed a copy of $T$ in $G_{l_n}$.\qed
\end{prf}

  \newpage
\section{Intersection Graph of Line Segments}
Besides the Mycielskian construction we also obtain triangle-free graphs of arbitrarily large chromatic number from the following construction by Pawlik et al. \cite{Paw14} using intersection graphs of line segments in an axis-aligned rectangle in the $\mathbb{R}^2$. As mentioned in a paper by Felsner et al. \cite{Fe18} the construction of Burling graphs results in exactly the same family of graphs. They state that the construction of these graphs actually traces back to the PhD Thesis by Burling himself \cite{Bu65}. Since the result is the same regardless of the construction, it is enough to consider one to exclude all constructions as candidates for disproving the Gyárfás-Sumner conjecture. We will look at both constructions and then prove that for all trees $T$ the family of graphs obtained by these constructions indeed contains infinitely many graphs that induce $T$.

\begin{defn}
Let $R$ be an axis-aligned rectangle in the $\mathbb{R}^2$ with boundaries $[a,b]\times [c,d]$ and $\mathcal{L}$ a family of line segments contained in the interior of $R$. A probe $P$ in $R$ is a rectangle $[a^\prime ,b]\times [c^\prime ,d^\prime]$ with the following properties:
\begin{enumerate}[(i)]
\item $a<a^\prime < b$ and $c<c^\prime <d^\prime < d$
\item no line segment in $\mathcal{L}$ intersects the left boundary of $P$
\item no line segment in $\mathcal{L}$ ends in $P$ or on the boundary on $P$
\item line segments in $\mathcal{L}$ intersecting $P$ are pairwise disjoint
\end{enumerate}
We call the rectangle $[a^\prime ,b^\prime ]\times [c^\prime ,d^\prime]$ with the maximal $b^\prime$, such that no line segment intersects it, the root of $P$. Figure \ref{f1ls} contains an example for a rectangle $R$ with a family of line segments and two probes.
\end{defn}

\begin{figure}[ht]
\begin{center}
\includegraphics[scale=1]{probe}
\end{center}
\caption{Illustration of line segments, probes and roots in a rectangle R}
\label{f1ls}
\end{figure}


Let $R$ be an axis-aligned rectangle with a non-empty interior. Let $(s_i)_{i\in\N}$ and $(p_i)_{i\in\N}$ be sequences inductively defined by setting $s_1=p_1=1$, $s_{i+1}=(p_i+1)s_i+p_i^2$ and $p_{i+1}=2p_i^2$. For each $k\in\N$, we then define a family $\mathcal{L}_k$ of $s_k$ line segments and a set $\mathcal{P}_k$ of $p_k$ probes in $R$ as follows. Let $\mathcal{L}_1$ contain an arbitrary non-horizontal line segment in $R$ and choose any rectangle in $R$ that shares its right boundary with $R$ and intersects the line segment with its lower and upper bound as the probe in $\mathcal{P}_1$. Now suppose $k\geq 2$ for the inductive step. To construct $\mathcal{L}_k$ draw $\mathcal{L}_{k-1}$ in $R$ and for each probe $p\in\mathcal{P}_{k-1}$ place another copy $\mathcal{L}_p$ of $\mathcal{L}_{k-1}$ with probes $\mathcal{P}_{p}$ in the root of $p$. Note that the probes in $\mathcal{P}_{p}$ are not extending to the right boundary of $R$ but instead end at the right boundary of the root of $p$. Afterwards, for each $p\in\mathcal{P}_{k-1}$ and every $q\in\mathcal{P}_{p}$ draw the diagonal $d_q$ of $q$, i.e. a line segment from its bottom-left corner to its upper-right corner. The diagonal $d_q$ crosses all segments intersecting $q$ but no other. Then, $\mathcal{L}_k$ contains the line segments from the $p_{k-1} + 1$ copies of $\mathcal{L}_{k-1}$ and the $p_{k-1}^2$ diagonals, i.e. $\vert \mathcal{L}_k\vert =(p_{k-1} + 1)s_{k-1}+p_{k-1}^2=s_k$. 

Now we want to construct $\mathcal{P}_k$. For each $p\in\mathcal{P}_{k-1}$ and every $q\in\mathcal{P}_{p}$ let $\mathcal{S}(p)$ be the set of line segments in $\mathcal{L}_{k-1}$ intersecting $p$ and $\mathcal{S}_p(q)$ the segments in $\mathcal{L}_k$ intersecting $q$. For each such $q$ we add two probes to $\mathcal{P}_k$. We place the first one, the upper probe $u_q$, close to the top of $q$, such that the diagonal $d_q$ but no other segment in $\mathcal{S}_p(q)$ intersects it and choose the second probe, the lower probe $l_q$, close to the bottom of $q$ such that it contains all segments in $\mathcal{S}_p(q)$ but not $d_q$. Then, both probes end at the right boundary of $R$. Since line segments intersecting $p$ are pairwise disjoint by the induction hypothesis, and since we placed the line segments in $\mathcal{S}_p$ in a way that they do not cross any segment in $p$ as well, $\mathcal{S}(p)\cup \lbrace d_q\rbrace$ and $\mathcal{S}(p)\cup\mathcal{L}_p(q)$ are both independent sets, i.e. both $u_q$ and $l_q$ are proper probes. Finally, observe that $\vert \mathcal{P}_k\vert = 2p_{k-1}^2=p_k$.

\begin{figure}[ht]
\begin{center}
\includegraphics[scale=0.75]{diagonal}
\end{center}
\caption{Probe $q$ with its diagonal $d_q$ and the two newly created probes $u_q$ and $l_q$}
\label{f2ls}
\end{figure}


Let $G_k$ denote the intersection graph of the line segments in $\mathcal{L}_k$, i.e. each vertex in $G_k$ represents a line segment of $\mathcal{L}_k$ and two vertices are adjacent if and only if the corresponding line segments intersect each other.\\

As mentioned before, the construction by Burling as presented in \cite{Fe18} results in the same family of graphs. For each $k\in\N$, construct the Burling graph $B_k$ and a corresponding set of stable sets $S(B_k)$ as follows. Let $B_1$ be the graph on a single vertex and let $S(B_1)$ contain the only stable set on one vertex in $B_1$. For $k\geq 2$, consider a copy $H$ of $B_{k-1}$ which we think of as the original copy and one further copy $H_S$ of $B_{k-1}$ for each stable set $S\in S(B_{k-1})$. Let $S(H_S)$ denote the copy of $S(H)$ in $H_S$. Furthermore, for each copy $H_S$ and stable set $X\in S(H_S)$ add another vertex $v_{S,X}$ adjacent to all vertices in $X$ but no other vertex. Denote the graph obtained from a copy $H_S$ and added new vertices for each stable set $X\in S(H_S)$ as $H_S^{\prime}$. Then, define the graph $B_k$ as $B_k=H\cup\bigcup_{S\in S(B_k)} H_S^{\prime}$. Its set of stable sets $S(B_k)$ consists of two sets for each $S\in S(H)$ and $X\in S(H_S)$, namely $S\cup \lbrace v_{S,X}\rbrace$ and $S\cup X$. Note that both sets are stable in $B_k$. Refer to Figure \ref{f4ls} for an illustration of the first three Burling graphs.

\begin{figure}[ht]
\begin{center}
\includegraphics[scale=1]{burling}
\end{center}
\caption{The first three Burling graphs with added vertices in the copies colored red}
\label{f4ls}
\end{figure}

\begin{lemma}
The families of graphs obtained by the two constructions are the same, i.e. $\lbrace G_k :k\in\N\rbrace=\lbrace B_k :k\in\N\rbrace$. 
\end{lemma}
\begin{prf}
We shall show that $G_k$ and $B_k$ are isomorphic under an isomorphism $\varphi_k$ for all $k\in\N$ by induction on $k$. The idea is to define a bijection $f_k:\mathcal{P}_k\to S(B_k)$ between the probes in $\mathcal{P}_k$ and the stable sets in $S(B_k)$ in each step such that the images of the vertices corresponding to the line segments in a probe $p\in\mathcal{P}_k$ under $\varphi_k$ are exactly the vertices in $f_k(p)$. We then use these maps to define an isomorphism between $G_{k+1}$ and $B_{k+1}$.

For $k=1$, $G_1$ and $B_1$ are graphs on a single vertex and thus isomorphic with isomorphism $\varphi_1$. Define a bijection $f_1:\mathcal{P}_1\to S(B_1)$ that maps the one probe in $\mathcal{P}_k$ to the one stable set in $S(B_k)$.

Suppose $k\geq 2$, $G_{k-1}$ and $B_{k-1}$ are isomorphic, and the bijection $f_{k-1}$ between $\mathcal{P}_{k-1}$ and $S(B_{k-1})$ exists as wanted. We extend the isomorphism to pairs of copies $(\mathcal{L}_p, H_{f(p)})$ of $G_{k-1}$ and $B_{k-1}$ respectively for all probes $p\in \mathcal{P}_{k-1}$ and thus covering all copies of $G_{k-1}$ in $G_k$ and $B_{k-1}$ in $B_{k}$. We do this by setting $\varphi_k (v^\prime)=u^\prime$ if and only if $v^\prime\in \mathcal{L}_p$ is the copy of a vertex $v\in G_{k-1}$ and $u^\prime$ is the copy of $\varphi_{k-1}(v)\in H_{f(p)}$. Then, we map the remaining vertices according to the bijection between $\mathcal{P}_k$ and $S(B_k)$: for each $p\in\mathcal{P}_{k-1}$ and $q\in\mathcal{P}_p$ ($q$ is the copy of some probe $q^\prime\in\mathcal{P}_{k-1}$) map the vertex corresponding to $d_q\in V(G_{k-1})$ to the vertex $v_{S,X}\in B_{k-1}$ where $S=f(p)$ and $X$ is the copy of $f(q^\prime)$ in $S(H_S)$. That yields an isomorphism $\varphi_k$ between $G_k$ and $B_k$ since the images of the neighbors of $d_q$ are the neighbors of $v_{S,X}$ in $B_k$. Finally, define $f_k:\mathcal{P}_k\to S(B_k)$ as $f(u_q)= S\cup \lbrace v_{S,X}\rbrace$ and $f(l_q)= S\cup X$.\qed
\end{prf}

Since both constructions yield the same family of graphs, let us call it $\mathcal{LG}$, it is enough to consider just the Burling graphs for the following proofs. The results hold for both constructions. First, we want to assert that the graphs are indeed triangle-free and that we can find graphs of arbitrarily large chromatic number in $\mathcal{LG}$.


\begin{thm}[Felsner et al. \cite{Fe18}]
All graphs in $\mathcal{LG}=\lbrace B_k :k\in\N\rbrace$ are triangle-free.
\end{thm}

\begin{prf}
We prove this by induction on $k$. The graph $B_1$ is trivially triangle-free. For each $k\geq 2$, the graph $B_k$ is triangle-free since $B_{k-1}$ is triangle-free by induction hypothesis and the neighborhood of every vertex $v_{S,X}$ we introduce in $B_k$ is a stable set.\qed
\end{prf}

\begin{thm}[Felsner et al. \cite{Fe18}]\label{t2ls}
For all $k\in\N$, the graph $B_k$ has chromatic number $\chi (B_k)$ at least $k$.
\end{thm}

\begin{prf}
We show this by proving a stronger statement by induction on $k$. For any proper coloring $c$ of $B_k$, there exists a stable set $S\in S(B_k)$ such that $c$ uses at least $k$ colors on the vertices in $S$.

This is again trivial for $k=1$. Now, suppose $k\geq 2$ and $c$ is a proper coloring of $B_k$. By induction hypothesis, there exists a stable set $S\in S(B_{k-1})$ $c$ uses at least $k-1$ colors on. Also by induction hypothesis, there exists a stable set $X\in S(H_S)$ $c$ uses at least $k-1$ colors on. If $c$ uses at least $k$ colors on $(S\cup X)\in S(B_k)$, we are done. Otherwise, $c$ uses the same $k-1$ colors on $S$ and $X$. Since $v_{S,X}$ is adjacent to each vertex in $X$, it is colored in a color not used on a vertex in $S$ and thus $c$ uses at least $k$ colors on $S\cup\lbrace v_{S,X}\rbrace$.\qed 
\end{prf}

\begin{comment}
\begin{thm}[Pawlik et al. \cite{Paw14}]
For every $k\in\N$ exists a family $\mathcal{L}$ of line segments in the plane with no three pairwise intersecting segments and $\chi (G)\geq k$, where $G$ is the intersection graph of $\mathcal{L}$.
\end{thm}

\begin{prf}
 For every $k\in\N$ we now construct a family $\mathcal{L}_k$ of $s_k$ line segments, such that no three are pairwise intersecting, and a family $\mathcal{P}_k$ of $p_k$ pairwise disjoint probes in $R$, such that for every proper coloring $c$ of $\mathcal{L}_k$ there is a probe $P\in\mathcal{P}_k$ for which $c$ uses at least $k$ colors on the line segments of $\mathcal{L}_k$ intersecting $P$, by induction on $k$. Then, the corresponding intersection graph $G_k$ for $\mathcal{L}_k$ is triangle-free and $\chi (G)\geq k$.

For the base case $k=1$ consider an arbitrary non-horizontal line segment in $R$. Let $\mathcal{L}_1$ just contain this segment and choose any rectangle in $R$ that shares its right boundary with $R$ and intersects the line segment with its lower and upper bound as the probe $p$ in $\mathcal{P}_1$. $G_1$ is obviously triangle-free and $p$ uses exactly one color on the segment it contains.

For the induction step, consider a given rectangle $R$ and families $\mathcal{L}_k$ and $\mathcal{P}_k$ for a fixed $k\in\N$. To construct $\mathcal{L}_{k+1}$ draw $\mathcal{L}_k$ in $R$ and for each probe $p\in\mathcal{P}_k$ place another copy $\mathcal{L}_p$ of $\mathcal{L}_k$ with probes $\mathcal{P}_{p}$ in the root of $p$. Afterwards, for each $p\in\mathcal{P}_k$ and every $q\in\mathcal{P}_{p}$ draw the diagonal $d_q$ of $q$, i.e. a line segment from its bottom-left corner to its upper-right corner. $d_q$ crosses all segments intersecting $q$ but no other. Then, $\mathcal{L}_k$ consists of the segments from the $p_k + 1$ copies of $\mathcal{L}_k$ and the $p_k^2$ diagonals, i.e. $\vert \mathcal{L}_{k+1}\vert =(p_k + 1)s_k+p_k^2=s_{k+1}$. Since the copies of $\mathcal{L}_k$ are triangle-free and disjoint and $d_q$ intersects only the segments in $q$, an independent set of segments, $\mathcal{L}_{k+1}$ is also triangle-free.

Now we want to construct $\mathcal{P}_{k+1}$. For each $p\in\mathcal{P}_k$ and every $q\in\mathcal{P}_{p}$ let $\mathcal{L}(p)$ be the set of segments in $\mathcal{P}_k$ intersecting $p$ and $\mathcal{L}_p(q)$ the segments in $\mathcal{L}(p)$ intersecting $q$. We add two probes to $\mathcal{P}_{k+1}$. We place the first one, the upper probe $u_q$, close to the top of $q$, such that the diagonal $d_q$ but no other segment in $\mathcal{L}_p(q)$ intersects it and choose the second probe, the lower probe $l_q$, close to the bottom of $q$ such that it contains all segments in $\mathcal{L}_p(q)$ but not $d_q$. Then, both probes end at the right boundary of $R$. By the induction hypothesis and the way we placed $\mathcal{L}_p$, $\mathcal{L}(p)\cup \lbrace d_q\rbrace$ and $\mathcal{L}(p)\cup\mathcal{L}_p(q)$ are both independent sets, i.e. both $u_q$ and $l_q$ are proper probes. Finally, observe that $\vert \mathcal{P}_{k+1}\vert = 2p_k^2=p_{k+1}$.

Let $c$ be a coloring of $\mathcal{L}_{k+1}$. Consider the restriction of $c$ to the original copy of $\mathcal{L}_k$. By the induction hypothesis, there exists a probe $p\in\mathcal{P}_k$ for which $c$ needs $k$ colors to paint the line segments in $p$. Now, consider the copy $\mathcal{L}_p$ of $\mathcal{L}_k$ in the root of $p$. Again, by induction hypothesis, there is a probe $q\in\mathcal{P}_p$ that uses $k$ colors on the segments in $\mathcal{L}_p$ intersecting $q$. If the colors used by $c$ in $p$ and $q$ are different, at least $k+1$ colors are used on the segments pierced by the lower probe $l_q$. Otherwise, $d_q$ has a different color than the colors used in $p$ and $q$ and thus, $c$ uses $k+1$ colors on the segments pierced by the upper probe $u_q$.\qed
\end{prf}
\end{comment}
Similar to the construction of triangle-free graphs of large chromatic number by Mycielski, there exist rather simple proofs showing that such graphs provided by the construction above contain induced copies of stars and paths as well.
First, note that $B_{k+1}$ contains an induced copy of $B_k$ for all $k\in\N$. This follows directly from the construction. With this knowledge it is enough to prove the existence of an induced copy of a tree $T$ in some $B_k$. Then $T$ is no suitable candidate to disprove the conjecture as it is contained in a graph of arbitrarily large chromatic number.

\begin{thm}
Let $k\in\N$. Then, $B_{k+1}$ contains an induced copy of $K_{1,k}$.
\end{thm}
\begin{prf}
Let $k\in\N$ be fixed and consider $B_k$. By the proof of Theorem \ref{t2ls}, there exists a stable set $S\in S(B_k)$ such that any proper coloring $c$ of $B_k$ uses at least $k$ colors on the vertices in $S$. Then $S$ has at least size $k$. Consider the copy $X\in S(H_S)$ of $S$ in $H_S$. By definition, $X$ is a stable set and contains at least $k$ vertices. Thus, the vertex set $X\cup \lbrace v_{S,X}\rbrace$ induces a copy of $K_{1,k}$ in $B_k$.\qed   
\end{prf}

\begin{thm}\label{t3ls}
Let $k\in\N$. Then, $B_{k+1}$ contains an induced copy of $P_k$.
\end{thm}
\begin{prf}
We prove the existence of an induced copy of $P_k$ in $B_{k+1}$ by induction on $k$. The idea is to construct an induced path $Q_k$ on $k$ vertices by choosing a vertex $v_{S,X}$ for some stable sets $S\in S(B_k)$ and $X\in S(H_S)$ as the vertex we add to our path in step $k$.

Launch the induction with $k=1$. Consider $B_1$. We have just one stable set $S$ in $S(B_1)$, i.e. there exists just one vertex $v_{S,X}$ in $B_2$ where $X$ is the copy of $S$ in $S(H_S)$. Define $Q_1=(v_{S,X})$. This is an induced copy of $P_1$  in $B_2$.

For the induction step, let $k\geq 2$ and let $Q_{k-1}$ be an induced path on $k-1$ vertices in $B_k$. We construct an induced path $Q_k$ on $k$ vertices as follows. Let $v=v_{S,X}$ be the last vertex of $Q_{k-1}$. Then, $S^\prime = S\cup \lbrace v\rbrace$ is a stable set in $S(B_k)$. Observe that the copies $H$ and $H_S$ of $B_{k-1}$ are edge-disjoint in $B_k$. Thus, no vertex in $V(Q_{k-1})\setminus\lbrace v\rbrace$ is also an element of $S$. Now consider $B_{k+1}$ and choose an arbitrary stable set $R\in S(B_k)$. Let $Q_{k-1}^\prime$ denote the copy of $Q_{k-1}$ in $H_R$ and let $v^\prime$ denote the copy of $v$. Note that $Q_{k-1}^\prime$ is an induced copy of $P_{k-1}$ as well. Consider the copy of $S^\prime$ in $S(H_R)$, call it $S^{\prime\prime}$. As before no vertex in $V(Q_{k-1}^\prime )\setminus\lbrace v^\prime\rbrace$ is a vertex of $S^{\prime\prime}$. Thus, the vertex $v_{R,S^{\prime\prime}}$ is adjacent to the copy of $v^\prime$ but no other vertex in $Q_{k-1}^\prime$ and $Q_k=Q_{k-1}^\prime v_{R,S^{\prime\prime}}$ is an induced path on $k$ vertices in $B_{k+1}$. See Figure \ref{f3ls} for an illustration.\qed
\end{prf}

\begin{figure}[ht]
\begin{center}
\includegraphics[scale=1]{path_ls}
\end{center}
\caption{Choice of the vertex added in the inductive step}
\label{f3ls}
\end{figure}

Finally, there exists an inductive proof showing that actually for every tree $T$ we can find a Burling graph $B_k$ for some $k\in\N$ that contains an induced copy of $T$. Therefore, let us define a sequence of trees $T_1, T_2, \dots$ where $T_1$ is the tree on one vertex and for all $i\in\N$ let $T_{i+1}$ be obtained from $T_i$ by adding a new vertex only adjacent $t$ for each vertex $t\in T_i$. Thus, $T_i$ is atree on $2^{i-1}$ vertices. Note that the following proof works by the same principle as the proof of Theorem \ref{t3ls}.

\begin{thm}\label{t1ls}
Let $k\in\N$. The graph $B_k$ contains an induced copy of $T_k$.
\end{thm}

\begin{prf}
We shall prove a stronger statement. We claim that $T_k\subseteq_I B_k$ for all $k\in\N$ and that for each vertex $t\in V(T_k)$ there exists a stable set $S_t\in S(B_k)$ containing exactly $t$ but no other vertex from $V(T_k)$. The proof goes by induction on $k$. The idea of the inductive step then is to choose some copy $H_S$ of $B_k$ in $B_{k+1}$, consider an induced copy of $T_k$ in $H_S$ and add the vertices $v_{S,S_t}$ for each $t$ to form an induced copy of $T_{k+1}$.

To launch the induction let $k=1$. The tree $T_1$ consists of one single vertex $t_1$. Thus, the graph on the single vertex in $B_1$ is an induced copy of $T_1$ in $B_1$. Observe that the stable set in $S(B_1)$ contains this vertex.

Now let $k\geq 2$ and suppose the $B_{k-1}$ contains an induced copy of $T_{k-1}$, call it $Q_{k-1}$. Consider $B_k$ and choose a stable set $S\in S(B_{k-1})$. Then, $H_S$ is an induced copy of $B_{k-1}$ in $B_k$ and thus it contains an induced copy of $T_{k-1}$ as well. We denote it as $Q_{k-1}^\prime$. Let $q_1, q_2,\dots , q_{2^{k-2}}$ be the vertices of $Q_{k-1}^\prime$. By induction hypothesis, for each vertex $q$ in $Q_{k-1}$ there exists a stable set in $S(B_{k-1})$ containing $q$ but no other vertex from $Q_{k-1}$. Thus, there also exists a stable set $S_{q_i}\in S(H_S)$ containing $q_i$ but no other vertex of $Q_{k-1}^\prime$ for each $i\in [2^{k-2}]$. Now choose $v_i=v_{S,S_{q_i}}$ as the new vertex. Observe that it is only adjacent to $q_i$ and no other vertex of $Q_{k-1}^\prime$. Since the vertices $v_i$, $i\in [2^{k-2}]$, are pairwise non-adjacent by definition in the construction of Burling graphs, the graph $Q_k$ induced by the vertex set $V(Q_{k-1}^\prime)\cup \lbrace v_i:i\in [2^{k-2}]\rbrace$ is a tree and by the choice of the new vertices isomorphic to $T_k$. Furthermore, with the stable sets $S\cup \lbrace v_i\rbrace$ and $S\cup S_{q_i}$ in $S(B_k)$ we have one, $S\cup \lbrace v_i\rbrace$, that contains only $v_i$ and one, $S\cup S_{q_i}$, that only contains $q_i$ of the vertices in $Q_k$.\qed
\end{prf}

\begin{cor}
For every tree $T$, there is a $k\in\N$ such that $T\subseteq_I B_k$.
\end{cor}
\begin{prf}
Let $T$ be a tree. Denote its radius by $r$ and let $v\in V(T)$ be a vertex such that the distance between $v$ and a vertex $u\in V(T)\setminus \lbrace v\rbrace$ is at most $r$. Define $k=\max_{u\in V(T)\setminus \lbrace v\rbrace} (dist(v,u)+\delta (u))$. Then $T$ is an induced subgraph of $T_k$ and therefore $T\subseteq_I B_k$ holds by Theorem \ref{t1ls}.\qed
\end{prf}


Considering the construction of Burling graphs it is striking that the graphs grow large very fast because when constructing $B_k$ for some $k\in\N$, $k>1$, we add a lot of copies isomorphic to $B_{k-1}$. In addition, the size of the set of stable sets increases vastly as well. It may be possible to modify the construction while preserving its properties, namely that each graph is triangle-free and we construct graphs of arbitrarily large chromatic number, and obtain graphs not inducing certain trees. The first idea is to simplify the construction by adding just one copy of $B_{k-1}$ when constructing $B_k$.

For $k\in\N$ define $B^\prime_k$ and $S(B^\prime_k)$ as follows. Let $B_1^\prime =B_1$, i.e. a graph on a single vertex and let $S(B_1^\prime )$ contain the stable set on this single vertex.

Now construct $B^\prime_k$ for $k\geq 2$. Let $H$ denote a copy of $B_{k-1}^\prime$ and $S(H)$ a copy of $S(B_{k-1}^\prime )$ for $H$. Take another copy $H^\prime$ of $B_{k-1}^\prime$ with stable sets $S(H^\prime )$. Introduce vertices $v_X$ for every stable set $X\in S(H^\prime )$ and add an edge between $v_X$ and each vertex in $X$ such that $X\cup \{v_X\}$ forms a star. Denote the graph obtained by the vertex introductions as $H^{\prime\prime}$ and define $B_k^\prime =H\cup H^{\prime\prime}$. For every $S\in S(H)$ and $X\in S(H^\prime)$ let $S(B_k^\prime )$ contain two stable sets $S\cup\{v_X\}$ and $S\cup X$.

\begin{thm}
For all $k\in\N$, the graph $B_k^\prime$ is triangle-free.
\end{thm}
\begin{prf}
This is trivial for $k=1$ and for $k\geq 2$ the neighborhood of the newly introduced vertices form stable sets. Thus, $B_k^\prime$ is triangle-free.\qed
\end{prf}

\begin{thm}
For all $k\in\N$, the graph $B_k^\prime$ has chromatic number $\chi (B_k^\prime)\geq k$.
\end{thm}
\begin{prf}
As before, we prove a stronger statement: For all $k\in\N$, there exists a stable set $S\in S(B_k^\prime )$ a proper coloring of $B_k^\prime$ uses at least $k$ colors on.

For $k=1$ this is trivial. For $k\geq 2$, let $c$ be a proper coloring of $B_k^\prime$. By induction hypothesis there exists a stable set $S\in S(H)$ and a stable set $X\in S(H^\prime )$ such that $c$ uses at least $k-1$ colors on each of the two sets. Observe that $S\cap X=\emptyset$. Then, $c$ uses either at least $k$ colors on the stable set $S\cup X\in S(B_k^\prime)$ or exactly the same $k-1$ colors on both. In this case $v_X$ is colored in a color not used in $X$ and $S\cup \{v_X\}\in S(B_k^\prime)$ needs at least $k$ colors.\qed
\end{prf}

Thus, the properties of the original Burling graphs are maintained. Unfortunately, that is also true for the property that $B_k^\prime$ contains an induced subgraph isomorphic to a tree $T_k$.

\begin{thm}
Let $k\in\N$. The graph $B_k^\prime$ contains an induced subgraph isomorphic to $T_k$.
\end{thm}
\begin{prf}
Similarly to the proof for the Burling graphs we prove a stronger statement by induction on $k$: For all $k\in\N$ there exists a graph $T$ isomorphic to $T_k$ with $T\subseteq_I B_k^\prime$ and for each vertex $t\in T$ there exists a stable set $S(B_k^\prime )$ containing $t$ but no other vertex from $V(T)$.

For $k=1$, the graphs $B_1^\prime$ and $T_1$ are both graphs on a single vertex and thus isomorphic. Furthermore, the stable set in $S(B_1^\prime)$ contains this vertex.

Now let $k\geq 2$ and consider $B_k^\prime$. The subgraph $H^\prime$ is a copy of $B_{k-1}^\prime$ and contains a subgraph $Q$ isomorphic to $T_{k-1}$ by induction hypothesis.  Let $q_1, q_2,\dots , q_{2^{k-2}}$ denote the vertices of $Q$. By induction hypothesis, for each vertex $q_i\in V(Q)$, $i\in [2^{k-2}]$, there exists a stable set $X\in S(H^\prime)$ containing $q_i$ but no other vertex from $Q$. We choose $v_i=v_X$ as the new vertex. It is only adjacent to $q_i$ but no other vertex of $Q$. Since the vertices $v_i$, $i\in [2^{k-2}]$ form a stable set by the construction, the graph $T$ induced by the vertices in $V(Q)\cup \{v_i:i\in [2^{k-2}]\}$ is a tree and furthermore isomorphic to $T_k$ since for each vertex $q\in Q$ we found exactly one new vertex only adjacent to $q$ as in the definition of $T_k$. Choose a stable set $S\in S(H)$. With the stable sets $S\cup \{v_i\}$ for $v_i$ and $S\cup X$ for $q_i$ we also have a stable sets containing exactly one vertex $t$ of $T$ for all $t\in V(T)$.\qed
\end{prf}

The second idea is to reduce the number of stable sets in $S(B_k)$ for a Burling graph $B_k$. 
\newpage



\listoffigures
\newpage
  % Literaturverzeichnis (beginnt auf einer ungeraden Seite)
%\begin{thebibliography}{Lam00}
%\end{thebibliography}
 
\bibliography{/Users/Frithjof/Documents/studium/ba/arbeit/bib/refs.bib}
\bibliographystyle{ieeetr}
      
  % ggf. hier Tabelle mit Symbolen 
  % (kann auch auf das Inhaltsverzeichnis folgen)

\newpage
  
 \thispagestyle{empty}


\vspace*{8cm}


\section*{Declaration}

%Hiermit versichere ich, dass ich diese Arbeit selbständig verfasst und keine anderen, %als die angegebenen Quellen und Hilfsmittel benutzt, die wörtlich oder inhaltlich %übernommenen Stellen als solche kenntlich gemacht und die Satzung des Karlsruher %Instituts für Technologie zur Sicherung guter wissenschaftlicher Praxis in der jeweils %gültigen Fassung beachtet habe. 
I hereby declare that I have written this thesis independently and did not use sources or means other than those specified in the text
%, identified literally or substantively adopted passages,
 and respected the statutes of the Karlruhe Institute of Technology for protection of good scientific practice in its valid version.
\\[2ex] 





\noindent
Karlsruhe, the Date \\[5ex]

% Unterschrift (handgeschrieben)



\end{document}

