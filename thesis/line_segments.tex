Besides the Mycielskian construction we also obtain triangle-free graphs of arbitrarily large chromatic number from the following construction by Pawlik et al. \cite{Paw14} using intersection graphs of line segments in an axis-aligned rectangle in the $\mathbb{R}^2$. As mentioned in a paper by Felsner et al. \cite{Fe18} the construction of Burling graphs results in exactly the same family of graphs. They state that the construction of these graphs actually traces back to the PhD Thesis by Burling himself \cite{Bu65}. Since the result is the same regardless of the construction, it is enough to consider one to exclude all constructions as candidates for disproving the Gyárfás-Sumner conjecture. We will look at both constructions and then prove that for all trees $T$ the family of graphs obtained by these constructions indeed contains infinitely many graphs that induce $T$.

\begin{defn}
Let $R$ be an axis-aligned rectangle in the $\mathbb{R}^2$ with boundaries $[a,b]\times [c,d]$ and $\mathcal{L}$ a family of line segments contained in the interior of $R$. A probe $P$ in $R$ is a rectangle $[a^\prime ,b]\times [c^\prime ,d^\prime]$ with the following properties:
\begin{enumerate}[(i)]
\item $a<a^\prime < b$ and $c<c^\prime <d^\prime < d$
\item no line segment in $\mathcal{L}$ intersects the left boundary of $P$
\item no line segment in $\mathcal{L}$ ends in $P$ or on the boundary on $P$
\item line segments in $\mathcal{L}$ intersecting $P$ are pairwise disjoint
\end{enumerate}
We call the rectangle $[a^\prime ,b^\prime ]\times [c^\prime ,d^\prime]$ with the maximal $b^\prime$, such that no line segment intersects it, the root of $P$. Figure \ref{f1ls} contains an example for a rectangle $R$ with a family of line segments and two probes.
\end{defn}

\begin{figure}[ht]
\begin{center}
\includegraphics[scale=1]{probe}
\end{center}
\caption{Illustration of line segments, probes and roots in a rectangle R}
\label{f1ls}
\end{figure}


Let $R$ be an axis-aligned rectangle with a non-empty interior. Let $(s_i)_{i\in\N}$ and $(p_i)_{i\in\N}$ be sequences inductively defined by setting $s_1=p_1=1$, $s_{i+1}=(p_i+1)s_i+p_i^2$ and $p_{i+1}=2p_i^2$. For each $k\in\N$, we then define a family $\mathcal{L}_k$ of $s_k$ line segments and a set $\mathcal{P}_k$ of $p_k$ probes in $R$ as follows. Let $\mathcal{L}_1$ contain an arbitrary non-horizontal line segment in $R$ and choose any rectangle in $R$ that shares its right boundary with $R$ and intersects the line segment with its lower and upper bound as the probe in $\mathcal{P}_1$. Now suppose $k\geq 2$ for the inductive step. To construct $\mathcal{L}_k$ draw $\mathcal{L}_{k-1}$ in $R$ and for each probe $p\in\mathcal{P}_{k-1}$ place another copy $\mathcal{L}_p$ of $\mathcal{L}_{k-1}$ with probes $\mathcal{P}_{p}$ in the root of $p$. Note that the probes in $\mathcal{P}_{p}$ are not extending to the right boundary of $R$ but instead end at the right boundary of the root of $p$. Afterwards, for each $p\in\mathcal{P}_{k-1}$ and every $q\in\mathcal{P}_{p}$ draw the diagonal $d_q$ of $q$, i.e. a line segment from its bottom-left corner to its upper-right corner. The diagonal $d_q$ crosses all segments intersecting $q$ but no other. Then, $\mathcal{L}_k$ contains the line segments from the $p_{k-1} + 1$ copies of $\mathcal{L}_{k-1}$ and the $p_{k-1}^2$ diagonals, i.e. $\vert \mathcal{L}_k\vert =(p_{k-1} + 1)s_{k-1}+p_{k-1}^2=s_k$. 

Now we want to construct $\mathcal{P}_k$. For each $p\in\mathcal{P}_{k-1}$ and every $q\in\mathcal{P}_{p}$ let $\mathcal{S}(p)$ be the set of line segments in $\mathcal{L}_{k-1}$ intersecting $p$ and $\mathcal{S}_p(q)$ the segments in $\mathcal{L}_k$ intersecting $q$. For each such $q$ we add two probes to $\mathcal{P}_k$. We place the first one, the upper probe $u_q$, close to the top of $q$, such that the diagonal $d_q$ but no other segment in $\mathcal{S}_p(q)$ intersects it and choose the second probe, the lower probe $l_q$, close to the bottom of $q$ such that it contains all segments in $\mathcal{S}_p(q)$ but not $d_q$. Then, both probes end at the right boundary of $R$. Since line segments intersecting $p$ are pairwise disjoint by the induction hypothesis, and since we placed the line segments in $\mathcal{S}_p$ in a way that they do not cross any segment in $p$ as well, $\mathcal{S}(p)\cup \lbrace d_q\rbrace$ and $\mathcal{S}(p)\cup\mathcal{L}_p(q)$ are both independent sets, i.e. both $u_q$ and $l_q$ are proper probes. Finally, observe that $\vert \mathcal{P}_k\vert = 2p_{k-1}^2=p_k$.

\begin{figure}[ht]
\begin{center}
\includegraphics[scale=0.75]{diagonal}
\end{center}
\caption{Probe $q$ with its diagonal $d_q$ and the two newly created probes $u_q$ and $l_q$}
\label{f2ls}
\end{figure}


Let $G_k$ denote the intersection graph of the line segments in $\mathcal{L}_k$, i.e. each vertex in $G_k$ represents a line segment of $\mathcal{L}_k$ and two vertices are adjacent if and only if the corresponding line segments intersect each other.\\

As mentioned before, the construction by Burling as presented in \cite{Fe18} results in the same family of graphs. For each $k\in\N$, construct the Burling graph $B_k$ and a corresponding set of stable sets $S(B_k)$ as follows. Let $B_1$ be the graph on a single vertex and let $S(B_1)$ contain the only stable set on one vertex in $B_1$. For $k\geq 2$, consider a copy $H$ of $B_{k-1}$ which we think of as the original copy and one further copy $H_S$ of $B_{k-1}$ for each stable set $S\in S(B_{k-1})$. Let $S(H_S)$ denote the copy of $S(H)$ in $H_S$. Furthermore, for each copy $H_S$ and stable set $X\in S(H_S)$ add another vertex $v_{S,X}$ adjacent to all vertices in $X$ but no other vertex. Denote the graph obtained from a copy $H_S$ and added new vertices for each stable set $X\in S(H_S)$ as $H_S^{\prime}$. Then, define the graph $B_k$ as $B_k=H\cup\bigcup_{S\in S(B_k)} H_S^{\prime}$. Its set of stable sets $S(B_k)$ consists of two sets for each $S\in S(H)$ and $X\in S(H_S)$, namely $S\cup \lbrace v_{S,X}\rbrace$ and $S\cup X$. Note that both sets are stable in $B_k$. Refer to Figure \ref{f4ls} for an illustration of the first three Burling graphs.

\begin{figure}[ht]
\begin{center}
\includegraphics[scale=1]{burling}
\end{center}
\caption{The first three Burling graphs with added vertices in the copies colored red}
\label{f4ls}
\end{figure}

\begin{lemma}
The families of graphs obtained by the two constructions are the same, i.e. $\lbrace G_k :k\in\N\rbrace=\lbrace B_k :k\in\N\rbrace$. 
\end{lemma}
\begin{prf}
We shall show that $G_k$ and $B_k$ are isomorphic under an isomorphism $\varphi_k$ for all $k\in\N$ by induction on $k$. The idea is to define a bijection $f_k:\mathcal{P}_k\to S(B_k)$ between the probes in $\mathcal{P}_k$ and the stable sets in $S(B_k)$ in each step such that the images of the vertices corresponding to the line segments in a probe $p\in\mathcal{P}_k$ under $\varphi_k$ are exactly the vertices in $f_k(p)$. We then use these maps to define an isomorphism between $G_{k+1}$ and $B_{k+1}$.

For $k=1$, $G_1$ and $B_1$ are graphs on a single vertex and thus isomorphic with isomorphism $\varphi_1$. Define a bijection $f_1:\mathcal{P}_1\to S(B_1)$ that maps the one probe in $\mathcal{P}_k$ to the one stable set in $S(B_k)$.

Suppose $k\geq 2$, $G_{k-1}$ and $B_{k-1}$ are isomorphic, and the bijection $f_{k-1}$ between $\mathcal{P}_{k-1}$ and $S(B_{k-1})$ exists as wanted. We extend the isomorphism to pairs of copies $(\mathcal{L}_p, H_{f(p)})$ of $G_{k-1}$ and $B_{k-1}$ respectively for all probes $p\in \mathcal{P}_{k-1}$ and thus covering all copies of $G_{k-1}$ in $G_k$ and $B_{k-1}$ in $B_{k}$. We do this by setting $\varphi_k (v^\prime)=u^\prime$ if and only if $v^\prime\in \mathcal{L}_p$ is the copy of a vertex $v\in G_{k-1}$ and $u^\prime$ is the copy of $\varphi_{k-1}(v)\in H_{f(p)}$. Then, we map the remaining vertices according to the bijection between $\mathcal{P}_k$ and $S(B_k)$: for each $p\in\mathcal{P}_{k-1}$ and $q\in\mathcal{P}_p$ ($q$ is the copy of some probe $q^\prime\in\mathcal{P}_{k-1}$) map the vertex corresponding to $d_q\in V(G_{k-1})$ to the vertex $v_{S,X}\in B_{k-1}$ where $S=f(p)$ and $X$ is the copy of $f(q^\prime)$ in $S(H_S)$. That yields an isomorphism $\varphi_k$ between $G_k$ and $B_k$ since the images of the neighbors of $d_q$ are the neighbors of $v_{S,X}$ in $B_k$. Finally, define $f_k:\mathcal{P}_k\to S(B_k)$ as $f(u_q)= S\cup \lbrace v_{S,X}\rbrace$ and $f(l_q)= S\cup X$.\qed
\end{prf}

Since both constructions yield the same family of graphs, let us call it $\mathcal{LG}$, it is enough to consider just the Burling graphs for the following proofs. The results hold for both constructions. First, we want to assert that the graphs are indeed triangle-free and that we can find graphs of arbitrarily large chromatic number in $\mathcal{LG}$.


\begin{thm}[Felsner et al. \cite{Fe18}]
All graphs in $\mathcal{LG}=\lbrace B_k :k\in\N\rbrace$ are triangle-free.
\end{thm}

\begin{prf}
We prove this by induction on $k$. The graph $B_1$ is trivially triangle-free. For each $k\geq 2$, the graph $B_k$ is triangle-free since $B_{k-1}$ is triangle-free by induction hypothesis and the neighborhood of every vertex $v_{S,X}$ we introduce in $B_k$ is a stable set.\qed
\end{prf}

\begin{thm}[Felsner et al. \cite{Fe18}]\label{t2ls}
For all $k\in\N$, the graph $B_k$ has chromatic number $\chi (B_k)$ at least $k$.
\end{thm}

\begin{prf}
We show this by proving a stronger statement by induction on $k$. For any proper coloring $c$ of $B_k$, there exists a stable set $S\in S(B_k)$ such that $c$ uses at least $k$ colors on the vertices in $S$.

This is again trivial for $k=1$. Now, suppose $k\geq 2$ and $c$ is a proper coloring of $B_k$. By induction hypothesis, there exists a stable set $S\in S(B_{k-1})$ $c$ uses at least $k-1$ colors on. Also by induction hypothesis, there exists a stable set $X\in S(H_S)$ $c$ uses at least $k-1$ colors on. If $c$ uses at least $k$ colors on $(S\cup X)\in S(B_k)$, we are done. Otherwise, $c$ uses the same $k-1$ colors on $S$ and $X$. Since $v_{S,X}$ is adjacent to each vertex in $X$, it is colored in a color not used on a vertex in $S$ and thus $c$ uses at least $k$ colors on $S\cup\lbrace v_{S,X}\rbrace$.\qed 
\end{prf}

\begin{comment}
\begin{thm}[Pawlik et al. \cite{Paw14}]
For every $k\in\N$ exists a family $\mathcal{L}$ of line segments in the plane with no three pairwise intersecting segments and $\chi (G)\geq k$, where $G$ is the intersection graph of $\mathcal{L}$.
\end{thm}

\begin{prf}
 For every $k\in\N$ we now construct a family $\mathcal{L}_k$ of $s_k$ line segments, such that no three are pairwise intersecting, and a family $\mathcal{P}_k$ of $p_k$ pairwise disjoint probes in $R$, such that for every proper coloring $c$ of $\mathcal{L}_k$ there is a probe $P\in\mathcal{P}_k$ for which $c$ uses at least $k$ colors on the line segments of $\mathcal{L}_k$ intersecting $P$, by induction on $k$. Then, the corresponding intersection graph $G_k$ for $\mathcal{L}_k$ is triangle-free and $\chi (G)\geq k$.

For the base case $k=1$ consider an arbitrary non-horizontal line segment in $R$. Let $\mathcal{L}_1$ just contain this segment and choose any rectangle in $R$ that shares its right boundary with $R$ and intersects the line segment with its lower and upper bound as the probe $p$ in $\mathcal{P}_1$. $G_1$ is obviously triangle-free and $p$ uses exactly one color on the segment it contains.

For the induction step, consider a given rectangle $R$ and families $\mathcal{L}_k$ and $\mathcal{P}_k$ for a fixed $k\in\N$. To construct $\mathcal{L}_{k+1}$ draw $\mathcal{L}_k$ in $R$ and for each probe $p\in\mathcal{P}_k$ place another copy $\mathcal{L}_p$ of $\mathcal{L}_k$ with probes $\mathcal{P}_{p}$ in the root of $p$. Afterwards, for each $p\in\mathcal{P}_k$ and every $q\in\mathcal{P}_{p}$ draw the diagonal $d_q$ of $q$, i.e. a line segment from its bottom-left corner to its upper-right corner. $d_q$ crosses all segments intersecting $q$ but no other. Then, $\mathcal{L}_k$ consists of the segments from the $p_k + 1$ copies of $\mathcal{L}_k$ and the $p_k^2$ diagonals, i.e. $\vert \mathcal{L}_{k+1}\vert =(p_k + 1)s_k+p_k^2=s_{k+1}$. Since the copies of $\mathcal{L}_k$ are triangle-free and disjoint and $d_q$ intersects only the segments in $q$, an independent set of segments, $\mathcal{L}_{k+1}$ is also triangle-free.

Now we want to construct $\mathcal{P}_{k+1}$. For each $p\in\mathcal{P}_k$ and every $q\in\mathcal{P}_{p}$ let $\mathcal{L}(p)$ be the set of segments in $\mathcal{P}_k$ intersecting $p$ and $\mathcal{L}_p(q)$ the segments in $\mathcal{L}(p)$ intersecting $q$. We add two probes to $\mathcal{P}_{k+1}$. We place the first one, the upper probe $u_q$, close to the top of $q$, such that the diagonal $d_q$ but no other segment in $\mathcal{L}_p(q)$ intersects it and choose the second probe, the lower probe $l_q$, close to the bottom of $q$ such that it contains all segments in $\mathcal{L}_p(q)$ but not $d_q$. Then, both probes end at the right boundary of $R$. By the induction hypothesis and the way we placed $\mathcal{L}_p$, $\mathcal{L}(p)\cup \lbrace d_q\rbrace$ and $\mathcal{L}(p)\cup\mathcal{L}_p(q)$ are both independent sets, i.e. both $u_q$ and $l_q$ are proper probes. Finally, observe that $\vert \mathcal{P}_{k+1}\vert = 2p_k^2=p_{k+1}$.

Let $c$ be a coloring of $\mathcal{L}_{k+1}$. Consider the restriction of $c$ to the original copy of $\mathcal{L}_k$. By the induction hypothesis, there exists a probe $p\in\mathcal{P}_k$ for which $c$ needs $k$ colors to paint the line segments in $p$. Now, consider the copy $\mathcal{L}_p$ of $\mathcal{L}_k$ in the root of $p$. Again, by induction hypothesis, there is a probe $q\in\mathcal{P}_p$ that uses $k$ colors on the segments in $\mathcal{L}_p$ intersecting $q$. If the colors used by $c$ in $p$ and $q$ are different, at least $k+1$ colors are used on the segments pierced by the lower probe $l_q$. Otherwise, $d_q$ has a different color than the colors used in $p$ and $q$ and thus, $c$ uses $k+1$ colors on the segments pierced by the upper probe $u_q$.\qed
\end{prf}
\end{comment}
Similar to the construction of triangle-free graphs of large chromatic number by Mycielski, there exist rather simple proofs showing that such graphs provided by the construction above contain induced copies of stars and paths as well.
First, note that $B_{k+1}$ contains an induced copy of $B_k$ for all $k\in\N$. This follows directly from the construction. With this knowledge it is enough to prove the existence of an induced copy of a tree $T$ in some $B_k$. Then $T$ is no suitable candidate to disprove the conjecture as it is contained in a graph of arbitrarily large chromatic number.

\begin{thm}
Let $k\in\N$. Then, $B_{k+1}$ contains an induced copy of $K_{1,k}$.
\end{thm}
\begin{prf}
Let $k\in\N$ be fixed and consider $B_k$. By the proof of Theorem \ref{t2ls}, there exists a stable set $S\in S(B_k)$ such that any proper coloring $c$ of $B_k$ uses at least $k$ colors on the vertices in $S$. Then $S$ has at least size $k$. Consider the copy $X\in S(H_S)$ of $S$ in $H_S$. By definition, $X$ is a stable set and contains at least $k$ vertices. Thus, the vertex set $X\cup \lbrace v_{S,X}\rbrace$ induces a copy of $K_{1,k}$ in $B_k$.\qed   
\end{prf}

\begin{thm}\label{t3ls}
Let $k\in\N$. Then, $B_{k+1}$ contains an induced copy of $P_k$.
\end{thm}
\begin{prf}
We prove the existence of an induced copy of $P_k$ in $B_{k+1}$ by induction on $k$. The idea is to construct an induced path $Q_k$ on $k$ vertices by choosing a vertex $v_{S,X}$ for some stable sets $S\in S(B_k)$ and $X\in S(H_S)$ as the vertex we add to our path in step $k$.

Launch the induction with $k=1$. Consider $B_1$. We have just one stable set $S$ in $S(B_1)$, i.e. there exists just one vertex $v_{S,X}$ in $B_2$ where $X$ is the copy of $S$ in $S(H_S)$. Define $Q_1=(v_{S,X})$. This is an induced copy of $P_1$  in $B_2$.

For the induction step, let $k\geq 2$ and let $Q_{k-1}$ be an induced path on $k-1$ vertices in $B_k$. We construct an induced path $Q_k$ on $k$ vertices as follows. Let $v=v_{S,X}$ be the last vertex of $Q_{k-1}$. Then, $S^\prime = S\cup \lbrace v\rbrace$ is a stable set in $S(B_k)$. Observe that the copies $H$ and $H_S$ of $B_{k-1}$ are edge-disjoint in $B_k$. Thus, no vertex in $V(Q_{k-1})\setminus\lbrace v\rbrace$ is also an element of $S$. Now consider $B_{k+1}$ and choose an arbitrary stable set $R\in S(B_k)$. Let $Q_{k-1}^\prime$ denote the copy of $Q_{k-1}$ in $H_R$ and let $v^\prime$ denote the copy of $v$. Note that $Q_{k-1}^\prime$ is an induced copy of $P_{k-1}$ as well. Consider the copy of $S^\prime$ in $S(H_R)$, call it $S^{\prime\prime}$. As before no vertex in $V(Q_{k-1}^\prime )\setminus\lbrace v^\prime\rbrace$ is a vertex of $S^{\prime\prime}$. Thus, the vertex $v_{R,S^{\prime\prime}}$ is adjacent to the copy of $v^\prime$ but no other vertex in $Q_{k-1}^\prime$ and $Q_k=Q_{k-1}^\prime v_{R,S^{\prime\prime}}$ is an induced path on $k$ vertices in $B_{k+1}$. See Figure \ref{f3ls} for an illustration.\qed
\end{prf}

\begin{figure}[ht]
\begin{center}
\includegraphics[scale=1]{path_ls}
\end{center}
\caption{Choice of the vertex added in the inductive step}
\label{f3ls}
\end{figure}

Finally, there exists an inductive proof showing that actually for every tree $T$ we can find a Burling graph $B_k$ for some $k\in\N$ that contains an induced copy of $T$. Therefore, let us define a sequence of trees $T_1, T_2, \dots$ where $T_1$ is the tree on one vertex and for all $i\in\N$ let $T_{i+1}$ be obtained from $T_i$ by adding a new vertex only adjacent $t$ for each vertex $t\in T_i$. Thus, $T_i$ is atree on $2^{i-1}$ vertices. Note that the following proof works by the same principle as the proof of Theorem \ref{t3ls}.

\begin{thm}\label{t1ls}
Let $k\in\N$. The graph $B_k$ contains an induced copy of $T_k$.
\end{thm}

\begin{prf}
We shall prove a stronger statement. We claim that $T_k\subseteq_I B_k$ for all $k\in\N$ and that for each vertex $t\in V(T_k)$ there exists a stable set $S_t\in S(B_k)$ containing exactly $t$ but no other vertex from $V(T_k)$. The proof goes by induction on $k$. The idea of the inductive step then is to choose some copy $H_S$ of $B_k$ in $B_{k+1}$, consider an induced copy of $T_k$ in $H_S$ and add the vertices $v_{S,S_t}$ for each $t$ to form an induced copy of $T_{k+1}$.

To launch the induction let $k=1$. The tree $T_1$ consists of one single vertex $t_1$. Thus, the graph on the single vertex in $B_1$ is an induced copy of $T_1$ in $B_1$. Observe that the stable set in $S(B_1)$ contains this vertex.

Now let $k\geq 2$ and suppose the $B_{k-1}$ contains an induced copy of $T_{k-1}$, call it $Q_{k-1}$. Consider $B_k$ and choose a stable set $S\in S(B_{k-1})$. Then, $H_S$ is an induced copy of $B_{k-1}$ in $B_k$ and thus it contains an induced copy of $T_{k-1}$ as well. We denote it as $Q_{k-1}^\prime$. Let $q_1, q_2,\dots , q_{2^{k-2}}$ be the vertices of $Q_{k-1}^\prime$. By induction hypothesis, for each vertex $q$ in $Q_{k-1}$ there exists a stable set in $S(B_{k-1})$ containing $q$ but no other vertex from $Q_{k-1}$. Thus, there also exists a stable set $S_{q_i}\in S(H_S)$ containing $q_i$ but no other vertex of $Q_{k-1}^\prime$ for each $i\in [2^{k-2}]$. Now choose $v_i=v_{S,S_{q_i}}$ as the new vertex. Observe that it is only adjacent to $q_i$ and no other vertex of $Q_{k-1}^\prime$. Since the vertices $v_i$, $i\in [2^{k-2}]$, are pairwise non-adjacent by definition in the construction of Burling graphs, the graph $Q_k$ induced by the vertex set $V(Q_{k-1}^\prime)\cup \lbrace v_i:i\in [2^{k-2}]\rbrace$ is a tree and by the choice of the new vertices isomorphic to $T_k$. Furthermore, with the stable sets $S\cup \lbrace v_i\rbrace$ and $S\cup S_{q_i}$ in $S(B_k)$ we have one, $S\cup \lbrace v_i\rbrace$, that contains only $v_i$ and one, $S\cup S_{q_i}$, that only contains $q_i$ of the vertices in $Q_k$.\qed
\end{prf}

\begin{cor}
For every tree $T$, there is a $k\in\N$ such that $T\subseteq_I B_k$.
\end{cor}
\begin{prf}
Let $T$ be a tree. Denote its radius by $r$ and let $v\in V(T)$ be a vertex such that the distance between $v$ and a vertex $u\in V(T)\setminus \lbrace v\rbrace$ is at most $r$. Define $k=\max_{u\in V(T)\setminus \lbrace v\rbrace} (dist(v,u)+\delta (u))$. Then $T$ is an induced subgraph of $T_k$ and therefore $T\subseteq_I B_k$ holds by Theorem \ref{t1ls}.\qed
\end{prf}


Considering the construction of Burling graphs it is striking that the graphs grow large very fast because when constructing $B_k$ for some $k\in\N$, $k>1$, we add a lot of copies isomorphic to $B_{k-1}$. In addition, the size of the set of stable sets increases vastly as well. It may be possible to modify the construction while preserving its properties, namely that each graph is triangle-free and we construct graphs of arbitrarily large chromatic number, and obtain graphs not inducing certain trees. The first idea is to simplify the construction by adding just one copy of $B_{k-1}$ when constructing $B_k$.

For $k\in\N$ define $B^\prime_k$ and $S(B^\prime_k)$ as follows. Let $B_1^\prime =B_1$, i.e. a graph on a single vertex and let $S(B_1^\prime )$ contain the stable set on this single vertex.

Now construct $B^\prime_k$ for $k\geq 2$. Let $H$ denote a copy of $B_{k-1}^\prime$ and $S(H)$ a copy of $S(B_{k-1}^\prime )$ for $H$. Take another copy $H^\prime$ of $B_{k-1}^\prime$ with stable sets $S(H^\prime )$. Introduce vertices $v_X$ for every stable set $X\in S(H^\prime )$ and add an edge between $v_X$ and each vertex in $X$ such that $X\cup \{v_X\}$ forms a star. Denote the graph obtained by the vertex introductions as $H^{\prime\prime}$ and define $B_k^\prime =H\cup H^{\prime\prime}$. For every $S\in S(H)$ and $X\in S(H^\prime)$ let $S(B_k^\prime )$ contain two stable sets $S\cup\{v_X\}$ and $S\cup X$.

\begin{thm}
For all $k\in\N$, the graph $B_k^\prime$ is triangle-free.
\end{thm}
\begin{prf}
This is trivial for $k=1$ and for $k\geq 2$ the neighborhood of the newly introduced vertices form stable sets. Thus, $B_k^\prime$ is triangle-free.\qed
\end{prf}

\begin{thm}
For all $k\in\N$, the graph $B_k^\prime$ has chromatic number $\chi (B_k^\prime)\geq k$.
\end{thm}
\begin{prf}
As before, we prove a stronger statement: For all $k\in\N$, there exists a stable set $S\in S(B_k^\prime )$ a proper coloring of $B_k^\prime$ uses at least $k$ colors on.

For $k=1$ this is trivial. For $k\geq 2$, let $c$ be a proper coloring of $B_k^\prime$. By induction hypothesis there exists a stable set $S\in S(H)$ and a stable set $X\in S(H^\prime )$ such that $c$ uses at least $k-1$ colors on each of the two sets. Observe that $S\cap X=\emptyset$. Then, $c$ uses either at least $k$ colors on the stable set $S\cup X\in S(B_k^\prime)$ or exactly the same $k-1$ colors on both. In this case $v_X$ is colored in a color not used in $X$ and $S\cup \{v_X\}\in S(B_k^\prime)$ needs at least $k$ colors.\qed
\end{prf}

Thus, the properties of the original Burling graphs are maintained. Unfortunately, that is also true for the property that $B_k^\prime$ contains an induced subgraph isomorphic to a tree $T_k$.

\begin{thm}
Let $k\in\N$. The graph $B_k^\prime$ contains an induced subgraph isomorphic to $T_k$.
\end{thm}
\begin{prf}
Similarly to the proof for the Burling graphs we prove a stronger statement by induction on $k$: For all $k\in\N$ there exists a graph $T$ isomorphic to $T_k$ with $T\subseteq_I B_k^\prime$ and for each vertex $t\in T$ there exists a stable set $S(B_k^\prime )$ containing $t$ but no other vertex from $V(T)$.

For $k=1$, the graphs $B_1^\prime$ and $T_1$ are both graphs on a single vertex and thus isomorphic. Furthermore, the stable set in $S(B_1^\prime)$ contains this vertex.

Now let $k\geq 2$ and consider $B_k^\prime$. The subgraph $H^\prime$ is a copy of $B_{k-1}^\prime$ and contains a subgraph $Q$ isomorphic to $T_{k-1}$ by induction hypothesis.  Let $q_1, q_2,\dots , q_{2^{k-2}}$ denote the vertices of $Q$. By induction hypothesis, for each vertex $q_i\in V(Q)$, $i\in [2^{k-2}]$, there exists a stable set $X\in S(H^\prime)$ containing $q_i$ but no other vertex from $Q$. We choose $v_i=v_X$ as the new vertex. It is only adjacent to $q_i$ but no other vertex of $Q$. Since the vertices $v_i$, $i\in [2^{k-2}]$ form a stable set by the construction, the graph $T$ induced by the vertices in $V(Q)\cup \{v_i:i\in [2^{k-2}]\}$ is a tree and furthermore isomorphic to $T_k$ since for each vertex $q\in Q$ we found exactly one new vertex only adjacent to $q$ as in the definition of $T_k$. Choose a stable set $S\in S(H)$. With the stable sets $S\cup \{v_i\}$ for $v_i$ and $S\cup X$ for $q_i$ we also have a stable sets containing exactly one vertex $t$ of $T$ for all $t\in V(T)$.\qed
\end{prf}

The second idea is to reduce the number of stable sets in $S(B_k)$ for a Burling graph $B_k$. 