Making statements about induced subgraphs in a graph based on nothing but its chromatic number poses a great challenge. A family of graphs $\mathcal{F}$ is said to be $\chi$-bounded if there exists a function $f:\N\to\N$ such that for every graph $G\in\mathcal{F}$, $\chi (G)\leq f(\omega (G))$, where $\chi (G)$ denotes the chromatic number of $G$ and $\omega (G)$ the size of the largest clique in $G$. A graph $H$ is called $\chi$-bounding if the family of graphs not containing an induced subgraph isomorphic to $H$ is $\chi$-bounded.

The problem of a classification of graphs into $\chi$-bounding and not $\chi$-bounding graphs dates back to Gyárfás in 1975 \cite{Gy75}. With the famous result by Erd\H{o}s and Hajnal proving that there exist graphs of arbitrarily large chromatic number and girth it is easy to see that $\chi$-bounding graphs have to be acyclic. Gyárfás \cite{Gy75} and Sumner \cite{Su81} independently conjectured that indeed every acyclic graph is $\chi$-bounding. Proving the Gyárfás-Sumner conjecture as a whole is a complex endeavor. Even though many tried to prove it in a period of over forty years, nobody succeded yet. That is why many proofs concentrate on certain graphs and prove it only partially. Nevertheless, there is still not much known and many cases are open.
\\

We begin in Chapter \ref{secBC} with providing the definitions we need later on. There, we also introduce the concepts we use in the following chapters, in particular one regarding $\chi$-bounding functions.

The Gyárfás-Sumner conjecture itself is presented in Chapter \ref{secCR}. After a collection of cases already known to be true, we look at proofs concerning simple graphs like stars and paths. Additionally, this chapter includes some results for weaker statements than the conjecture. We give a proof that for each tree $T$ there exist graphs with certain girth and minimum degree containing an induced subgraph isomorphic to $T$. Furthermore, we reproduce a theorem by Kierstead and Rödl \cite{Ki96} showing that for each tree $T$ the family of graphs containing no $K_{n,n}$ for some $n\in\N$ and no induced subgraph isomorphic to $T$ is $\chi$-bounded.
\\

As mentioned, proving the conjecture is hard and mathematicians do not agree on whether it is actually true. While many are working on proving more cases of the conjecture, it might be worth looking for a counterexample, i.e. finding a tree $T$ such that there exist graphs of arbitrarily large chromatic number that do not contain a subgraph isomorphic to $T$. We consider constructions providing families of triangle-free graphs with chromatic number at least $k$ for each $k\in\N$ since the constant clique number assures that there exists no bounding function for these families of graphs.

The first family of graphs $\mathcal{M}$ obtained by Mycielski's construction \cite{My55} is examined in Chapter \ref{secMy}. After proving the basic properties of the constructed graphs, we slowly work towards proving that for each $T$ there exists a graph in $\mathcal{M}$ containing an induced subgraph isomorphic to $T$.

In Chapter \ref{secLS} we consider a second family of graphs $\mathcal{B}$. There exist two different constructions both resulting in this family of graphs: One, by Pawlik et al. \cite{Paw14}, uses the intersection graph of line segments in an axis-aligned rectangle while the other, by Burling \cite{Bu65}, relies on certain stable sets to construct the next graph in the sequence. We present both and again prove the basic properties of graphs in $\mathcal{B}$. We give a proof for the existence of a graph in $B$ containing an induced subgraph isomorphic to $T$ for each tree $T$ as well. Furthermore, we modify the Burling construction while preserving the property to obtain triangle-free graphs of arbitrarily large chromatic number and show that even then, it can not be used as counterexample for the Gyárfás-Sumner conjecture since for each $T$ there still exists a graph that contains an induced subgraph isomorphic to $T$.

Finally, Chapter \ref{secConc} provides a summary of the results we proved throughout this thesis and gives an outlook which cases of the Gyárfás-Sumner conjecture still remain open.