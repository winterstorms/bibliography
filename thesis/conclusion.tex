In this thesis we examined the Gyárfás-Sumner conjecture and considered some cases for simple graphs like stars and paths. We went on with a result by Kierstead and Rödl showing that $\textit{Forb}(K_{n,n},T)$ is $\chi$-bounded for each tree $T$ and $n\in\N$. Then, we proved some weaker statements. For example, a graph with girth at least $k+1$, $k\in\N$, and minumum degree at least $d\in\N$ contains an induced subgraph isomorphic to $T$ for each tree $T$ on at most $k$ vertices and maximum degree at most $d$.

Afterwards we focused on finding a counterexample for the Gyárfás-Sumner conjecture by checking the Mycielski construction and the Burling construction if one of the families of graphs provided by these constructions is a subset of $\textit{Forb}(T)$ for a tree $T$. Since the result was negative, we are now able to eliminate both constructions as candidates for disproving the Gyárfás-Sumner conjecture. This also applies to the modified construction we derived from Burlings construction.% Another approach to find a counterexample could be to combine these results and try to come up with a construction of graphs with arbitrarily large chromatic number where vertices of a high degree have a certain distance such that these graphs would not contain a subgraph isomorphic to 

In contrast, there are many cases of the Gyárfás-Sumner conjecture that are still open. It may be worth a try to prove the conjecture for trees obtained from three stars joined by a star through their center vertices, an arbitrary caterpillar tree or, although that seems hard, trees of a certain radius of at least three.

