This chapter focuses on providing the basis for all following theorems and proofs by displaying most of the definitions regularly used in this thesis. I collected and structured them starting with the simple ones and advancing to more complex and specific concepts. This should give an overview and a better understanding of the methods we will use later on.
\\[2ex]
We will start with defining the concept of a graph. For a set $X$, let us denote the set of all $2$-element subsets of $X$ with ${X\choose 2} =\lbrace \lbrace x,y\rbrace : x,y\in V, x\neq y\rbrace$.
\begin{defn}[Graph]
A \textit{graph} $G$ is a tuple $G=(V,E)$, where $V=V(G)$ denotes the set of vertices and $E=E(G)\subseteq {V\choose 2}$ is the set of edges. Here, each edge is a set of two vertices from $V$.
\end{defn}

By giving each edge a direction, i.e. saying the edge $(x,y)$ points from $x$ to $y$, we create a \textit{directed graph} $D=(V,E)$ where $V$ again denotes the set of vertices but $E\subseteq \lbrace (x,y) : x,y\in V, x\neq y\rbrace$ is now a set of tuples. In a directed graph, there can exist two edges between two vertices $x$ and $y$: one pointing to $x$ and one pointing to $y$. If there exists at most one edge for any pair of vertices in a directed graph $D$, we call $D$ also an \textit{oriented graph}.

\begin{note}
Note that any of the following concepts and definitions can be applied to directed graphs as well.
\end{note}

If two vertices are neighbors in a graph $G=(V,E)$, i.e. there exists an edge between them, they are \textit{adjacent} to each other. Then, the edge and the vertices are called \textit{incident} to each other. We denote the neighbors of a vertex $v\in V$ in $G$ as $N_G(v)$ or just $N(v)$ if there is no confusion in regard of the underlying graph. Furthermore, we say $v$ is of \textit{degree} $\delta (v) =\vert N(v)\vert$. The maximum degree in a graph is denoted by $\Delta (G)$. Since edges in a directed graph do not necessarily all point in the same direction, there we distinguish between the \textit{in-degree} $\delta^{in}(v)$ of a vertex $v$ which counts the number of edges ending in $v$ and the \textit{out-degree} $\delta^{out}(v)$ which counts the number of edges starting at $v$.
\\[2ex]
After introducing the concept of a graph and the relations between edges and vertices we now start looking for structures of a greater scale in it either by considering just the vertices and edges or later by additionally defining colorings of the graph. 

Let $G=(V,E)$ be a graph. A graph $H=(V^\prime ,E^\prime )$ is a \textit{subgraph} of $G$, written $H\subseteq G$, if $V^\prime\subseteq V$, $E^\prime\subseteq \lbrace \lbrace x,y\rbrace : x,y\in V^\prime, x\neq y\rbrace$ and any edge $e\in E^\prime$ is also in $E$. The graph $H$ is called an \textit{induced subgraph} of $G$ and we write $H\subseteq_I G$ if $E^\prime = E\vert_{V^\prime}$, i.e. for any edge $e=\lbrace x,y\rbrace\in E$ holds: if $x\in V^\prime$ and $y\in V^\prime$, then $e\in E^\prime$. The subgraph induced in $G$ by a certain set of vertices $A\subseteq V$ is denoted as $G[A]$. Furthermore, the structures of two graphs can coincide completely: Two graphs $G$ and $H$ are called \textit{isomorphic} if there exists a bijection $\phi :V(G)\to V(H)$ such that $(\phi (u), \phi (v))\in E(H)$ if and only if $(u, v)\in E(G)$. We denote such a relation as $G\approx H$.

A subset $S\subseteq V$ of pairwise non-adjacent vertices, i.e. $E\vert_S =\emptyset$, is called a \textit{stable} or \textit{independent} set. Opposite to that, a \textit{clique} is a set of vertices $S\subseteq V$ such that all vertices in $S$ are pairwise adjacent. The size of the largest clique in $G$ is denoted by $\omega (G)$. A \textit{coloring} $c$ of $G$ with $r$ colors is a function $c:V\to [r]$ that maps a color to each vertex. It is a proper coloring if $\forall e = \lbrace x,y\rbrace\in E: c(x)\neq c(y)$, i.e. no two adjacent vertices have the same color. Then, we can define the chromatic number $\chi (G)$ of $G$ as the minimum number of colors needed for a proper coloring of $G$. Observe that each color class of a proper coloring, i.e. all vertices of the same color, is a stable set.

A way of obtaining a proper coloring for a graph $G$ that always works is the \textit{greedy coloring}: Order the vertices $v_1, v_2, \dots ,v_{\vert V(G) \vert }$ and define a coloring $c$ by assigning each vertex the first available color, i.e. the first color not used on a neighbor with smaller index. Note that it often is far from optimal regarding the number of used colors.
\\[2ex]
Since the crucial part of this thesis deals with the conjecture of Gyárfás and Sumner, we want to introduce the notion of $\chi$-bounded families of graphs. In addition, this part states the definitions of a hypergraph and the Ramsey number because they are used in many proofs in the following chapters.

\begin{defn}[Bounding function]\label{d1}
A $\chi$\textit{-bounding function} $f:\N\to\N$ for a family $\mathcal{F}$ of graphs is a function such that $\chi (G)\leq f(\omega (G))$ holds for each $G\in\mathcal{F}$. Then, $\mathcal{F}$ is called $\chi$-bounded.
\end{defn}

A \textit{hypergraph} $H=(V,E)$ is a graph where $V$ is a set of vertices and $E$ a set of non-empty subsets of $V$, called hyperedges. If each edge has the same size $s$, $H$ is a $s$-uniform hypergraph. Note that the concepts of colorings also apply for hypergraphs. Here, a coloring $c$ of a hypergraph $H$ is proper if there exists no monochromatic edge, i.e. there is no edge $e\in E(H)$ with $c(v)=\alpha$ for all $v\in e$ and some color $\alpha$.

The Ramsey number $R=R(r_1,\dots ,r_s)$ is the minimum number $R\in\N$, such that for every coloring $c:[R]^2\to [s]$, there exist a color $\alpha\in [s]$ and a subset $X\subseteq [R]$, $\vert X\vert =r_{\alpha}$, with $c(i,j)=\alpha$ for all $(i,j)\in [X]^2$, $i\neq j$.
\\[2ex]
Proving the conjecture as a whole is obviously a complex endeavor. That is why many proofs concentrate on certain graphs and prove it only partially. Therefore, the next definitions are a collection of the most used special graphs or characterizations of some graphs that appear throughout this thesis.


Consider a graph $G=(V,E)$ with $\vert V\vert = n$. If $E=\emptyset$, we call $G$ the empty graph on $n$ vertices. If $E= {V\choose{2}}$, $G$ is also denoted as $K_n$, the complete graph on $n$ vertices. Note that then a clique of size $k$ is an induced copy of a $K_k$. The graph $G$ is called bipartite if $V=A\dot{\cup} B$ for two disjoint stable sets $A$ and $B$. The complete bipartite graph $K_{m,n}$ is a bipartite graph with $\vert A\vert =m$, $\vert B\vert =n$ and $E=\lbrace \lbrace a,b\rbrace :a\in A,b\in B\rbrace$.

An $u$-$v$\textit{-path} is a sequence of distinct vertices starting with $u$ and ending with $v$, such that there exists an edge between every two consecutive vertices. If $u=v$ but all other vertices are still distinct, the sequence is called a \textit{cycle}. A graph that does not contain a cycle is \textit{acyclic}. We say a graph $G=(V,E)$ is \textit{connected}, if there exist a $u$-$v$-path for every pair of vertices $u,v\in V$.
With the previous definition we can define a \textit{tree} $T$ as a connected and acyclic graph. A union of disjoint trees is called a \textit{forest}.

Additionally, we can group trees after their form. We denote a \textit{path} on $n>2$ vertices as $P_n$. A \textit{star} $K_{1,n}$ on $n+1$ vertices is a tree with one center vertex of degree $n$ and $n$ leafs, i.e. vertices in a tree of degree $1$. Then, we can define a \textit{broom} as a path with a star at one end and a \textit{caterpillar} as a path where each vertex in the path furthermore may have an arbitrary number of leafs as neighbors.

A \textit{subdivision} of a graph $G=(V,E)$ is a graph obtained by subdividing edges of $G$. A subdivision of an edge $e=(u,v)\in E$ creates a new vertex $w$ and replaces $e$ with two new edges $(u,w)$ and $(w,v)$ in $E$. Observe that a subdivision of a tree is still a tree.
\\[2ex]
For the sake of the completeness of this collection of definitions, I want to add a final one, since it is used to describe some trees already proven to be $\chi$-bounding. Again, let $G=(V,E)$ be a graph. The eccentricity of a vertex $v\in V$ is defined as $\displaystyle e(v)=\max_{u \in V}\lbrace dist(v,u)\rbrace$. Then, the radius $\displaystyle r= \min_{v\in V}\lbrace e(v)\rbrace$ of $G$ is the minimum number $r\in\N$, such that there exist a vertex $v\in V$ where the distance between $v$ and any other vertex $u\in V$ is at most $r$. We often name this vertex $v$ the root of the tree.