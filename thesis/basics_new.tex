This chapter focuses on providing the basis for all following theorems and proofs by displaying most of the definitions regularly used in this thesis. They are collected and structured starting with the simple ones and advancing to more complex and specific concepts. This should give an overview and a better understanding of the methods we will use later on.\\

We will start with defining the concept of a graph. For a set $X$, let us denote the set of all $2$-element subsets of $X$ with ${X\choose 2} =\lbrace \lbrace x,y\rbrace : x,y\in V, x\neq y\rbrace$. A \textit{graph} $G$ is a tuple $G=(V,E)$, where $V=V(G)$ denotes the set of vertices and $E=E(G)\subseteq {V\choose 2}$ is the set of edges. Here, each edge is a set of two vertices from $V$.

By giving each edge a direction, i.e. saying the edge $(x,y)$ points from $x$ to $y$, we create a \textit{directed graph} $D=(V,E)$ where $V$ again denotes the set of vertices but $E\subseteq \lbrace (x,y) : x,y\in V, x\neq y\rbrace$ is now a set of tuples. In a directed graph, there can exist two edges between two vertices $x$ and $y$: one pointing to $x$ and one pointing to $y$. We call a directed graph $D$ also an \textit{oriented graph} if there exists at most one edge for any pair of vertices in $D$. See Figure \ref{f1ba} for an illustration of the differences between undirected, directed and oriented graphs. 

Note that any of the following concepts and definitions can be applied to directed graphs as well.

\begin{figure}[ht]
\begin{center}
\includegraphics[scale=1]{undir_dir_ori}
\end{center}
\caption{Undirected graph (left), directed graph (center) and oriented graph (right) on the same vertex set}
\label{f1ba}
\end{figure}

In a graph $G=(V,E)$ two vertices are \textit{adjacent} if they are neighbors, i.e. there exists an edge between them. Then, the edge and the vertices are called \textit{incident}. We denote the neighbors of a vertex $v\in V$ in $G$ as $N_G(v)$ or just $N(v)$ if there is no confusion in regard of the underlying graph. Furthermore, the \textit{degree} $d(v)$ of $v$ is defined as $d(v) =\vert N(v)\vert$. The minimum degree in a graph is denoted by $\delta (G)$, the maximum degree by $\Delta (G)$. Since edges in a directed graph do not necessarily all point in the same direction, there we distinguish between the \textit{in-degree} $d^{in}(v)$ of a vertex $v$ which counts the number of edges ending in $v$ and the \textit{out-degree} $d^{out}(v)$ which counts the number of edges starting at $v$. This concept also applies to the minimum and maximum degree of a directed graph.\\

After introducing the concept of a graph and the relations between edges and vertices we now start looking for structures of a greater scale in it either by considering just the vertices and edges or later by additionally defining colorings of the graph. 

Let $G=(V,E)$ be a graph. For a vertex set $A\subseteq V$ let us denote the set of edges in $B\subseteq E$ induced by $A$ as $B\vert_A=\lbrace \lbrace x,y\rbrace\in B : x,y\in A\rbrace$. Then, a graph $H=(V^\prime ,E^\prime )$ is a \textit{subgraph} of $G$, written $H\subseteq G$, if $V^\prime\subseteq V$ and $E^\prime\subseteq E\vert_{V^\prime}$. The graph $H$ is called an \textit{induced subgraph} of $G$ and we write $H\subseteq_I G$ if $E^\prime = E\vert_{V^\prime}$. The subgraph $(A,E\vert_A)$ induced in $G$ by a certain set of vertices $A\subseteq V$ is denoted as $G[A]$. If $A=V\setminus \{v\}$ for a single vertex $v\in V$ we often write just $G-\{v\}$ for the subgraph induced by $A$. Furthermore, the structures of two graphs can coincide completely: Two graphs $G$ and $H$ are called \textit{isomorphic} if there exists a bijection $\phi :V(G)\to V(H)$ such that $\lbrace\phi (u), \phi (v)\rbrace\in E(H)$ if and only if $\lbrace u, v\rbrace\in E(G)$. We denote such a relation as $G\approx H$.

A subset $S\subseteq V$ of pairwise non-adjacent vertices is called a \textit{stable} or \textit{independent} set. Opposite to that, a \textit{clique} is a set of vertices $Q\subseteq V$ such that all vertices in $Q$ are pairwise adjacent. Observe that for a stable set $S$ $E\vert_S =\emptyset$ holds while we have $E\vert_Q={Q\choose 2}$ for a clique $Q$. The size of the largest clique in $G$ is denoted by $\omega (G)$. A \textit{coloring} $c$ of $G$ with $r$ colors is a function $c:V\to [r]$ that maps a color to each vertex. It is a proper coloring if for all edges $e = \lbrace x,y\rbrace\in E: c(x)\neq c(y)$, i.e. no two adjacent vertices have the same color. Then, we can define the chromatic number $\chi (G)$ of $G$ as the minimum number of colors needed for a proper coloring of $G$. Observe that each color class of a proper coloring, i.e. all vertices of the same color, is a stable set. If $G$ is a directed graph, its chromatic number is defined as the chromatic number of the underlying undirected graph, i.e. the undirected graph on the same vertex set, where the edge set contains an edge $\{x,y\}$ if and only if $(x,y)$ or $(y,x)$ or both are in the edge set of $G$.

A way of obtaining a proper coloring for a graph $G$ that always works is the \textit{greedy coloring}: Order the vertices $v_1, v_2, \dots ,v_{\vert V(G) \vert }$ and define a coloring $c$ by assigning each vertex the first available color, i.e. the first color not used on a neighbor with smaller index. Note that it often is far from optimal regarding the number of used colors.\\

Since the crucial part of this thesis deals with the conjecture of Gyárfás and Sumner, we want to introduce the notion of $\chi$-bounded families of graphs. In addition, this part states the definitions of a hypergraph and the Ramsey number because they are used in many proofs in the following chapters.

\begin{defn}[Bounding function]\label{d1}
A $\chi$\textit{-bounding function} $f:\N\to\N$ for a family $\mathcal{F}$ of graphs is a function such that $\chi (G)\leq f(\omega (G))$ holds for each $G\in\mathcal{F}$. Such a function does not necessarily exist for each family $\mathcal{F}$. If it exists though, $\mathcal{F}$ is called $\chi$-bounded. 
\end{defn}

A \textit{hypergraph} $H=(V,E)$ is a tuple where $V$ is a set of vertices and $E$ a set of non-empty subsets of $V$, called hyperedges. If each edge has the same size $s$, $H$ is an $s$-uniform hypergraph. Note that the concepts of colorings also apply for hypergraphs. We define a coloring $c$ of the hypergraph $H$ with $r$ colors as a function $c:V(H)\to [r]$ that assigns each vertex a color. A hypergraph coloring $c$ is proper if there exists no monochromatic edge, i.e. there is no edge $e\in E(H)$ with $c(v)=\alpha$ for all $v\in e$ and some color $\alpha\in [r]$.

The Ramsey number $R=R(r_1, r_2, \dots ,r_s)$ for integers $r_1, r_2, \dots r_s$ is the minimum number $R\in\N$, such that for every coloring $c:{[R]\choose 2}\to [s]$, there exist a color $\alpha\in [s]$ and a subset $X\subseteq [R]$, $\vert X\vert =r_{\alpha}$, with $c(\lbrace i,j\rbrace )=\alpha$ for all $\lbrace i,j\rbrace\in {X\choose 2}$. Note that the definition of the Ramsey number is based on edge colorings while the colorings of graphs we defined so far are vertex colorings.\\

The next definitions are a collection of the most used special graphs or characterizations of some graphs that appear throughout this thesis.

Consider a graph $G=(V,E)$ with $\vert V\vert = n$. If $E=\emptyset$, we call $G$ the empty graph on $n$ vertices. If $E= {V\choose{2}}$, $G$ is also denoted as $K_n$, the complete graph on $n$ vertices. Note that then a clique of size $k$ is an induced copy of a $K_k$. The graph $G$ is called bipartite if $V=A\dot{\cup} B$ for two disjoint stable sets $A$ and $B$. The complete bipartite graph $K_{m,n}$ is a bipartite graph with $\vert A\vert =m$, $\vert B\vert =n$ and $E=\lbrace \lbrace a,b\rbrace :a\in A,b\in B\rbrace$.

A path of length $n\in\N$ is a graph $P=(V,E)$ on $n+1$ vertices, where we can order the vertices $v_0, v_1, \dots ,v_n$ such that $E=\lbrace\lbrace v_{i-1},v_i\rbrace : i\in [n]\rbrace$. We say it starts in $v_0$ and ends in $v_n$. We obtain a \textit{cycle} $C=(V,E^\prime)$ of length $n$ from a path $P=(V,E)$ of length $n$ by adding the edge $\lbrace v_0,v_n\rbrace$ to the edge set, i.e. $E^\prime = E\cup\lbrace v_0,v_n\rbrace$. The \textit{girth} of a graph $G$ is the length of the shortest cycle contained in $G$. A graph that does not contain a cycle as a subgraph is \textit{acyclic} and has infinite girth. We say a graph $G=(V,E)$ is \textit{connected}, if there exist a path starting at $u$ and ending at $v$ for each pair of vertices $u,v\in V$.

With the previous definition we define a \textit{tree} $T$ as a connected and acyclic graph. A union of disjoint trees is called a \textit{forest}.

Additionally, we can group trees according to their form. We denote a \textit{path} on $n\in\N$ vertices as $P_n$. In the special case $n=1$, $P_1$ contains just a single vertex and no edge. A \textit{star} $K_{1,n}$ on $n+1$ vertices is a tree with one center vertex of degree $n$ and $n$ leaves, i.e. vertices of degree $1$. A \textit{caterpillar} is a tree $T$ that has a subgraph $H\subseteq T$ such that $H$ is isomorphic to a path and $H$ contains all vertices in $V(T)$ of degree at least two.  %We refer to the leaves of $T$ that are not contained in $H$ as the legs of $T$. 
Figure \ref{f3ba} gives an example of all three types of trees.

\begin{figure}[ht]
\begin{center}
\includegraphics[scale=1]{trees}
\end{center}
\caption{A path $P$, a star $S$ and a caterpillar $T$}
\label{f3ba}
\end{figure}

A \textit{subdivision} of a graph $G=(V,E)$ is a graph resulting from a sequence of subdivisions of edges in $G$. Subdividing an edge $e=\lbrace u,v\rbrace\in E$ yields a graph $G^\prime =(V\dot{\cup} \lbrace w\rbrace, E^\prime )$ with $E^\prime =(E\setminus \lbrace e\rbrace ) \cup \lbrace \lbrace u,w\rbrace, \lbrace v,w \rbrace\rbrace$. Informally, we add a new vertex $w$ and replace $e$ with two new edges $\lbrace u,w\rbrace$ and $\lbrace w,v\rbrace$ as depicted in Figure \ref{f2ba}. Observe that a subdivision of a tree is still a tree.\\

\begin{figure}[ht]
\begin{center}
\includegraphics[scale=1]{subdivision}
\end{center}
\caption{Subdivision of an edge $e$ in a graph $G$}
\label{f2ba}
\end{figure}


For the sake of the completeness of this collection of definitions, we add a final one, since it is used to describe some trees already proven to be $\chi$-bounding. Again, let $G=(V,E)$ be a graph. Let the distance $dist(u,v)$ between two vertices $u,v\in V$ denote the length of the shortest path in $G$ starting at $u$ and ending at $v$. The eccentricity of a vertex $v\in V$ is defined as $\displaystyle e(v)=\max_{u \in V}\lbrace dist(v,u)\rbrace$. Then, the radius $\displaystyle r= \min_{v\in V}\lbrace e(v)\rbrace$ of $G$ is the minimum number $r\in\N$, such that there exists a vertex $v\in V$ where the distance between $v$ and any other vertex $u\in V$ is at most $r$. We often name this vertex $v$ the root of the tree.