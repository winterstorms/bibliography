Finding a counterexample for the Gyárfás-Sumner conjecture implies constructing a family $\mathcal{F}$ of graphs such that $\mathcal{F}\subseteq\textit{Forb}(T)$ for some tree $T$ but there is no bounding function for $\mathcal{F}$. Mycielski's construction of triangle-free graphs with high chromatic number \cite{My55} may provide such a family. The goal of this chapter is to eliminate the family of graphs obtained by the Mycielskian construction as candidate for disproving the Gyárfás-Sumner conjecture by showing that in each of these families we find infinitly many graphs containing an induced copy of $T$ for all trees $T$.

\begin{defn}
Let $G=(V,E)$ be a graph. Its \textit{Mycielskian graph} $\mu(G)$ is defined as the graph $G^\prime=(V\cup U\cup\{w\}, E^\prime)$, where $G^\prime [V]$ is isomorphic to $G$, the set $U$ contains a vertex $u_i$ for every vertex $v_i\in V$, $i\in [\vert V\vert]$, such that $V\cap U=\emptyset$, and $w\notin V\cup U$ is an additional vertex. In addition to the edges of the subgraph isomorphic to $G$, for each $u_i\in U$, $i\in [\vert V\vert]$, $E^\prime$ contains edges from $u_i$ to every vertex in $N_G(v_i)$ and an edge $\lbrace u_i ,w\rbrace$. Figure \ref{f1my} shows an example for the construction of a Mycielskian graph for a graph $G$.
\end{defn}

\begin{figure}[ht]
\begin{center}
\includegraphics[scale=1]{mycielskian}
\end{center}
\caption{Graph $G$ and its Mycielskian graph $\mu (G)$}
\label{f1my}
\end{figure}

Mycielski's construction provides a sequence of graphs $G_0, G_1, G_2,\dots $ starting with $G_0=K_2$, where $G_k = \mu (G_{k-1})$, $k\in\N$, and let $G_k=(V_k,E_k)$. Let us denote the family of graphs in this sequence as $\mathcal{M}$. See Figure \ref{f2my} for an illustration of the first three graphs of the sequence. 

\begin{figure}[ht]
\begin{center}
\includegraphics[scale=1]{mycielskian_sequence}
\end{center}
\caption{First three graphs of the graph sequence obtained by Mycielski's construction}
\label{f2my}
\end{figure}

First, we state some properties of a graph $G_k$. Let us denote the set of copies of the vertices in $V_{k-1}$ as $U_k$ and the one additional vertex as $w_k$. For each vertex $v\in V_{k-1}$ we refer to the vertex $u\in U_k$ that is adjacent to exactly the vertices in $N_{G_{k-1}}(v)$ as the copy of $v$ in $G_k$. By the construction, the set $U_k$ is a stable set. Thus, together with the vertex $w_k$ the vertices in $U_k$ induce a star $K_{1, \vert U_k\vert}$ in $G_k$. Furthermore, observe that the neighborhood of each vertex we add, is a stable set.

With these facts in mind, we obtain the following two theorems which Myvielski proved as part of his construction.

\begin{thm}[Mycielski \cite{My55}]\label{t2my}
All graphs in $\mathcal{M}=\{G_k:k\in\Z, k\geq 0\}$ are triangle-free.
\end{thm}
\begin{prf}
We prove this by induction on $k$. The graph $G_0=K_2$ trivially does not contain any triangles. Now, let $k\geq 1$. Then $G_{k-1}$ is triangle-free. By the specifications of the construction the neighborhood of each vertex we add is a stable set: The vertex $w_k$ is only adjacent to vertices in $U_k$ which is a stable set. And the neighbors in $V_k$ of each vertex in $U_k$ must form a stable set as well since otherwise they would induce a triangle in $G_{k-1}$. Thus, no vertex we add creates a triangle in $G_k$.\qed 
\end{prf}

\begin{thm}[Mycielski \cite{My55}]\label{t3my}
For all $k\in\Z$, $k\geq 0$, the chromatic number of $G_k$ is at least $k$.
\end{thm}
\begin{prf}
We prove this also by induction on $k$. The graph $G_0=K_2$ has chromatic number $\chi (K_2)=2 \geq 0$. Let $k\geq 1$. We shall show that $G_k$ is not properly colorable with $k-1$ colors. By induction hypothesis, we know that there exists no proper coloring of $G_{k-1}$ that uses only $k-2$ colors. Thus, for every coloring $c$ of $G_k$ with $k-1$ colors there exists a vertex set $S\subseteq V_{k-1}$ of $k-1$ vertices such that the vertices have pairwise different colors and are adjacent to a vertex of each of the other colors. Then, for each vertex $v\in S$ the vertex in $U_k$ that is adjacent to the neighbors of $s$ must have the same color as $s$. Thus, for each of the $k-1$ colors, there exists a vertex in $U_k$ with this color. The vertex $w_k$ must be colored in a $k^{th}$ color which is a contradiction to the assumption that $c$ uses only $k-1$ colors.\qed
\end{prf}

Next, observe that if a graph $G_k$, $k\in\N$, contains an induced copy of a graph $H$, then all graphs of the sequence with a higher index contain an induced copy of $H$ as well.

Last, note that this construction indeed provides a family of graphs worthy to look at for disproving Conjecture \ref{c1cr}. As we have shown in Theorem \ref{t2my} and Theorem \ref{t3my}, $\mathcal{M}$ contains triangle-free graphs of arbitrarily high chromatic number. Thus, if it were possible to prove that $\mathcal{M}\subseteq\textit{Forb}(T)$, that would disprove Conjecture \ref{c1cr}. The first idea is to simply apply Theorem \ref{t2cr} to prove that $\mathcal{M}\subseteq\textit{Forb}(T)$. That is not possible as the following lemma shows since for all $n\in\N$ there exists a graph $G_k\in\mathcal{M}$ that contains an induced subgraph isomorphic to $K_{n,n}$. Thus, we need to have a closer look at the construction to decide whether $\mathcal{M}\subseteq\textit{Forb}(T)$. 

\begin{lemma}\label{l1my}
For all $n\in\N$ there exists a graph $G\in\mathcal{M}$ that contains an induced subgraph isomorphic to $K_{n,n}$.
\end{lemma}
\begin{prf}
Consider a graph $G_k\in\mathcal{M}$ such that $U_k$ has size at least $n$. Choose a subset $X\subseteq U_k$ of size $n$. Then, the vertex $w_k$ is adjacent to every vertex in $X$. Let $w_k^{i}$, $i\in [n]$, denote copy of $w_k$ in $U_{k+i}$. Then, each $w_k^{i}$, $i\in [n]$, is adjacent to the neighbors of $w_k$, including all vertices in $X$. Look at Figure \ref{f3my} for a visualization of the chosen vertices. The vertices $w_k^{i}$, $i\in [n]$, form a stable set $S$ in $G_{k+n}$ since $w_k$ is not adjacent to any of them. Thus, the vertices in $S\cup X$ induce a $K_{n,n}$ in $G_{k+n}$.\qed
\end{prf}

\begin{figure}[ht]
\begin{center}
\includegraphics[scale=1]{induced_knn}
\end{center}
\caption{Part of an induced subgraph of $G_{k+n}$ isomorphic to a $K_{n,n}$}
\label{f3my}
\end{figure}

As mentioned at the beginning of this chapter, our goal is to show that for each tree $T$, there exists a graph $G_k\in\mathcal{M}$ containing an induced subgraph isomorphic to $T$. Then, all graphs in $\mathcal{M}$ with a higher index contain an induced copy of $T$ as well and thus, it is not possible to use $\mathcal{M}$ aas counterexample for Conjecture \ref{c1cr}.
We start with two proofs by induction proving the existence of induced subgraphs isomorphic to paths and caterpillars in graphs of $\mathcal{M}$. For $k\in\N$ denote the copy of $w_k$ in $U_{k+1}$ as $w_k^1$ as in the proof of Lemma \ref{l1my} . Let $Q_n =(w_1^1 ,w_2^1 ,\dots ,w_n^1)$ be the path through the first $n$ such vertices. 
\begin{thm}\label{t1my}
For all $n\in\N$, $Q_n$ is an induced subgraph of $G_{n+1}$.
\end{thm}
\begin{prf}
We prove this theorem by induction on $n$. To launch the induction let $n=1$. The vertex $w_1^1$ trivially is a path on one vertex in the graph $G_2$.

Now let $n\geq 2$. By induction hypothesis, $Q_{n-1}$ is an induced subgraph of $G_n$. Then, the path $Q_{n-1}$ is also an induced subgraph of $G_{n+1}$. 
Now consider $w_n$. It is adjacent to any vertex in $U_n$ - including $w^1_{n-1}$ - but no other vertex, in particular no $w^1_i$, $i\in [n-2]$. Thus, as the copy of $w_n$ in $U_{n+1}$, $w^1_n$ is only adjacent to $w^1_{n-1}$ and no other vertex of $Q_{n-1}$. Thus, $Q_n$ is an induced subgraph of $G_{n+1}$.\qed
\end{prf}

\begin{cor}\label{c1my}
The graph $G_{n+1}$ contains an induced subgraph isomorphic to $P_n$ for all $n\in\N$.
\end{cor}
\begin{prf}
As a path on $n$ vertices $Q_n$ is isomorphic to $P_n$ and $Q_n\subseteq_I G_{n+1}$ by Theorem \ref{t1my}.\qed
\end{prf}

In the next proof we will use certain subgraphs of $Q_n$ to construct an induced copy of a caterpillar in some $G\in\mathcal{M}$. A \textit{caterpillar} is a tree $T$ with a subgraph $P$ isomorphic to a path that contains all vertices in $T$ with degree at least two. Then, $T$ also contains a subgraph isomorphic to a path that contains exactly the vertices in $T$ of degree at least $T$. That is since a vertex of degree one can only be an endpoint of $P$ and considering the subpath of $P$ without the endpoints if they are leaves yields a path as wanted.

Also, let $Q_{i,j}=(w_i^1 ,\dots , w_j^1 )$, $1\leq i\leq j\leq n$, denote the subpath of $Q_n$ with endpoints $w_i^\prime$ and $w_j^\prime$.

\begin{thm}
Let $T$ be a caterpillar. Then, there is a $G\in\mathcal{M}$ containing an induced subgraph isomorphic to $T$.
\end{thm}
\begin{prf}
Let $T$ be a caterpillar with maximum degree $d:=\Delta (T)$. Let $P=(p_1, p_2, \dots , p_n)$ denote the path in $T$ containing exactly the vertices in $T$ of degree at least two and let $n$ denote the number of vertices in $P$. Note that if such a $P$ does not exist, all vertices in $T$ have degree one. Then, $T$ is a path on one or two vertices and therefore $G_0=K_2$ trivially contains an induced subgraph isomorphic to $T$.

Suppose $P$ exists. All vertices with degree one have to be adjacent to a vertex in $P$ since $T$ is connected. If $d\leq 2$, $T$ is a path and Corollary \ref{c1my} then states, that it is an induced subgraph of $G_{n+1}$.

If not, we have $d>2$. Every vertex in $P$ is adjacent to no more than $d$ leaves of $T$. Now, let $G_k$ be a graph of size at least $d$ and consider $G_{k+n+1}$. By Theorem \ref{t1my}, the path $Q_{k+n}=(w^1_1,w^1_2,...,w^1_{k+n})\subseteq_I G_{k+n+1}$. Thus, $Q_{k+1,k+n}$ is an induced path on $n$ vertices in $G_{k+n+1}$. Let $N_1:=U_{k+1}$ and for $1<l\leq n$ let $N_l\subseteq U_{k+l}$ denote the copies in $U_{k+l}$ of the vertices in $N_{l-1}$. Then, no vertex from $\bigcup_{l\in [n]}N_l$ is also a vertex of $Q_{k+1,k+n}$. Observe, that $\vert N_l \vert \geq d$ for all $l\in [n]$. We then claim that the vertices in $S_i:=V(Q_{k+1,k+i})\cup\bigcup_{l=1}^i N_l$ induce a tree in $G_{k+n+1}$. We prove this by induction on $i$.

For the base case let $i=1$. The vertices in $N_1=U_{k+1}$ form a stable set  and are adjacent to $w_{k+1}$ by definition of $U_{k+1}$. Therefore, they are adjacent to $w^1_{k+1}$ as well. Thus, $S_1$ induces a subgraph isomorphic to a star $K_{1,\vert N_1\vert}$, i.e. $G_{k+n}[S_1]$ is an induced tree.

Now let $1<i\leq n$. By the induction hypothesis, $S_{i-1}$ is an induced tree, i.e. from the vertices in $S_{i-1}$ the vertices in $N_{i-1}$ are only adjacent to $w_{k+i-1}^1$. Observe that $(S_{i-1}\setminus \lbrace w_{k+i-1}^1\rbrace )\subseteq V_{k+i-1}$. Consider $N_i\subseteq U_{k+i}$. By the definition of vertices in $U_{k+i}$, vertices in $N_i$ do not have an edge to any vertex from $S_{i-1}\setminus \lbrace w_{k+i-1}^1\rbrace$ since the vertices in $N_{i-1}$ do not have them. Furthermore, $N_i\cup\lbrace w_{k+i-1}^1\rbrace \subseteq U_{k+i}$ is a stable set by definition but each vertex in this set has an edge to $w_{k+i}$ and therefore $w_{k+i}^1$. What is left to show, is that $w_{k+i}^1$ is not adjacent to any vertex in $S_{i-1}\setminus \lbrace w_{k+i-1}^1\rbrace$. That is easy to see, since $(S_{i-1}\setminus \lbrace w_{k+i-1}^1\rbrace )\cap U_{k+i}=\emptyset$. Thus, the vertices in $S_i$ induce a tree in $G_{k+i+1}$.

By mapping $p_j$ to the vertices of $w^1_{k+j}$ and the leaves adjacent to the vertex $p_j$ to a vertex from $N_j$, $j\in [n]$, we obtain an induced subgraph of $G_{k+n+1}$ that is isomorphic to $T$.\qed
\end{prf}

After stating the proofs for some special cases of trees, there is actually a slightly different approach that proves that for all trees $T$ we can find a graph $G\in\mathcal{M}$ containing an induced subgraph isomorphic to $T$. Therefore, we need to introduce the concept of a rooted tree first. A \textit{rooted} tree $T$ is a tree in which one vertex $r\in T$ is set to be the \textit{root} of $T$. Then, the \textit{parent} of a vertex $v\in T$ is the vertex adjacent to it in the path from $v$ to the root. All vertices of $T$ except the root have a unique parent. We call a vertex $u\in T$ a \textit{child} of $v$ if $v$ is its parent.

Second, we want to order the vertices $v_1, v_2, \dots , v_n$ of a rooted tree on $n$ vertices in a certain way. An order of the vertices is called \textit{in-order} if it can be obtained by the following procedure. Let $v_1$ be the root of $T$ and for each $i\in [n-1]$ choose $v_{i+1}$ as a child of $v_i$ that has not been assigned yet. If there is no such child left, consider the previous vertices in decreasing order and choose a child of them if possible. Observe that each vertex except $v_1$ is a child of a vertex with a smaller index. An example of a rooted tree with an inorder vertex ordering is displayed in Figure \ref{f4my}.\\


\begin{figure}[ht]
\begin{center}
\includegraphics[scale=1]{inorder}
\end{center}
\caption{Rooted tree $T$ on $11$ vertices with root $r$ and vertex order that is inorder}
\label{f4my}
\end{figure}


\begin{thm}
Let $T$ be tree. Then, there exists a $G\in\mathcal{M}$ containing an induced subgraph isomorphic to $T$.
\end{thm}
\begin{prf}
Let $n$ be the number of vertices in $T$ and choose a vertex $r$ as the root of $T$. Denote the vertices as $r=v_1, v_2, \dots , v_n$ such they are in-order.

We shall construct an induced subgraph $H$ isomorphic to $T$ with vertices $h_1, h_2, \dots , h_n$. In preparation of the construction, note that each vertex $h_i$ will be a copy in a set $U_{j_i}$, $j_i\in\N$, such that the sets $U_{j_i}$, $i\in [n]$, are pairwise disjoint.

We construct $H$ by choosing the root vertex $v_1$ as a copy from $U_1$ and inductively adding the other vertices in their order in the following way. Each vertex $v_i$, $i>1$, has a unique parent $v_k$ with $k<i$. The parent $v_k$ is element of the set $U_{j_k}$ and the vertex $v_{i-1}$ is element of the set $U_{j_{i-1}}$. Let vertex $h_i$ be the copy of $w_{j_k}$ in $U_m$ such that $m> j_{i-1}$. Since $w_{j_k}$ and $h_k$ are adjacent by definition, as copy of $w_{j_k}$ $h_i$ is adjacent to $c_k$ as well. Thus, for every edge in $T$, there exists an edge in $H$. Observe that all vertices $h_1, h_2, \dots , h_n$ are copies in different sets $U_i$, $i\in [n]$. 

We need to confirm that $H$ is indeed induced. We prove this by induction on $i$, claiming that the vertices in $H_i:=\lbrace h_1, h_2, \dots , h_i\rbrace$ induce a tree in $G_{j_i}$. To launch the induction, let $i=1$. One vertex by itself induces no edge. Thus, it trivially induces a tree in $G_m$.

Let $1<i\leq n$. By the induction hypothesis, the vertices in $H_{i-1}$ induce a tree in $G_{j_{i-1}}$ and therefore also in $G_{j_i}$ since $j_i >j_{i-1}$ and $G_{j_i}$ then contains an induced subgraph isomorphic to $G_{j_{i-1}}$. The vertex $h_i$ is a copy of some $w_{j_k}$, $k<i$. The vertex $w_{j_k}$ is only adjacent to vertices $U_{j_k}$, i.e. it is not adjacent to any vertex in $\lbrace h_1, h_2, \dots , h_{k-1}\rbrace$. Therefore, $h_i$ is also not adjacent to them. Since vertices $w_x$ and $w_y$, $x,y\in\N$, $x\neq y$, are never adjacent by definition, no vertex in $h_l\in\lbrace h_{k+1}, \dots , h_{i-1}\rbrace$ chosen after $h_k$ is, as a copy of some $w_x$, such that the parent of $h_l$ is element of $U_x$, adjacent to $w_{j_k}$. Therefore, $h_i$ is only adjacent to $h_k$ as wanted and the vertices $\lbrace h_1, h_2, \dots , h_i\rbrace$ induce a tree in $G_{j_i}$.

As we observed at the beginning $H=G_{j_i}[H_n]$ contains all edges that correspond to the edges present in $T$. It does not contain any other edges since then $H$ were not an induced tree anymore. Thus, $H\subseteq_I G_{j_i}$ is isomorphic to $T$.\qed
\end{prf}
